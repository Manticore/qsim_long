% =============================================================================
\section{Functional Wigner representation}
% =============================================================================

First, we will define functional analogue of the displacement operator from \defref{sm-wigner:displacement-op}:

\begin{definition}
	\begin{eqn*}
		\hat{D} :: \mathbb{F}_L \rightarrow	\mathbb{H}_L \\
		\hat{D}[\Lambda, \Lambda^*] = \exp \int d\xvec \left(
			\Lambda(\xvec) \Psiop^\dagger(\xvec) - \Lambda^*(\xvec) \Psiop(\xvec)
		\right).
	\end{eqn*}
	It is convenient to also define displacement functional as
	\begin{eqn*}
		D :: \mathbb{F}_L \rightarrow \mathbb{F}_L \rightarrow \mathbb{C} \\
		D[\Lambda, \Lambda^*, \Psi, \Psi^*] = \exp \int d\xvec \left(
			-\Lambda(\xvec) \Psi^*(\xvec) + \Lambda^*(\xvec) \Psi(\xvec)
		\right).
	\end{eqn*}
\end{definition}

It can be shown that the displacement operator has properties similar to~\eqnref{sm-wigner:displacement-derivatives}.

\begin{lemma}
\label{lmm:func-wigner:displacement-derivatives}
	For any $\Lambda \in \mathbb{F}_L$:
	\begin{eqn*}
		\frac{\delta}{\delta \Lambda^\prime} \hat{D}[\Lambda, \Lambda^*]
		= \hat{D}[\Lambda, \Lambda^*] (\Psiop^{\prime\dagger} + \frac{1}{2} \Lambda^{\prime*})
		= (\Psiop^{\prime\dagger} - \frac{1}{2} \Lambda^{\prime*}) \hat{D}[\Lambda, \Lambda^*], \\
		-\frac{\delta}{\delta \Lambda^{\prime*}} \hat{D}[\Lambda, \Lambda^*]
		= \hat{D}(\Lambda, \Lambda^*) (\Psiop^\prime + \frac{1}{2} \Lambda^\prime)
		= (\Psiop^\prime - \frac{1}{2} \Lambda^\prime) \hat{D}[\Lambda, \Lambda^*].
	\end{eqn*}
\end{lemma}
\begin{proof}
Proved using Baker-Hausdorff theorem and evaluating integrals.
\end{proof}

\begin{definition}
	Functional Wigner transformation $\mathcal{W}$ is defined as
	\begin{eqn*}
	\label{eqn:func-wigner:w-transformation}
		\mathcal{W} :: \mathbb{FH}_L \rightarrow (\mathbb{F}_L \rightarrow \mathbb{C}) \\
		\mathcal{W}[\hat{A}]
		= \frac{1}{\pi^{2N}} \int \delta^2 \Lambda
			D[\Lambda, \Lambda^*, \Psi, \Psi^*]
			\Trace{ \hat{A} \hat{D}[\Lambda, \Lambda^*] }.
	\end{eqn*}
	It transforms an operator $\hat{A}$ on a restricted Hilbert space to a functional $(\mathcal{W}[\hat{A}])[\Psi, \Psi^*]$.
\end{definition}

\begin{definition}
\label{def:func-wigner:w-functional}
	The Wigner functional is
	\begin{eqn*}
		W :: \mathbb{F}_L \rightarrow \mathbb{C} \\
		W [\Psi, \Psi^*]
		\equiv \mathcal{W}[\hat{\rho}]
		= \frac{1}{\pi^{2N}} \int \delta^2 \Lambda
			D[\Lambda, \Lambda^*, \Psi, \Psi^*]
			\chi_W [\Lambda, \Lambda^*],
	\end{eqn*}
	where $\chi_W [\Lambda, \Lambda^*]$ is the characteristic functional
	\begin{eqn*}
		\chi_W :: \mathbb{F}_L \rightarrow \mathbb{R} \\
		\chi_W [\Lambda, \Lambda^*]
		= \Trace{ \hat{\rho} \hat{D}[\Lambda, \Lambda^*] }.
	\end{eqn*}
\end{definition}

\begin{lemma}[Functional extension of \lmmref{sm-wigner:zero-integrals}]
\label{lmm:func-wigner:zero-integrals}
	If $\Lambda \in \mathbb{F}_L$ and $\Psi \in \mathbb{F}_L$:
	\begin{eqn*}
		\int \delta^2\Lambda
			\frac{\delta}{\delta \Lambda^\prime} \left(
				D[\Lambda, \Lambda^*, \Psi, \Psi^*]
				\left( \frac{\delta}{\delta \Lambda^\prime} \right)^r
				\left( -\frac{\delta}{\delta \Lambda^{\prime*}} \right)^s
				\hat{D}[\Lambda, \Lambda^*]
			\right)
		= 0, \\
		\int \delta^2\Lambda
			\frac{\delta}{\delta \Lambda^{\prime*}}
			\left(
				D[\Lambda, \Lambda^*, \Psi, \Psi^*]
				\left( \frac{\delta}{\delta \Lambda^\prime} \right)^r
				\left( -\frac{\delta}{\delta \Lambda^{\prime*}} \right)^s
				\hat{D}[\Lambda, \Lambda^*]
			\right)
		= 0.
	\end{eqn*}
\end{lemma}
\begin{proof}
Displacement operator and displacement functions can be represented as functions of vectors:
\begin{eqn}
	\hat{D}[\Lambda, \Lambda^*]
	= \prod_{\nvec \in L} \exp \left(
		\lambda_{\nvec} \hat{a}_{\nvec}^\dagger - \lambda_{\nvec}^* \hat{a}_{\nvec}
	\right),
\end{eqn}
\begin{eqn}
	D[\Lambda, \Lambda^*, \Psi, \Psi^*]
	= \prod_{\nvec \in L} \exp
		(-\lambda_{\nvec} \alpha_{\nvec}^* + \lambda_{\nvec}^* \alpha_{\nvec}),
\end{eqn}
The proof consists of substituting these in the equations from the statement and applying \lmmref{c-numbers:zero-integrals}.
\end{proof}

\begin{lemma}[Functional extension of \lmmref{sm-wigner:moments-from-chi}]
\label{lmm:func-wigner:moments-from-chi}
	\begin{eqn*}
		\langle \symprod{ (\Psiop^\prime)^r (\Psiop^{\prime\dagger})^s } \rangle
		= \left.
			\left( \frac{\delta}{\delta \Lambda^\prime} \right)^s
			\left( -\frac{\delta}{\delta \Lambda^{\prime*}} \right)^r
			\chi_W [\Lambda, \Lambda^*]
		\right|_{\Lambda \equiv 0}.
	\end{eqn*}
\end{lemma}
\begin{proof}
The proof follows the same general scheme from the single-mode case.
The displacement operator can be expanded as
\begin{eqn}
	\exp (\Lambda \Psiop^\dagger - \Lambda^* \Psiop)
	= \sum_{r,s}
		\frac{
			\symprod{
				\left( \int d\xvec \Lambda \Psiop^\dagger \right)^r
				\left( -\int d\xvec \Lambda^* \Psiop \right)^s
			}
		}
		{r!s!}.
\end{eqn}
We can swap functional derivative with both integration and multiplication by independent function, so:
\begin{eqn}
	\frac{\delta}{\delta \Lambda^\prime} \left( \int d\xvec \Lambda \Psiop^\dagger \right)^r
	= r \Psiop^{\prime\dagger} \left( \int d\xvec \Lambda \Psiop^\dagger \right)^{r-1},
\end{eqn}
and multiple application of the differential gives us
\begin{eqn}
	\left( \frac{\delta}{\delta \Lambda^\prime} \right)^r
	\left( \int d\xvec \Lambda \Psiop^\dagger \right)^r
	= r! ( \Psiop^{\prime\dagger} )^r.
\end{eqn}
Similarly for the other differential:
\begin{eqn}
	\left( -\frac{\delta}{\delta \Lambda^{\prime*}} \right)^s
	\left( -\int d\xvec \Lambda \Psiop^\dagger \right)^s
	= s! ( \Psiop^{\prime\dagger} )^s.
\end{eqn}

Thus, same as in single-mode case,
differentiation will eliminate all lower order terms in the expansion,
and all higher order terms will be eliminated by setting $\Lambda \equiv 0$,
leaving only one operator product with required order.
\end{proof}

\begin{theorem}[Functional extension of \thmref{sm-wigner:moments}]
\label{thm:func-wigner:moments}
	\begin{eqn*}
		\langle \symprod{ \Psiop^r (\Psiop^\dagger)^s } \rangle
		= \int \delta^2\Psi\, \Psi^r (\Psi^*)^s W[\Psi, \Psi^*]
	\end{eqn*}
\end{theorem}
\begin{proof}
By definition of Wigner functional:
\begin{eqn}
	\int \delta^2\Psi\, \Psi^r (\Psi^*)^s W[\Psi, \Psi^*] \\
	= \frac{1}{\pi^{2N}} \Trace{ \hat{\rho}
		\int \delta^2\Psi\, \Psi^r (\Psi^*)^s
		\int \delta^2\Lambda D[\Lambda, \Lambda^*, \Psi, \Psi^*]
		\hat{D}[\Lambda, \Lambda^*]
	}
\end{eqn}
Integrating by parts and eliminating terms which fit \lmmref{func-wigner:zero-integrals}:
\begin{eqn}
\fl	= \frac{1}{\pi^{2N}} \Trace{ \hat{\rho}
		\int \delta^2\Psi \int \delta^2\Lambda
		D[\Lambda, \Lambda^*, \Psi, \Psi^*]
		\left( \frac{\delta}{\delta \Lambda} \right)^s
		\left( -\frac{\delta}{\delta \Lambda^*} \right)^r
		\hat{D}[\Lambda, \Lambda^*]
	}
\end{eqn}
Evaluating integral over $\Psi$ using \lmmref{func-calculus:fourier-of-moments}:
\begin{eqn*}
	& = \int \delta^2\Lambda\,
		\Delta[\Lambda]
		\left( \frac{\delta}{\delta \Lambda} \right)^s
		\left( -\frac{\delta}{\delta \Lambda^*} \right)^r
		\Trace{
			\hat{\rho}
			\hat{D}[\Lambda, \Lambda^*]
		} \\
	& = \left.
		\left( \frac{\delta}{\delta \Lambda} \right)^s
		\left( -\frac{\delta}{\delta \Lambda^*} \right)^r
		\chi_W [\Lambda, \Lambda^*]
	\right|_{\Lambda \equiv 0}.
\end{eqn*}
Now, recognising the final expression as a part of \lmmref{func-wigner:moments-from-chi},
we immediately get the statement of the theorem.
\end{proof}

\begin{theorem}[Functional extension of \thmref{sm-wigner:correspondences}]
\label{thm:func-wigner:correspondences}
	\begin{eqn*}
		\mathcal{W} [ \Psiop \hat{A} ]
			& = \left( \Psi + \frac{1}{2} \frac{\delta}{\delta \Psi^*} \right) \mathcal{W}[\hat{A}],
		\quad
		\mathcal{W} [ \Psiop^\dagger \hat{A} ]
			= \left( \Psi^* - \frac{1}{2} \frac{\delta}{\delta \Psi} \right) \mathcal{W}[\hat{A}], \\
		\mathcal{W} [ \hat{A} \Psiop ]
			& = \left( \Psi - \frac{1}{2} \frac{\delta}{\delta \Psi^*} \right) \mathcal{W}[\hat{A}],
		\quad
		\mathcal{W} [ \hat{A} \Psiop^\dagger ]
			= \left( \Psi^* + \frac{1}{2} \frac{\delta}{\partial \Psi} \right) \mathcal{W}[\hat{A}].
	\end{eqn*}
\end{theorem}
\begin{proof}
The proof follows the same scheme as in the single-mode case, with \lmmref{func-wigner:displacement-derivatives} used to transform the $\hat{A}\hat{D}$ product inside the trace, and \lmmref{func-wigner:zero-integrals} to move the differentials.
\end{proof}

\thmref{func-wigner:moments} and \thmref{func-wigner:correspondences} can be further extended to the case of several components.
They can be proved in exactly the same way.

\begin{theorem}[Multi-component extension of \thmref{func-wigner:moments}]
\label{thm:func-wigner:mc-moments}
	If $C$ is the number of components, then
	\begin{eqn*}
		\int \delta^2\Psi_1 \ldots \int \delta^2\Psi_C\,
			\prod_{c=1}^C \Psi_c^{r_c} (\Psi_c^*)^{s_c} W[\Psivec, \Psivec^*]
		= \langle \symprod{ \prod_{c=1}^C \Psiop_c^{r_c} (\Psiop_c^\dagger)^{s_c} } \rangle.
	\end{eqn*}
\end{theorem}

\begin{theorem}[Multi-component extension of \thmref{func-wigner:correspondences}]
\label{thm:func-wigner:mc-correspondences}
\begin{eqn*}
	\mathcal{W} [ \Psiop_c \hat{A} ]
		& = \left( \Psi_c + \frac{1}{2} \frac{\delta}{\delta \Psi_c^*} \right) \mathcal{W}[\hat{A}],
	\quad
	\mathcal{W} [ \Psiop_c^\dagger \hat{A} ]
		= \left( \Psi_c^* - \frac{1}{2} \frac{\delta}{\delta \Psi_c} \right) \mathcal{W}[\hat{A}], \\
	\mathcal{W} [ \hat{A} \Psiop_c ]
		& = \left( \Psi_c - \frac{1}{2} \frac{\delta}{\delta \Psi_c^*} \right) \mathcal{W}[\hat{A}],
	\quad
	\mathcal{W} [ \hat{A} \Psiop_c^\dagger ]
		= \left( \Psi_c^* + \frac{1}{2} \frac{\delta}{\partial \Psi_c} \right) \mathcal{W}[\hat{A}].
\end{eqn*}
\end{theorem}
