% =============================================================================
\section{Functional Wigner representation}
% =============================================================================

The single-mode Wigner transformation of the operator $\hat{A}$ is defined as
\begin{eqn}
	\mathcal{W}_{\mathrm{sm}}[\hat{A}]
	= \frac{1}{\pi^2} \int d^2 \lambda \exp(-\lambda \alpha^* + \lambda^* \alpha)
		\Trace{ \hat{A} \hat{D}(\lambda, \lambda^*) },
\end{eqn}
where the displacement operator $\hat{D}(\lambda, \lambda^*) = \exp(\lambda \hat{a}^\dagger - \lambda^* \hat{a})$ was first introduced by Weyl~\cite{Weyl1950}.
The detailed description of the Wigner function $W(\alpha, \alpha^*) \equiv \mathcal{W}_{\mathrm{sm}}[\hat{\rho}]$ can be found in~\cite{Gardiner2004}.
In this section we will extend this definition to the multimode case.

The important part of the definition is the functional analogue of the displacement operator.
\todo{Need to explicitly state that $F^\prime \equiv F(\xvec^\prime)$. Also, explain why we write $F[\Lambda, \Lambda^*]$ and not $F[\Lambda]$.}

\begin{definition}
    Functional displacement operator
	\begin{eqn*}
		\hat{D} :: \mathbb{F}_{\restbasis} \rightarrow \mathbb{H}_{\restbasis} \\
		\hat{D}[\Lambda, \Lambda^*] = \exp \int d\xvec \left(
			\Lambda(\xvec) \Psiop^\dagger(\xvec) - \Lambda^*(\xvec) \Psiop(\xvec)
		\right).
	\end{eqn*}
	It is also convenient to define the displacement functional as
	\begin{eqn*}
		D :: \mathbb{F}_{\restbasis} \rightarrow \mathbb{F}_{\restbasis} \rightarrow \mathbb{C} \\
		D[\Lambda, \Lambda^*, \Psi, \Psi^*] = \exp \int d\xvec \left(
			-\Lambda(\xvec) \Psi^*(\xvec) + \Lambda^*(\xvec) \Psi(\xvec)
		\right).
	\end{eqn*}
\end{definition}

It can be shown that the functional displacement operator has properties similar to its single-mode equivalent.

\begin{lemma}
\label{lmm:func-wigner:displacement-derivatives}
	\begin{eqn*}
		\frac{\delta}{\delta \Lambda^\prime} \hat{D}[\Lambda, \Lambda^*]
		= \hat{D}[\Lambda, \Lambda^*] (\Psiop^{\prime\dagger} + \frac{1}{2} \Lambda^{\prime*})
		= (\Psiop^{\prime\dagger} - \frac{1}{2} \Lambda^{\prime*}) \hat{D}[\Lambda, \Lambda^*], \\
		-\frac{\delta}{\delta \Lambda^{\prime*}} \hat{D}[\Lambda, \Lambda^*]
		= \hat{D}(\Lambda, \Lambda^*) (\Psiop^\prime + \frac{1}{2} \Lambda^\prime)
		= (\Psiop^\prime - \frac{1}{2} \Lambda^\prime) \hat{D}[\Lambda, \Lambda^*].
	\end{eqn*}
\end{lemma}
\begin{proof}
Proved using Baker-Hausdorff theorem and evaluating integrals.
\end{proof}

\begin{definition}
\label{eqn:func-wigner:w-transformation}
	Functional Wigner transformation $\mathcal{W}$ is defined as
	\begin{eqn*}
		\mathcal{W} :: \mathbb{FH}_{\restbasis} \rightarrow (\mathbb{F}_{\restbasis} \rightarrow \mathbb{C}) \\
		\mathcal{W}[\hat{A}]
		= \frac{1}{\pi^{2|\restbasis|}} \int \delta^2 \Lambda
			D[\Lambda, \Lambda^*, \Psi, \Psi^*]
			\Trace{ \hat{A} \hat{D}[\Lambda, \Lambda^*] }.
	\end{eqn*}
	It transforms an operator $\hat{A}$ on a restricted Hilbert space to a functional $(\mathcal{W}[\hat{A}])[\Psi, \Psi^*]$.
	By analogy with the single-mode case, the Wigner functional is
	\begin{eqn*}
		W :: \mathbb{F}_{\restbasis} \rightarrow \mathbb{C} \\
		W [\Psi, \Psi^*]
		\equiv \mathcal{W}[\hat{\rho}]
		= \frac{1}{\pi^{2|\restbasis|}} \int \delta^2 \Lambda
			D[\Lambda, \Lambda^*, \Psi, \Psi^*]
			\chi_W [\Lambda, \Lambda^*],
	\end{eqn*}
	where $\chi_W [\Lambda, \Lambda^*]$ is the characteristic functional
	\begin{eqn*}
		\chi_W :: \mathbb{F}_{\restbasis} \rightarrow \mathbb{R} \\
		\chi_W [\Lambda, \Lambda^*]
		= \Trace{ \hat{\rho} \hat{D}[\Lambda, \Lambda^*] }.
	\end{eqn*}
\end{definition}

The multi-mode Wigner function has two important properties analogous to the single-mode case.
First one is used to successively transform operator products.
In order to prove it, we will need an auxiliary lemma.

\begin{lemma}
\label{lmm:func-wigner:zero-integrals}
	\begin{eqn*}
		\int \delta^2\Lambda
			\frac{\delta}{\delta \Lambda^\prime} \left(
				D[\Lambda, \Lambda^*, \Psi, \Psi^*]
				\left( \frac{\delta}{\delta \Lambda^\prime} \right)^r
				\left( -\frac{\delta}{\delta \Lambda^{\prime*}} \right)^s
				\hat{D}[\Lambda, \Lambda^*]
			\right)
		= 0, \\
		\int \delta^2\Lambda
			\frac{\delta}{\delta \Lambda^{\prime*}}
			\left(
				D[\Lambda, \Lambda^*, \Psi, \Psi^*]
				\left( \frac{\delta}{\delta \Lambda^\prime} \right)^r
				\left( -\frac{\delta}{\delta \Lambda^{\prime*}} \right)^s
				\hat{D}[\Lambda, \Lambda^*]
			\right)
		= 0.
	\end{eqn*}
\end{lemma}
\begin{proof}
Displacement operator and displacement functional can be represented as functions of vectors:
\begin{eqn}
	\hat{D}[\Lambda, \Lambda^*]
	= \prod_{\nvec \in \restbasis} \exp \left(
		\lambda_{\nvec} \hat{a}_{\nvec}^\dagger - \lambda_{\nvec}^* \hat{a}_{\nvec}
	\right),
\end{eqn}
\begin{eqn}
	D[\Lambda, \Lambda^*, \Psi, \Psi^*]
	= \prod_{\nvec \in \restbasis} \exp
		(-\lambda_{\nvec} \alpha_{\nvec}^* + \lambda_{\nvec}^* \alpha_{\nvec}),
\end{eqn}
The proof consists of substituting these in the equations from the statement and applying \lmmref{c-numbers:zero-integrals}.
\end{proof}

\begin{theorem}[Functional correspondences]
\label{thm:func-wigner:correspondences}
    If $\mathcal{W} [ \hat{A} ] \equiv (\mathcal{W} [ \hat{A} ]) [\Psi, \Psi^*]$, then
	\begin{eqn*}
		\mathcal{W} [ \Psiop \hat{A} ]
			& = \left( \Psi + \frac{1}{2} \frac{\delta}{\delta \Psi^*} \right) \mathcal{W}[\hat{A}],
		\quad
		\mathcal{W} [ \Psiop^\dagger \hat{A} ]
			= \left( \Psi^* - \frac{1}{2} \frac{\delta}{\delta \Psi} \right) \mathcal{W}[\hat{A}], \\
		\mathcal{W} [ \hat{A} \Psiop ]
			& = \left( \Psi - \frac{1}{2} \frac{\delta}{\delta \Psi^*} \right) \mathcal{W}[\hat{A}],
		\quad
		\mathcal{W} [ \hat{A} \Psiop^\dagger ]
			= \left( \Psi^* + \frac{1}{2} \frac{\delta}{\partial \Psi} \right) \mathcal{W}[\hat{A}].
	\end{eqn*}
\end{theorem}
\begin{proof}
The proof uses \lmmref{func-wigner:displacement-derivatives} to transform the $\hat{A}\hat{D}$ product inside the trace, and \lmmref{func-wigner:zero-integrals} to integrate by parts, effectively moving the differentials to intended places.
\end{proof}

The second property complements the first one, providing the way to obtain expectations of operator products given the Wigner function.
Again, it requires a supplementary lemma.

\begin{lemma}
\label{lmm:func-wigner:moments-from-chi}
    For any non-negative integer $r$ and $s$:
	\begin{eqn*}
		\langle \symprod{ (\Psiop^\prime)^r (\Psiop^{\prime\dagger})^s } \rangle
		= \left.
			\left( \frac{\delta}{\delta \Lambda^\prime} \right)^s
			\left( -\frac{\delta}{\delta \Lambda^{\prime*}} \right)^r
			\chi_W [\Lambda, \Lambda^*]
		\right|_{\Lambda \equiv 0}.
	\end{eqn*}
\end{lemma}
\begin{proof}
The displacement operator can be expanded as
\begin{eqn}
	\exp (\Lambda \Psiop^\dagger - \Lambda^* \Psiop)
	= \sum_{r,s}
		\frac{
			\symprod{
				\left( \int d\xvec \Lambda \Psiop^\dagger \right)^r
				\left( -\int d\xvec \Lambda^* \Psiop \right)^s
			}
		}
		{r!s!}.
\end{eqn}
We can swap functional derivative with both integration and multiplication by independent function, so:
\begin{eqn}
	\frac{\delta}{\delta \Lambda^\prime} \left( \int d\xvec \Lambda \Psiop^\dagger \right)^r
	= r \Psiop^{\prime\dagger} \left( \int d\xvec \Lambda \Psiop^\dagger \right)^{r-1},
\end{eqn}
and multiple application of the differential gives us
\begin{eqn}
	\left( \frac{\delta}{\delta \Lambda^\prime} \right)^r
	\left( \int d\xvec \Lambda \Psiop^\dagger \right)^r
	= r! ( \Psiop^{\prime\dagger} )^r.
\end{eqn}
Similarly for the other differential:
\begin{eqn}
	\left( -\frac{\delta}{\delta \Lambda^{\prime*}} \right)^s
	\left( -\int d\xvec \Lambda \Psiop^\dagger \right)^s
	= s! ( \Psiop^{\prime\dagger} )^s.
\end{eqn}

Thus, differentiation will eliminate all lower order terms in the expansion, and all higher order terms will be eliminated by setting $\Lambda \equiv 0$, leaving only one operator product with required order.
\end{proof}

\begin{theorem}[Calculation of expectations]
\label{thm:func-wigner:moments}
    For any non-negative integer $r$ and $s$:
	\begin{eqn*}
		\langle \symprod{ \Psiop^r (\Psiop^\dagger)^s } \rangle
		= \int \delta^2\Psi\, \Psi^r (\Psi^*)^s W[\Psi, \Psi^*].
	\end{eqn*}
\end{theorem}
\begin{proof}
By definition of Wigner functional:
\begin{eqn}
	\int \delta^2\Psi\, \Psi^r (\Psi^*)^s W[\Psi, \Psi^*] \\
	= \frac{1}{\pi^{2|\restbasis|}} \Trace{ \hat{\rho}
		\int \delta^2\Psi\, \Psi^r (\Psi^*)^s
		\int \delta^2\Lambda D[\Lambda, \Lambda^*, \Psi, \Psi^*]
		\hat{D}[\Lambda, \Lambda^*]
	}
\end{eqn}
Integrating by parts and eliminating terms which fit \lmmref{func-wigner:zero-integrals}:
\begin{eqn}
\fl	= \frac{1}{\pi^{2|\restbasis|}} \Trace{ \hat{\rho}
		\int \delta^2\Psi \int \delta^2\Lambda
		D[\Lambda, \Lambda^*, \Psi, \Psi^*]
		\left( \frac{\delta}{\delta \Lambda} \right)^s
		\left( -\frac{\delta}{\delta \Lambda^*} \right)^r
		\hat{D}[\Lambda, \Lambda^*]
	}
\end{eqn}
Evaluating the integral over $\Psi$ using \lmmref{func-calculus:fourier-of-moments}:
\begin{eqn*}
	& = \int \delta^2\Lambda\,
		\Delta_{\restbasis}[\Lambda]
		\left( \frac{\delta}{\delta \Lambda} \right)^s
		\left( -\frac{\delta}{\delta \Lambda^*} \right)^r
		\Trace{
			\hat{\rho}
			\hat{D}[\Lambda, \Lambda^*]
		} \\
	& = \left.
		\left( \frac{\delta}{\delta \Lambda} \right)^s
		\left( -\frac{\delta}{\delta \Lambda^*} \right)^r
		\chi_W [\Lambda, \Lambda^*]
	\right|_{\Lambda \equiv 0},
\end{eqn*}
where $\Delta_{\restbasis}[\Lambda]$ is a delta functional from \defref{func-calculus:delta-functional}.
Now, recognising the final expression as a part of \lmmref{func-wigner:moments-from-chi},
we immediately get the statement of the theorem.
\end{proof}

\thmref{func-wigner:correspondences} and \thmref{func-wigner:moments} can be further extended to the case of several components.
\todo{Do we need them here?}

\begin{theorem}[Multi-component extension of \thmref{func-wigner:correspondences}]
\label{thm:func-wigner:mc-correspondences}
    If $\mathcal{W} [ \hat{A} ] \equiv (\mathcal{W} [ \hat{A} ]) [\Psivec, \Psivec^*]$, then
    \begin{eqn*}
    	\mathcal{W} [ \Psiop_c \hat{A} ]
    		& = \left( \Psi_c + \frac{1}{2} \frac{\delta}{\delta \Psi_c^*} \right) \mathcal{W}[\hat{A}],
    	\quad
    	\mathcal{W} [ \Psiop_c^\dagger \hat{A} ]
    		= \left( \Psi_c^* - \frac{1}{2} \frac{\delta}{\delta \Psi_c} \right) \mathcal{W}[\hat{A}], \\
    	\mathcal{W} [ \hat{A} \Psiop_c ]
    		& = \left( \Psi_c - \frac{1}{2} \frac{\delta}{\delta \Psi_c^*} \right) \mathcal{W}[\hat{A}],
    	\quad
    	\mathcal{W} [ \hat{A} \Psiop_c^\dagger ]
    		= \left( \Psi_c^* + \frac{1}{2} \frac{\delta}{\partial \Psi_c} \right) \mathcal{W}[\hat{A}].
    \end{eqn*}
\end{theorem}

\begin{theorem}[Multi-component extension of \thmref{func-wigner:moments}]
\label{thm:func-wigner:mc-moments}
	For any non-negative integers $r_c$ and $s_c$, $c \in [1, C]$
	\begin{eqn*}
	    \langle \symprod{ \prod_{c=1}^C \Psiop_c^{r_c} (\Psiop_c^\dagger)^{s_c} } \rangle
		= \left( \prod_{c=1}^C \int \delta^2\Psi_c \right)
		\left( \prod_{c=1}^C \Psi_c^{r_c} (\Psi_c^*)^{s_c} \right) W[\Psivec, \Psivec^*].
	\end{eqn*}
\end{theorem}
