% =============================================================================
\section{Functional Wigner representation}
% =============================================================================

The single-mode Wigner transformation of the operator $\hat{A}$ is defined as
\begin{eqn}
	\mathcal{W}_{\mathrm{sm}}[\hat{A}]
	= \frac{1}{\pi^2} \int d^2 \lambda \exp(-\lambda \alpha^* + \lambda^* \alpha)
		\Trace{ \hat{A} \hat{D}(\lambda, \lambda^*) },
\end{eqn}
where the displacement operator $\hat{D}(\lambda, \lambda^*) = \exp(\lambda \hat{a}^\dagger - \lambda^* \hat{a})$ was first introduced by Weyl~\cite{Weyl1950}.
The detailed description of the Wigner function $W(\alpha, \alpha^*) \equiv \mathcal{W}_{\mathrm{sm}}[\hat{\rho}]$ can be found in~\cite{Gardiner2004}.
In this section we will extend this definition to the multimode case.

The important part of the definition is the functional analogue of the displacement operator.
\todo{Need to explain why we write $F[\Lambda, \Lambda^*]$ and not $F[\Lambda]$?}

\begin{definition}
    Functional displacement operator
	\begin{eqn*}
		\hat{D}_j :: \mathbb{F}_{\restbasis_j} \rightarrow \mathbb{H}_{\restbasis_j} \\
		\hat{D}_j[\Lambda, \Lambda^*] = \exp \int d\xvec \left(
			\Lambda \Psiop_j^\dagger - \Lambda^* \Psiop_j
		\right).
	\end{eqn*}
	It is also convenient to define the displacement functional as
	\begin{eqn*}
		D :: \mathbb{F}_{\restbasis_j} \rightarrow \mathbb{F}_{\restbasis_j} \rightarrow \mathbb{C} \\
		D[\Lambda, \Lambda^*, \Psi, \Psi^*] = \exp \int d\xvec \left(
			-\Lambda \Psi^* + \Lambda^* \Psi
		\right).
	\end{eqn*}
\end{definition}

It can be shown that the functional displacement operator has properties similar to its single-mode equivalent.

\begin{lemma}
\label{lmm:func-wigner:displacement-derivatives}
	\begin{eqn*}
		\frac{\delta}{\delta \Lambda^\prime} \hat{D}_j[\Lambda, \Lambda^*]
		= \hat{D}_j[\Lambda, \Lambda^*] (\Psiop_j^{\prime\dagger} + \frac{1}{2} \Lambda^{\prime*})
		= (\Psiop_j^{\prime\dagger} - \frac{1}{2} \Lambda^{\prime*}) \hat{D}_j[\Lambda, \Lambda^*], \\
		-\frac{\delta}{\delta \Lambda^{\prime*}} \hat{D}_j[\Lambda, \Lambda^*]
		= \hat{D}_j(\Lambda, \Lambda^*) (\Psiop_j^\prime + \frac{1}{2} \Lambda^\prime)
		= (\Psiop_j^\prime - \frac{1}{2} \Lambda^\prime) \hat{D}_j[\Lambda, \Lambda^*].
	\end{eqn*}
\end{lemma}
\begin{proof}
Proved using Baker-Hausdorff theorem and evaluating integrals.
\end{proof}

It is convenient to first define a general Wigner transformation.

\begin{definition}
\label{def:func-wigner:w-transformation}
	Multi-component functional Wigner transformation $\mathcal{W}$ is defined as
	\todo{Is the return value real or complex?}
	\begin{eqn*}
		\mathcal{W} :: \left( \mathbb{R}^D \rightarrow \prod_{j=1}^C \mathbb{H}_{\restbasis_j} \right)
			\rightarrow \prod_{j=1}^C \mathbb{F}_{\restbasis_j}
			\rightarrow \mathbb{R} \\
		\mathcal{W}[\hat{A}]
		= \frac{1}{\pi^{2 \sum|\restbasis_j|}} \int \delta^2 \bLambda
			\left( \prod_{j=1}^C D[\Lambda_j, \Lambda_j^*, \Psi_j, \Psi_j^*] \right)
			\Trace{ \hat{A} \prod_{j=1}^C \hat{D}_j[\Lambda_j, \Lambda_j^*] },
	\end{eqn*}
	where $\Lambda_j \in \mathbb{F}_{\restbasis_j}$, and $\int \delta^2 \bLambda \equiv \int \delta^2 \Lambda_1 \ldots \delta^2 \Lambda_C$.
	It transforms a coordinate-dependent operator $\hat{A}$ on a restricted subset of a Hilbert space to a functional $(\mathcal{W}[\hat{A}])[\bPsi, \bPsi^*]$.
	\todo{Add expression for Weyl transformation?}
\end{definition}

Wigner functional is a special case of Wigner transformation.

\begin{definition}
\label{def:func-wigner:w-functional}
	The Wigner functional is
	\begin{eqn*}
		W :: \prod_{j=1}^C \mathbb{F}_{\restbasis_j} \rightarrow \mathbb{R} \\
		W [\bPsi, \bPsi^*]
		\equiv \mathcal{W}[\hat{\rho}]
		= \frac{1}{\pi^{2 \sum|\restbasis_j|}} \int \delta^2 \bLambda
			\left( \prod_{j=1}^C D[\Lambda_j, \Lambda_j^*, \Psi_j, \Psi_j^*] \right) \chi_W,
	\end{eqn*}
	where $\chi_W [\bLambda, \bLambda^*]$ is the characteristic functional
	\begin{eqn*}
		\chi_W [\bLambda, \bLambda^*]
		= \Trace{ \hat{\rho} \prod_{j=1}^C \hat{D}_j[\Lambda_j, \Lambda_j^*] }.
	\end{eqn*}
\end{definition}

The Wigner functional has two important properties analogous to the single-mode case.
First one is used to successively transform operator products.
In order to prove it, we will need an auxiliary lemma.

\begin{lemma}
\label{lmm:func-wigner:zero-integrals}
	For $\Lambda \in \mathbb{F}_j$, $\Psi \in \mathbb{F}_j$:
	\begin{eqn*}
		\int \delta^2\Lambda
			\frac{\delta}{\delta \Lambda^\prime} \left(
				D[\Lambda, \Lambda^*, \Psi_j, \Psi_j^*]
				\left( \frac{\delta}{\delta \Lambda^\prime} \right)^r
				\left( -\frac{\delta}{\delta \Lambda^{\prime*}} \right)^s
				\hat{D}_j[\Lambda, \Lambda^*]
			\right)
		= 0, \\
		\int \delta^2\Lambda
			\frac{\delta}{\delta \Lambda^{\prime*}}
			\left(
				D[\Lambda, \Lambda^*, \Psi, \Psi^*]
				\left( \frac{\delta}{\delta \Lambda^\prime} \right)^r
				\left( -\frac{\delta}{\delta \Lambda^{\prime*}} \right)^s
				\hat{D}_j[\Lambda, \Lambda^*]
			\right)
		= 0.
	\end{eqn*}
\end{lemma}
\begin{proof}
Displacement operator and displacement functional can be represented as functions of vectors:
\begin{eqn}
	\hat{D}[\Lambda, \Lambda^*]
	= \prod_{\nvec \in \restbasis_j} \exp \left(
		\lambda_{\nvec} \hat{a}_{j,\nvec}^\dagger - \lambda_{\nvec}^* \hat{a}_{j,\nvec}
	\right),
\end{eqn}
\begin{eqn}
	D[\Lambda, \Lambda^*, \Psi, \Psi^*]
	= \prod_{\nvec \in \restbasis_j} \exp
		(-\lambda_{\nvec} \alpha_{j,\nvec}^* + \lambda_{\nvec}^* \alpha_{j,\nvec}),
\end{eqn}
The proof consists of substituting these in the equations from the statement and applying \lmmref{c-numbers:zero-integrals}.
\end{proof}

\begin{theorem}[Functional correspondences]
\label{thm:func-wigner:correspondences}
    If $\mathcal{W} [ \hat{A} ] \equiv (\mathcal{W} [ \hat{A} ]) [\bPsi, \bPsi^*]$, then
	\begin{eqn*}
		\mathcal{W} [ \Psiop_j \hat{A} ]
			& = \left( \Psi_j + \frac{1}{2} \frac{\delta}{\delta \Psi_j^*} \right) \mathcal{W}[\hat{A}],
		\quad
		\mathcal{W} [ \Psiop_j^\dagger \hat{A} ]
			= \left( \Psi_j^* - \frac{1}{2} \frac{\delta}{\delta \Psi_j} \right) \mathcal{W}[\hat{A}], \\
		\mathcal{W} [ \hat{A} \Psiop_j ]
			& = \left( \Psi_j - \frac{1}{2} \frac{\delta}{\delta \Psi_j^*} \right) \mathcal{W}[\hat{A}],
		\quad
		\mathcal{W} [ \hat{A} \Psiop_j^\dagger ]
			= \left( \Psi_j^* + \frac{1}{2} \frac{\delta}{\partial \Psi_j} \right) \mathcal{W}[\hat{A}].
	\end{eqn*}
\end{theorem}
\begin{proof}
The proof uses \lmmref{func-wigner:displacement-derivatives} to transform the $\hat{A} \prod_j \hat{D}_j$ product inside the trace, and \lmmref{func-wigner:zero-integrals} to integrate by parts, effectively moving the differentials to intended places.
\end{proof}

The second property complements the first one, providing the way to obtain expectations of operator products given the Wigner function.
Again, it requires a supplementary lemma.

\begin{lemma}
\label{lmm:func-wigner:moments-from-chi}
    For any non-negative integer $r$ and $s$:
	\begin{eqn*}
		\langle \symprod{ (\Psiop_j^\prime)^r (\Psiop_j^{\prime\dagger})^s } \rangle
		= \left.
			\left( \frac{\delta}{\delta \Lambda_j^\prime} \right)^s
			\left( -\frac{\delta}{\delta \Lambda_j^{\prime*}} \right)^r
			\chi_W [\bLambda, \bLambda^*]
		\right|_{\bLambda \equiv 0}.
	\end{eqn*}
\end{lemma}
\begin{proof}
The factor corresponding to $j$-th component in the displacement operator can be expanded as
\begin{eqn}
	\exp (\Lambda_j \Psiop_j^\dagger - \Lambda_j^* \Psiop_j)
	= \sum_{r,s}
		\frac{
			\symprod{
				\left( \int d\xvec \Lambda_j \Psiop_j^\dagger \right)^r
				\left( -\int d\xvec \Lambda_j^* \Psiop_j \right)^s
			}
		}
		{r!s!}.
\end{eqn}
We can swap functional derivative with both integration and multiplication by independent function, so:
\begin{eqn}
	\frac{\delta}{\delta \Lambda_j^\prime} \left( \int d\xvec \Lambda_j \Psiop_j^\dagger \right)^r
	= r \Psiop_j^{\prime\dagger} \left( \int d\xvec \Lambda_j \Psiop_j^\dagger \right)^{r-1},
\end{eqn}
and multiple application of the differential gives us
\begin{eqn}
	\left( \frac{\delta}{\delta \Lambda_j^\prime} \right)^r
	\left( \int d\xvec \Lambda_j \Psiop_j^\dagger \right)^r
	= r! ( \Psiop_j^{\prime\dagger} )^r.
\end{eqn}
Similarly for the other differential:
\begin{eqn}
	\left( -\frac{\delta}{\delta \Lambda_j^{\prime*}} \right)^s
	\left( -\int d\xvec \Lambda_j \Psiop_j^\dagger \right)^s
	= s! ( \Psiop_j^{\prime\dagger} )^s.
\end{eqn}

Thus, differentiation will eliminate all lower order terms in the expansion, and all higher order terms will be eliminated by setting $\Lambda_j \equiv 0$ for every $j$, leaving only one operator product with required order.
\end{proof}

\begin{theorem}[Calculation of expectations]
\label{thm:func-wigner:moments}
	For any non-negative integer $r_j$, $s_j$
	\begin{eqn*}
		\langle \symprod{ \prod_{j=1}^C \Psiop_j^{r_j} (\Psiop_j^\dagger)^{s_j} } \rangle
		= \int \delta^2 \bPsi
			\left( \prod_{j=1}^C \Psi_j^{r_j} (\Psi_j^*)^{s_j} \right) W[\bPsi, \bPsi^*].
	\end{eqn*}
\end{theorem}
\begin{proof}
By definition of the Wigner functional:
\begin{eqn}
\fl	\int \delta^2\bPsi \left( \prod_{j=1}^C \Psi_j^{r_j} (\Psi_j^*)^{s_j} \right) W[\bPsi, \bPsi^*] \\
\fl	= \frac{1}{\pi^{2\sum|\restbasis_j|}} \Trace{ \hat{\rho}
		\prod_{j=1}^C
			\int \delta^2 \Lambda_j \left(
				\int \delta^2 \Psi_j\, \Psi_j^{r_j} (\Psi_j^*)^{s_j}
				D[\Lambda_j, \Lambda_j^*, \Psi_j, \Psi_j^*]
			\right)
			\hat{D}_j[\Lambda_j, \Lambda_j^*]
	}.
\end{eqn}
Evaluating the integral over $\Psi_j$ using \lmmref{func-calculus:fourier-of-moments}:
\begin{eqn}
	= \Trace{ \hat{\rho}
		\prod_{j=1}^C
			\int \delta^2 \Lambda_j \left(
				\left( -\frac{\delta}{\delta \Lambda_j^*} \right)^{r_j}
				\left( \frac{\delta}{\delta \Lambda_j} \right)^{s_j}
				\Delta_{\restbasis_j}[\Lambda_j]
			\right)
			\hat{D}_j[\Lambda_j, \Lambda_j^*]
	}.
\end{eqn}
Integrating by parts for each component in turn and eliminating terms which fit \lmmref{func-wigner:zero-integral}:
\begin{eqn}
	& = \Trace{ \hat{\rho}
		\prod_{j=1}^C \int \delta^2 \Lambda_j
			\Delta_{\restbasis_j}[\Lambda_j]
			\left( -\frac{\delta}{\delta \Lambda_j^*} \right)^{r_j}
			\left( \frac{\delta}{\delta \Lambda_j} \right)^{s_j}
			\hat{D}_j[\Lambda_j, \Lambda_j^*]
	} \\
	& = \left.
		\left(
			\prod_{j=1}^C
			\left( \frac{\delta}{\delta \Lambda_j} \right)^{s_j}
			\left( -\frac{\delta}{\delta \Lambda_j^*} \right)^{r_j}
		\right)
		\chi_W [\bLambda, \bLambda^*]
	\right|_{\bLambda \equiv 0},
\end{eqn}
where $\Delta_{\restbasis_j}[\Lambda_j]$ is a delta functional from \defref{func-calculus:delta-functional}.
Now, recognising the final expression as a part of \lmmref{func-wigner:moments-from-chi},
we immediately get the statement of the theorem.
\end{proof}
