% =============================================================================
\section{Initial states}
% =============================================================================

The simplest case of initial state is a coherent state.

\begin{theorem}
    The Wigner distribution for a multi-mode coherent state with with expectation values
    $\alpha_{\nvec}(0) = \alpha_{\nvec}^{(0)}$, $\nvec \in L$ is
    \begin{eqn*}
    	W_c (\balpha^{(0)})
    	= \left( \frac{2}{\pi} \right)^N \prod_{\nvec \in L}
    		\exp(-2 |\alpha_{\nvec} - \alpha_{\nvec}^{(0)}|^2).
    \end{eqn*}
\end{theorem}
\begin{proof}
The density matrix of the state is
\begin{eqn}
	\hat{\rho}
	= \vert \alpha_{\nvec}^{(0)},\, \nvec \in L \rangle
		\langle \alpha_{\nvec}^{(0)},\, \nvec \in L \vert
	= \left( \prod_{\nvec \in L} \vert \alpha_{\nvec}^{(0)} \rangle \right)
		\left( \prod_{\nvec \in L} \langle \alpha_{\nvec}^{(0)} \vert \right).
\end{eqn}
Then the characteristic function is
\begin{eqn}
	\chi_W (\balpha^{(0)})
	= \prod_{\nvec \in L}
		\langle \alpha_{\nvec}^{(0)} \vert
		\hat{D}_{\nvec} (\lambda_{\nvec}, \lambda_{\nvec}^*)
		\vert \alpha_{\nvec}^{(0)} \rangle
\end{eqn}
Using the properties of the displacement operator, this can be transformed to
\begin{eqn}
	\chi_W (\balpha^{(0)})
	= \prod_{\nvec \in L}
		\exp(
			- \lambda_{\nvec}^* \alpha_{\nvec}^{(0)}
			+ \lambda_{\nvec} (\alpha_{\nvec}^{(0)})^*
			- \frac{1}{2} |\lambda|^2
		).
\end{eqn}

Finally, Wigner function is
\begin{eqn}
	W_c (\balpha^{(0)})
	& = \frac{1}{\pi^{2N}} \prod_{\nvec \in L} \left(
		\int d^2\lambda_{\nvec}
			\exp(
				- \lambda_{\nvec} (\alpha_{\nvec}^* - (\alpha_{\nvec}^{(0)})^*)
				+ \lambda_{\nvec}^* (\alpha_{\nvec} - \alpha_{\nvec}^{(0)})
				- \frac{1}{2} |\lambda|^2
			)
	\right) \\
	& = \left( \frac{2}{\pi} \right)^N \prod_{\nvec \in L}
		\exp(-2 |\alpha_{\nvec} - \alpha_{\nvec}^{(0)}|^2).
	\qedhere
\end{eqn}
\end{proof}

The resulting Wigner distribution is a product of independent complex-valued Gaussian distributions for each mode,
with the expectation value equal to the expectation value of the mode,
and the variance equal to $\frac{1}{2}$.
Therefore the initial state can be sampled as
\begin{eqn}
	\alpha_{\nvec} = \alpha_{\nvec}^{(0)} + \frac{1}{\sqrt{2}} \eta_{\nvec},
\end{eqn}
where $\eta_{\nvec}$ are normally distributed complex random numbers with zero mean,
$\langle \eta_{\mvec} \eta_{\nvec} \rangle = 0$ and
$\langle \eta_{\mvec} \eta_{\nvec}^* \rangle = \delta_{\mvec,\nvec}$
(in other words, with components distributed independently with variance $\frac{1}{2}$).
This looks like adding half a ``vacuum particle'' to each mode.
In functional form this can be written as
\begin{eqn}
	\Psi_j(\xvec, 0)
	= \Psi_j^{(0)}(\xvec, 0)
		+ \sum_{\nvec \in L} \frac{\eta_{j,\nvec}}{2} \phi_{\nvec}(\xvec),
\end{eqn}
where $\Psi_j^{(0)}(\xvec, 0)$ is the ``classical'' ground state of the system.
