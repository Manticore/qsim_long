% =============================================================================
\section{Initial states}
% =============================================================================

Initial values for the numerical integration of equations~\eqnref{fpe:sdes} are obtained by finding the Wigner transformation of the density matrix for the desired initial state, and then sampling the initial values according to the resulting Wigner function.
As an example, consider the simple case with a coherent initial state.

\begin{theorem}
    The Wigner distribution for a multi-mode coherent state with with expectation values
    $\alpha_{\nvec}(0) = \alpha_{\nvec}^{(0)}$, $\nvec \in \restbasis$ is
    \begin{eqn*}
    	W_c (\balpha^{(0)})
    	= \left( \frac{2}{\pi} \right)^{|\restbasis|} \prod_{\nvec \in \restbasis}
    		\exp(-2 |\alpha_{\nvec} - \alpha_{\nvec}^{(0)}|^2).
    \end{eqn*}
\end{theorem}
\begin{proof}
The density matrix of the state is
\begin{eqn}
	\hat{\rho}
	= \vert \alpha_{\nvec}^{(0)},\, \nvec \in \restbasis \rangle
		\langle \alpha_{\nvec}^{(0)},\, \nvec \in \restbasis \vert
	= \left( \prod_{\nvec \in \restbasis} \vert \alpha_{\nvec}^{(0)} \rangle \right)
		\left( \prod_{\nvec \in \restbasis} \langle \alpha_{\nvec}^{(0)} \vert \right).
\end{eqn}
Then the characteristic function is
\begin{eqn}
	\chi_W (\balpha^{(0)})
	= \prod_{\nvec \in \restbasis}
		\langle \alpha_{\nvec}^{(0)} \vert
		\hat{D}_{\nvec} (\lambda_{\nvec}, \lambda_{\nvec}^*)
		\vert \alpha_{\nvec}^{(0)} \rangle
\end{eqn}
Using the properties of the displacement operator, this can be transformed to
\begin{eqn}
	\chi_W (\balpha^{(0)})
	= \prod_{\nvec \in \restbasis}
		\exp(
			- \lambda_{\nvec}^* \alpha_{\nvec}^{(0)}
			+ \lambda_{\nvec} (\alpha_{\nvec}^{(0)})^*
			- \frac{1}{2} |\lambda|^2
		).
\end{eqn}
Finally, Wigner function is
\begin{eqn}
\fl	W_c (\balpha^{(0)})
	= \frac{1}{\pi^{2|\restbasis|}} \prod_{\nvec \in \restbasis} \left(
		\int d^2\lambda_{\nvec}
			\exp(
				- \lambda_{\nvec} (\alpha_{\nvec}^* - (\alpha_{\nvec}^{(0)})^*)
				+ \lambda_{\nvec}^* (\alpha_{\nvec} - \alpha_{\nvec}^{(0)})
				- \frac{1}{2} |\lambda|^2
			)
	\right) \\
	= \left( \frac{2}{\pi} \right)^{|\restbasis|} \prod_{\nvec \in \restbasis}
		\exp(-2 |\alpha_{\nvec} - \alpha_{\nvec}^{(0)}|^2).
	\qedhere
\end{eqn}
\end{proof}

The resulting Wigner distribution is a product of independent complex-valued Gaussian distributions for each mode,
with the expectation value equal to the expectation value of the mode,
and the variance equal to $\frac{1}{2}$.
Therefore the initial state can be sampled as
\begin{eqn}
	\alpha_{\nvec} = \alpha_{\nvec}^{(0)} + \frac{1}{\sqrt{2}} \eta_{\nvec},
\end{eqn}
where $\eta_{\nvec}$ are normally distributed complex random numbers with zero mean,
$\langle \eta_{\mvec} \eta_{\nvec} \rangle = 0$ and
$\langle \eta_{\mvec} \eta_{\nvec}^* \rangle = \delta_{\mvec,\nvec}$
(in other words, with components distributed independently with variance $\frac{1}{2}$).
This looks like adding half a ``vacuum particle'' to each mode.
In functional form this can be written as
\begin{eqn}
	\Psi_j(\xvec, 0)
	= \Psi_j^{(0)}(\xvec, 0)
		+ \sum_{\nvec \in \restbasis} \frac{\eta_{j,\nvec}}{2} \phi_{\nvec}(\xvec),
\end{eqn}
where $\Psi_j^{(0)}(\xvec, 0)$ is the ``classical'' ground state of the system.

\todo{We can describe the Wigner function for the number state here, as an example of non-positive Wigner function.}
