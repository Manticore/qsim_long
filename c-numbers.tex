% =============================================================================
\section{Wirtinger differentiation}
% =============================================================================

In this paper we are using differentiation of complex functions extensively.
Instead of the classical definition of differential which only works for holomorphic functions, we use Wirtinger differentiation~\cite{Wirtinger1927}.
One can find thorough description of these rules, for example, in~\cite{Kreutz-Delgado2009}; in this section we will only outline the basics.

\begin{definition}
	For a complex variable $z = x + iy$ and a function $f(z) = u(x, y) + iv(x, y)$ the Wirtinger differential is
	\begin{eqn*}
		\frac{\partial f(z)}{\partial z}
		= \frac{1}{2} \left(
			\frac{\partial f}{\partial x} - i \frac{\partial f}{\partial y}
		\right).
	\end{eqn*}
\end{definition}

One can easily prove that if $f(z)$ is holomorphic, then the above definition coincides with the classical differential for complex functions.
Wirtinger differential obeys sum, product, quotient, and chain differentiation rules (the former one is applied as if $f(z) \equiv f(z, z^*)$).

In addition, we will need a non-standard integration over the complex variable:

\begin{definition}
	For a complex variable $z = x + iy$ the integral
	\begin{eqn*}
		\int d^2 z \equiv \int_{-\infty}^{\infty} \int_{-\infty}^{\infty} dx\, dy,
	\end{eqn*}
	or, in other words, stands for the two-dimensional integral over the complex plane.
\end{definition}

Such integration has a property similar to a Fourier transformation in real space.

\begin{lemma}
\label{lmm:c-numbers:fourier-of-moments}
	If $\lambda$ is a complex variable, then for any non-negative integers $r$ and $s$:
	\begin{eqn*}
		\int d^2\alpha\, \alpha^r (\alpha^*)^s \exp(-\lambda \alpha^* + \lambda^* \alpha)
		= \pi^2
			\left( -\frac{\partial}{\partial \lambda^*} \right)^r
			\left( \frac{\partial}{\partial \lambda} \right)^s
			\delta(\Real \lambda) \delta(\Imag \lambda)
	\end{eqn*}
\end{lemma}
\begin{proof}
First, using known Fourier transform relations, it is easy to prove that for real $x$ and $v$, and non-negative integer $n$
\begin{eqn*}
	\int\limits_{-\infty}^{\infty} dv\, v^n \exp(\pm 2 i x v)
	= \pi (\mp i / 2)^n \delta^{(n)}(x).
\end{eqn*}
Substituting $\alpha = x + iy$, expanding the $\alpha^r (\alpha^*)^s$ term using binomial theorem and employing the above property, one can reach the statement of the lemma.
\end{proof}

Another important property is used extensively throughout the paper.

\begin{lemma}
\label{lmm:c-numbers:zero-integrals}
	If $f(\lambda, \lambda^*)$ is bounded, then for any complex $\alpha$:
	\begin{eqn*}
		\int d^2\lambda
			\frac{\partial}{\partial \lambda} \left(
				\exp(-\lambda \alpha^* + \lambda^* \alpha)
				f(\lambda, \lambda^*)
			\right)
		& = 0 \\
		\int d^2\lambda
			\frac{\partial}{\partial \lambda^*}
			\left(
				\exp(-\lambda \alpha^* + \lambda^* \alpha)
				f(\lambda, \lambda^*)
			\right)
		& = 0.
	\end{eqn*}
\end{lemma}
\begin{proof}
\todo{Needs proof?}
\end{proof}
