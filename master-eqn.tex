% =============================================================================
\section{Master equation}
% =============================================================================

Phase-space treatment of multimode problems can be simplified by working with multimode field operators instead of single-mode operators.
This approach was initially introduced by Graham~\cite{Graham1970,Graham1970a}.
Examples of usage can be found in~\cite{Steel1998,Norrie2006a}.

The system can be described by the second-quantized Hamiltonian
\begin{eqn}
\label{eqn:master-eqn:hamiltonian}
\fl	\hat{H}
    = \int d\xvec \left\{
		\Psiop_j^{\dagger} K_{jk} \Psiop_k
		+ \frac{1}{2} \int d\xvec^\prime
			\Psiop_j^\dagger (\xvec) \Psiop_k^\dagger (\xvec^\prime)
			U_{jk}(\xvec - \xvec^\prime)
			\Psiop_j (\xvec^\prime) \Psiop_k (\xvec)
	\right\},
\end{eqn}
where $\xvec$ is a $D$-dimensional coordinate vector, $U_{jk}$ is the two-body scattering potential, and $\Psiop_j^{\dagger}(\xvec)$ and $\Psiop_j(\xvec)$ are bosonic field creation and annihilation operators, which obey standard bosonic commutation relation
\begin{eqn}
\label{eqn:master-eqn:commutators}
	[ \Psiop_{j}(\xvec), \Psiop_{k}^{\dagger}(\xvec^\prime) ]
	= \delta_{jk} \delta(\xvec^\prime-\xvec).
\end{eqn}
Single-particle Hamiltonian $K_{jk}$ is
\begin{eqn}
	K_{jk} = \left(
			-\frac{\hbar^2}{2m} \nabla^2 + \hbar \omega_j + V_j(\xvec)
		\right) \delta_{jk}
		+ \hbar \Omega_{jk}(t),
\end{eqn}
where $m$ is the atomic mass, $V_j$ is the external trapping potential for spin $j$, $\hbar \omega_j$ is the internal energy of spin $j$, and $\Omega_{jk}$ represents a time-dependent coupling that is used to rotate one spin projection into another.

If we impose an energy cutoff $\ecut$ and only take into account low-energy modes, the non-local scattering potential $U_{jk}(\xvec - \xvec^\prime)$ can be replaced by the contact potential $U_{jk} \delta(\xvec - \xvec^\prime)$~\cite{Morgan2000}, giving the effective Hamiltonian
\begin{eqn}
\label{eqn:master-eqn:effective-H}
	\hat{H}
	= \int d\xvec \left\{
		\Psiop_j^{\dagger} K_{jk} \Psiop_k
		+ \frac{U_{jk}}{2} \Psiop_j^\dagger \Psiop_k^\dagger \Psiop_j \Psiop_k
	\right\},
\end{eqn}
where $\Psiop_j^{\dagger}$ and $\Psiop_j$ are field operators in the new restricted basis of low-energy modes, which is described in detail in the next section.
For the $s$-wave scattering in three dimensions the coefficient is $U_{jk} = 4 \pi \hbar^2 a_{jk} / m$, where $a_{jk}$ is the scattering length.
Note that in general case the coefficient must be renormalised depending on the grid~\cite{Sinatra2002}, but the change is small if $dx \gg a_{jk}$.

The Markovian master equation for the system with the inclusion of losses can be written as~\cite{Jack2002}
\begin{eqn}
\label{eqn:master-eqn:master-eqn}
	\frac{d\hat{\rho}}{dt} =
		- \frac{i}{\hbar} \left[ \hat{H}, \hat{\rho} \right]
		+ \sum_{\lvec} \kappa_{\lvec} \int d\xvec
			\mathcal{L}_{\lvec} \left[ \hat{\rho} \right],
\end{eqn}
where $\lvec = (l_1, l_2, \ldots, l_C)$ is a tuple indicating the number of atoms from each component involved in the interaction, $C$ being the total number of components, and we have introduced local Liouville loss terms,
\begin{eqn}
	\mathcal{L}_{\lvec} \left[ \hat{\rho} \right] =
		2\hat{O}_{\lvec} \hat{\rho} \hat{O}_{\lvec}^\dagger
		- \hat{O}_{\lvec}^\dagger \hat{O}_{\lvec} \hat{\rho}
		- \hat{\rho} \hat{O}_{\lvec}^\dagger \hat{O}_{\lvec}.
\end{eqn}
The reservoir coupling operators $\hat{O}_{\lvec}$ are products of local field annihilation operators:
\begin{eqn}
    \hat{O}_{\lvec}
    \equiv \hat{O}_{\lvec} (\Psiopvec)
    = \prod_{c=1}^C \Psiop_c^{l_c} (\xvec),
\end{eqn}
describing local $\left( \sum_{c=1}^C l_c \right)$-body collision losses.
