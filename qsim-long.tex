\documentclass[pra,twocolumn,amssymb,superscriptaddress]{revtex4}
\usepackage{amsmath}
\usepackage{amsthm}
\usepackage{amsfonts}
\usepackage[pdftex]{graphicx}
\usepackage{bbm}
\usepackage{braket}
\usepackage[usenames,dvipsnames]{color}
\usepackage{wasysym}

\newcommand{\jvec}{\boldsymbol{j}}
\newcommand{\kvec}{\boldsymbol{k}}
\newcommand{\lvec}{\boldsymbol{l}}
\newcommand{\mvec}{\boldsymbol{m}}
\newcommand{\nvec}{\boldsymbol{n}}
\newcommand{\pvec}{\boldsymbol{p}}
\newcommand{\xvec}{\boldsymbol{x}}

\newcommand{\Tr}{\operatorname{Tr}}
\newcommand{\Trace}[1]{\Tr \left\{ #1 \right\}}

\newcommand{\symprod}[1]{\left\{ #1 \right\}_{\mathrm{sym}}}
\newcommand{\pathavg}[1]{\langle #1 \rangle_{\mathrm{paths}}}
\newcommand{\Real}{\operatorname{Re}}
\newcommand{\Imag}{\operatorname{Im}}

\newcommand{\Psivec}{\boldsymbol{\Psi}}
\newcommand{\Psiop}{\hat{\Psi}}
\newcommand{\Psiopvec}{\hat{\boldsymbol{\Psi}}}

\newcommand{\eqnref}[1]{Eq.~(\ref{eqn:#1})}
\newcommand{\figref}[1]{Fig.~\ref{fig:#1}}
\newcommand{\thmref}[1]{Theorem~\ref{thm:#1}}
\newcommand{\lmmref}[1]{Lemma~\ref{lmm:#1}}

\newtheorem{theorem}{Theorem}
\newtheorem{definition}{Definition}
\newtheorem{lemma}{Lemma}

\newcommand{\swinaffiliation}{Centre for Atom Optics and Ultrafast Spectroscopy, Swinburne University of Technology, Hawthorn, VIC 3122, Australia}

\begin{document}
\title{Wigner representation of BEC}

\author{B.~Opanchuk}
\author{P.~D.~Drummond}
\affiliation{\swinaffiliation}

\date{\today}
\begin{abstract}
Abstract goes here.
\end{abstract}

%\pacs{}
\maketitle

% =============================================================================
\section{Introduction}
% =============================================================================

Introduction goes here.

% =============================================================================
\section{Wirtinger differentiation}
% =============================================================================

In this paper we are using differentiation of complex functions extensively.
Instead of classical definition of the differential which only works for holomorphic functions we use Wirtinger differentiation~\cite{Wirtinger1927}.
One can find thorough description of these rules, for example, in~\cite{Kreutz-Delgado2009}; in this section we will only outline the basics.

\begin{definition}
	For a complex variable $z = x + iy$ and a function $f(z) = u(x, y) + iv(x, y)$ the Wirtinger differential is
	\begin{equation*}
		\frac{df(z)}{dz}
		= \frac{1}{2} \left(
			\frac{\partial f}{\partial x} - i \frac{\partial f}{\partial y}
		\right).
	\end{equation*}
\end{definition}

One can easily prove that if $f(z)$ is holomorphic, then the above definition coincides with the classical differential for complex functions.
Wirtinger differential obeys sum, product, quotient, and chain differentiation rules (the former one is applied as if $f(z) \equiv f(z, z^*)$).

We will need some lemmas about integration.
For convenience, we will use the following definition:

\begin{definition}
	For a complex variable $z = x + iy$ the integral
	\begin{equation*}
		\int d^2 z \equiv \int_{-\infty}^{\infty} \int_{-\infty}^{\infty} dx\, dy,
	\end{equation*}
	or, in other words, stands for the two-dimensional integral over the complex plane.
\end{definition}

\begin{lemma}
\label{lmm:c-numbers:fourier-of-moments}
	If $\alpha$ and $\lambda$ are complex variables,
	then for any non-negative integers $r$ and $s$:
	\begin{equation}
	\begin{split}
		& \int d^2\alpha\, \alpha^r (\alpha^*)^s \exp(-\lambda \alpha^* + \lambda^* \alpha) \\
		& = \pi^2
			\left( -\frac{\partial}{\partial \lambda^*} \right)^r
			\left( \frac{\partial}{\partial \lambda} \right)^s
			\delta(\Real \lambda) \delta(\Imag \lambda)
	\end{split}
	\end{equation}
\end{lemma}
\begin{proof}
First, changing the variables in the integrals and using known Fourier transform relations, we can prove that for real $x$ and $v$, and non-negative integer $n$
\begin{equation*}
	\int\limits_{-\infty}^{\infty} dv\, v^n \exp(\pm 2 i x v)
	= \pi (\mp i / 2)^n \delta^{(n)}(x),
\end{equation*}
Expanding the $\alpha^r (\alpha^*)^s$ term using binomial theorem and using the above property, one can reach the statement of the lemma.
\end{proof}

A notable special case of \lmmref{c-numbers:fourier-of-moments} is
\begin{equation*}
	\int d^2\alpha \exp(-\lambda \alpha^* + \lambda^* \alpha)
	= \pi^2 \delta(\Real \lambda) \delta(\Imag \lambda).
\end{equation*}

\begin{lemma}
\label{lmm:c-numbers:zero-integrals}
	For any non-negative integers $r$, $s$ and complex $\alpha$:
	\begin{equation*}
	\begin{split}
		\int d^2\lambda
			\frac{\partial}{\partial \lambda} \left(
				\exp(-\lambda \alpha^* + \lambda^* \alpha)
				\exp(ixy) x^r y^s
			\right)
		& = 0, \\
		\int d^2\lambda
			\frac{\partial}{\partial \lambda^*}
			\left(
				\exp(-\lambda \alpha^* + \lambda^* \alpha)
				\exp(ixy) x^r y^s
			\right)
		& = 0,
	\end{split}
	\end{equation*}
	where $\lambda = x + iy$.
\end{lemma}
\begin{proof}
We will prove the first equation.
First, note that complex-valued integral of derivative is evaluated as
\begin{equation*}
\begin{split}
	\int d^2\lambda \frac{\partial}{\partial \lambda} f(\lambda, \lambda^*)
	& =	\frac{1}{2} \int\limits_{-\infty}^{\infty} dy \left(
			\left. g(x, y) \right|_{x=-\infty}^{\infty}
		\right) \\
	& - \frac{i}{2} \int\limits_{-\infty}^{\infty} dx \left(
			\left. h(x, y) \right|_{y=-\infty}^{\infty}
		\right),
\end{split}
\end{equation*}
where we expanded $f = g + ih$.
Thus
\begin{equation*}
\begin{split}
	& \int d^2\lambda
		\frac{\partial}{\partial \lambda} \left(
			\exp(-\lambda \alpha^* + \lambda^* \alpha)
			\exp(ixy) x^r y^s
		\right) \\
	& = \left(
			\frac{1}{2} \exp(2ixv) x^r \int dy \exp(iy(x-2u)) y^s
		\right)_{x = -\infty}^\infty \\
	& - \left(
			\frac{i}{2} \exp(-2ixy) y^s \int dx \exp(ix(y+2v)) x^r
		\right)_{y = -\infty}^\infty \\
	& = \left(
			\frac{1}{2} \exp(2ixv) x^r 2 \pi i^s \delta^{(s)}(x-2u)
		\right)_{x = -\infty}^\infty \\
	& - \left(
			\frac{i}{2} \exp(-2ixy) y^s 2 \pi i^r \delta^{(r)}(y+2v)
		\right)_{y = -\infty}^\infty \\
	& = 0,
\end{split}
\end{equation*}
because any derivative of delta function is zero on the infinity.
\end{proof}

% =============================================================================
\section{Single-mode Wigner representation}
% =============================================================================

We will need the displacement operator which was first introduced by Weyl~\cite{Weyl1950}:

\begin{definition}
\label{def:sm-wigner:displacement-op}
	If $\hat{a}^\dagger$ and $\hat{a}$ are bosonic creation and annihilation operators, displacement operator $\hat{D}$ is
	\begin{equation*}
		\hat{D}(\lambda, \lambda^*) = \exp(\lambda \hat{a}^\dagger - \lambda^* \hat{a}).
	\end{equation*}
\end{definition}

Using Baker-Hausdorff theorem to split non-commuting operators in the exponent, one can find that
\begin{eqn}
\label{eqn:sm-wigner:displacement-derivatives}
	\frac{\partial}{\partial \lambda} \hat{D}(\lambda, \lambda^*)
	= \hat{D}(\lambda, \lambda^*) (\hat{a}^\dagger + \frac{1}{2} \lambda^*)
	= (\hat{a}^\dagger - \frac{1}{2} \lambda^*) \hat{D}(\lambda, \lambda^*), \\
	-\frac{\partial}{\partial \lambda^*} \hat{D}(\lambda, \lambda^*)
	= \hat{D}(\lambda, \lambda^*) (\hat{a} + \frac{1}{2} \lambda)
	= (\hat{a} - \frac{1}{2} \lambda) \hat{D}(\lambda, \lambda^*).
\end{eqn}

Wigner transformation converts an operator $\hat{A}$ on a Hilbert space to a function $\mathcal{W}[\hat{A}](\alpha, \alpha^*)$ on phase space.
In terms of the displacement operator Wigner transformation $\mathcal{W}$ and Wigner function $W$ can be defined as

\begin{definition}
\label{def:sm-wigner:w-transformation}
	Wigner transformation of an operator $\hat{A}$:
	\begin{eqn*}
		\mathcal{W}[\hat{A}]
		= \frac{1}{\pi^2} \int d^2 \lambda \exp(-\lambda \alpha^* + \lambda^* \alpha)
			\Trace{ \hat{A} \hat{D}(\lambda, \lambda^*) }.
	\end{eqn*}
	Wigner function is a Wigner transformation of a density matrix:
	\begin{eqn*}
		W(\alpha, \alpha^*) \equiv \mathcal{W}[\hat{\rho}].
	\end{eqn*}
\end{definition}

The Wigner function always exists for any density matrix~\cite{Gardiner2004}.
The correspondence $W \leftrightarrow \hat{\rho}$ is a bijection.
In some cases it is convenient to use the Wigner function in form
\begin{equation}
\label{eqn:sm-wigner:w-function}
	W (\alpha, \alpha^*)
	= \frac{1}{\pi^2} \int d^2 \lambda \exp(-\lambda \alpha^* + \lambda^* \alpha)
		\chi_W (\lambda, \lambda^*),
\end{equation}
where $\chi_W (\lambda, \lambda^*)$ is the characteristic function:
\begin{equation}
	\chi_W (\lambda, \lambda^*)
	= \Trace{ \hat{\rho} \hat{D}(\lambda, \lambda^*) }.
\end{equation}

\begin{lemma}
\label{lmm:sm-wigner:zero-integrals}
	For any integer $m$ and $n$
	\begin{eqn*}
		\int d^2\lambda
			\frac{\partial}{\partial \lambda} & \left(
				\exp(-\lambda \alpha^* + \lambda^* \alpha)
				\left( \frac{\partial}{\partial \lambda} \right)^m
				\left( -\frac{\partial}{\partial \lambda^*} \right)^n
				\hat{D}(\lambda, \lambda^*)
			\right)
		= 0, \\
		\int d^2\lambda
			\frac{\partial}{\partial \lambda^*} & \left(
				\exp(-\lambda \alpha^* + \lambda^* \alpha)
				\left( \frac{\partial}{\partial \lambda} \right)^m
				\left( -\frac{\partial}{\partial \lambda^*} \right)^n
				\hat{D}(\lambda, \lambda^*)
			\right)
		= 0.
	\end{eqn*}
\end{lemma}
\begin{proof}
We will prove the first equation.
Expanding $\lambda = x + iy$ and applying Baker-Hausdorff theorem:
\begin{eqn}
	\hat{D}(\lambda, \lambda^*)
	= \exp(ixy) \exp(x(\hat{a}^\dagger - \hat{a})) \exp(iy(\hat{a}^\dagger + \hat{a}))
\end{eqn}
Expanding derivatives over $\partial/\partial\lambda$ and $\partial/\partial\lambda^*$ in terms of $\partial/\partial x$ and $\partial/\partial y$, and exponents in the expression for $\hat{D}$ as power series:
\begin{eqn}
	\int d^2\lambda
		\frac{\partial}{\partial \lambda} \left(
			\exp(-\lambda \alpha^* + \lambda^* \alpha)
			\left( \frac{\partial}{\partial \lambda} \right)^m
			\left( -\frac{\partial}{\partial \lambda^*} \right)^n
			\hat{D}(\lambda, \lambda^*)
		\right) \\
	= \sum_{r=0}^{\infty} \sum_{s=0}^{\infty} \left(
			\int d^2\lambda
			\frac{\partial}{\partial \lambda} \left(
				\exp(-\lambda \alpha^* + \lambda^* \alpha)
				\exp(ixy) f_{mnrs}(x, y)
			\right)
		\right)
		g_{rs}(\hat{a}, \hat{a}^\dagger) \\
	= 0,
\end{eqn}
where $f_{mnrs}(x, y)$ and $g_{rs}(\hat{a}, \hat{a}^\dagger)$ are some finite-order polynomials,
and we used \lmmref{c-numbers:zero-integrals} to evaluate integrals over $\lambda$.
\end{proof}

\begin{lemma}
	\label{lmm:sm-wigner:moments-from-chi}
	For any integer $r$ and $s$
	\begin{eqn*}
		\langle \symprod{ \hat{a}^r (\hat{a}^\dagger)^s } \rangle
		= \left.
			\left( \frac{\partial}{\partial \lambda} \right)^s
			\left( -\frac{\partial}{\partial \lambda^*} \right)^r
			\chi_W (\lambda, \lambda^*)
		\right|_{\lambda=0}.
	\end{eqn*}
\end{lemma}
\begin{proof}
The exponent in the $\chi_W$ can be expanded as
\begin{eqn}
	\exp (\lambda \hat{a}^\dagger - \lambda^* \hat{a})
	= \sum_{r,s}
		\frac{(-\lambda^*)^r \lambda^s}{r!s!}
		\symprod{ \hat{a}^r (\hat{a}^\dagger)^s }.
\end{eqn}
Thus
\begin{eqn}
	\chi_W(\lambda, \lambda^*)
	& = \sum_{r,s}
		\frac{(-\lambda^*)^r \lambda^s}{r!s!}
		\Trace{
			\hat{\rho} \symprod{ \hat{a}^r (\hat{a}^\dagger)^s }
		} \\
	& = \sum_{r,s}
		\frac{(-\lambda^*)^r \lambda^s}{r!s!}
		\langle \symprod{ \hat{a}^r (\hat{a}^\dagger)^s } \rangle.
\end{eqn}
Apparently, the application of $(\partial / \partial \lambda)^s$ and $(-\partial / \partial \lambda^*)^r$ will eliminate all lower order moments, and setting $\lambda = 0$ afterwards will eliminate all higher order moments, leaving only $\symprod{ \hat{a}^r (\hat{a}^\dagger)^s }$:
\begin{eqn}
	\left.
		\left( \frac{\partial}{\partial \lambda} \right)^s
		\left( -\frac{\partial}{\partial \lambda^*} \right)^r
		\chi_W (\lambda, \lambda^*)
	\right|_{\lambda=0}
	& = r! s! \frac{1}{r! s!}
		\langle \symprod{ \hat{a}^r (\hat{a}^\dagger)^s } \rangle \\
	& = \langle \symprod{ \hat{a}^r (\hat{a}^\dagger)^s } \rangle.
	\qedhere
\end{eqn}
\end{proof}

Now we can get the final relation.

\begin{theorem}
\label{thm:sm-wigner:moments}
	Expectations of symmetrically ordered operator products are moments of the Wigner function:
	\begin{eqn*}
		\langle \symprod{ \hat{a}^r (\hat{a}^\dagger)^s } \rangle
		= \int d^2\alpha\, \alpha^r (\alpha^*)^s W(\alpha, \alpha^*)
	\end{eqn*}
\end{theorem}
\begin{proof}
By definition of the Wigner function:
\begin{eqn}
\int d^2\alpha\, \alpha^r (\alpha^*)^s W(\alpha, \alpha^*) \\
	= \frac{1}{\pi^2} \Trace{ \hat{\rho}
			\int d^2\alpha\, \alpha^r (\alpha^*)^s
			\int d^2\lambda \exp(-\lambda \alpha^* + \lambda^* \alpha)
			\hat{D}(\lambda, \lambda^*)
		}
\end{eqn}
Integrating by parts and eliminating terms which fit \lmmref{sm-wigner:zero-integrals}:
\begin{eqn}
\fl	= \frac{1}{\pi^2} \Trace{ \hat{\rho}
			\int d^2\alpha \int d^2\lambda
			\exp(-\lambda \alpha^* + \lambda^* \alpha)
			\left( \frac{\partial}{\partial \lambda} \right)^s
			\left( -\frac{\partial}{\partial \lambda^*} \right)^r
			\hat{D} (\lambda, \lambda^*)
		}
\end{eqn}
Evaluating integral over $\alpha$ using \lmmref{c-numbers:fourier-of-moments}:
\begin{eqn}
	& = \int d^2\lambda\,
		\delta (\Real \lambda) \delta (\Imag \lambda)
		\left( \frac{\partial}{\partial \lambda} \right)^s
		\left( -\frac{\partial}{\partial \lambda^*} \right)^r
		\Trace{
			\hat{\rho}
			\hat{D}(\lambda, \lambda^*)
		} \\
	& = \left.
		\left( \frac{\partial}{\partial \lambda} \right)^s
		\left( -\frac{\partial}{\partial \lambda^*} \right)^r
		\chi_W (\lambda, \lambda^*)
	\right|_{\lambda=0}.
\end{eqn}
Now, recognising the final expression as a part of \lmmref{sm-wigner:moments-from-chi},
we immideately get the statement of the theorem.
\end{proof}

\begin{theorem}[Operator correspondences]
\label{thm:sm-wigner:correspondences}
\begin{eqn*}
	\mathcal{W} [ \hat{a} \hat{A} ]
		& = \left( \alpha + \frac{1}{2} \frac{\partial}{\partial \alpha^*} \right) \mathcal{W}[\hat{A}],
	\quad
	\mathcal{W} [ \hat{a}^\dagger \hat{A} ]
		= \left( \alpha^* - \frac{1}{2} \frac{\partial}{\partial \alpha} \right) \mathcal{W}[\hat{A}], \\
	\mathcal{W} [ \hat{A} \hat{a} ]
		& = \left( \alpha - \frac{1}{2} \frac{\partial}{\partial \alpha^*} \right) \mathcal{W}[\hat{A}],
	\quad
	\mathcal{W} [ \hat{A} \hat{a}^\dagger ]
		= \left( \alpha^* + \frac{1}{2} \frac{\partial}{\partial \alpha} \right) \mathcal{W}[\hat{A}].
\end{eqn*}
\end{theorem}
\begin{proof}
We will prove the first correspondence.
First, let us transform the trace using~\eqnref{sm-wigner:displacement-derivatives}:
\begin{eqn}
	\Trace{ \hat{a} \hat{A} \hat{D} }
	& = \Trace{ \hat{A} \hat{D} \hat{a}} \\
	& = \Trace{ \hat{A} \left(
		-\frac{\partial}{\partial \lambda^*}
		-\frac{1}{2} \lambda
	\right) \hat{D}} \\
	& = \left(
		-\frac{\partial}{\partial \lambda^*}
		-\frac{1}{2} \lambda
	\right) \Trace{ \hat{A} \hat{D}}
\end{eqn}
Now we need to somehow move this additional multiplier outside the integral in the expression for Wigner function:
\begin{eqn}
\fl	\mathcal{W} [ \hat{a} \hat{A} ]
	& = \frac{1}{\pi^2} \int d^2 \lambda \exp(-\lambda \alpha^* + \lambda^* \alpha)
		\Trace{ \hat{a} \hat{A} \hat{D}(\lambda, \lambda^*) } \\
\fl	& = \frac{1}{2} \frac{\partial}{\partial \alpha^*} \mathcal{W} [\hat{A}]
	- \frac{1}{\pi^2} \int d^2 \lambda \exp(-\lambda \alpha^* + \lambda^* \alpha)
		\frac{\partial}{\partial \lambda^*}
		\Trace{ \hat{A} \hat{D}(\lambda, \lambda^*) } \\
\fl	& = \frac{1}{2} \frac{\partial}{\partial \alpha^*} \mathcal{W} [\hat{A}]
	+ \frac{1}{\pi^2} \int d^2 \lambda \left(
		\frac{\partial}{\partial \lambda^*} \exp(-\lambda \alpha^* + \lambda^* \alpha)
	\right)
	\Trace{ \hat{A} \hat{D}(\lambda, \lambda^*) } \\
\fl	& = \left( \alpha + \frac{1}{2} \frac{\partial}{\partial \alpha^*} \right) \mathcal{W} [\hat{A}].
\end{eqn}
Note that we used~\lmmref{sm-wigner:zero-integrals} to move the partial derivative over $\lambda^*$.
\end{proof}


\bibliographystyle{apsrev}
\bibliography{qsim-pra}

\end{document}