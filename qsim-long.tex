\documentclass[12pt]{iopart}
\usepackage{iopams}
\usepackage{amsthm}
\usepackage{dsfont}
\usepackage[pdftex]{graphicx}
\usepackage{bbm}
\usepackage{braket}
\usepackage{wasysym}

\newcommand{\alphavec}{\boldsymbol{\alpha}}
\newcommand{\jvec}{\boldsymbol{j}}
\newcommand{\kvec}{\boldsymbol{k}}
\newcommand{\lvec}{\boldsymbol{l}}
\newcommand{\mvec}{\boldsymbol{m}}
\newcommand{\nvec}{\boldsymbol{n}}
\newcommand{\pvec}{\boldsymbol{p}}
\newcommand{\xvec}{\boldsymbol{x}}

%\newcommand{\Tr}{\operatorname{Tr}}
\newcommand{\Trace}[1]{\Tr \left\{ #1 \right\}}

\newcommand{\symprod}[1]{\left\{ #1 \right\}_{\mathrm{sym}}}
\newcommand{\pathavg}[1]{\langle #1 \rangle_{\mathrm{paths}}}
\newcommand{\Real}{\mathrm{Re}}
\newcommand{\Imag}{\mathrm{Im}}

\newcommand{\Psivec}{\boldsymbol{\Psi}}
\newcommand{\Psiop}{\hat{\Psi}}
\newcommand{\Psiopvec}{\hat{\boldsymbol{\Psi}}}

\def\starteqalign#1\end{\eqalign{#1}\end} % magic to wrap \eqalign{} in the environment
\newenvironment{eqn}
	{\begin{eqnarray}\starteqalign}
	{\end{eqnarray}}
\newenvironment{eqns}
	{\begin{eqnarray}}
	{\end{eqnarray}}
\newenvironment{eqn*}
	{\begin{eqnarray*}}
	{\end{eqnarray*}}

\newcommand{\binom}[2]{{#1 \choose #2}}

\newcommand{\eqnref}[1]{\eref{eqn:#1}}
\newcommand{\figref}[1]{Fig.~\ref{fig:#1}}
\newcommand{\thmref}[1]{Theorem~\ref{thm:#1}}
\newcommand{\lmmref}[1]{Lemma~\ref{lmm:#1}}

\newtheorem{theorem}{Theorem}
\newtheorem{definition}{Definition}
\newtheorem{lemma}{Lemma}

\newcommand{\swinaffiliation}{Centre for Atom Optics and Ultrafast Spectroscopy, Swinburne University of Technology, Hawthorn, VIC 3122, Australia}

\begin{document}
\title{Wigner representation of BEC}

\author{B.~Opanchuk}
\author{P.~D.~Drummond}
\address{\swinaffiliation}

\date{\today}
\begin{abstract}
Abstract goes here.
\end{abstract}

% Uncomment to set PACs
%\pacs{}

% uncomment if the separate page for title is needed
%\maketitle

% =============================================================================
\section{Introduction}
% =============================================================================

Introduction goes here.

% =============================================================================
\section{Wirtinger differentiation}
% =============================================================================

In this paper we are using differentiation of complex functions extensively.
Instead of classical definition of the differential which only works for holomorphic functions we use Wirtinger differentiation~\cite{Wirtinger1927}.
One can find thorough description of these rules, for example, in~\cite{Kreutz-Delgado2009}; in this section we will only outline the basics.

\begin{definition}
	For a complex variable $z = x + iy$ and a function $f(z) = u(x, y) + iv(x, y)$ the Wirtinger differential is
	\begin{equation*}
		\frac{df(z)}{dz}
		= \frac{1}{2} \left(
			\frac{\partial f}{\partial x} - i \frac{\partial f}{\partial y}
		\right).
	\end{equation*}
\end{definition}

One can easily prove that if $f(z)$ is holomorphic, then the above definition coincides with the classical differential for complex functions.
Wirtinger differential obeys sum, product, quotient, and chain differentiation rules (the former one is applied as if $f(z) \equiv f(z, z^*)$).

We will need some lemmas about integration.
For convenience, we will use the following definition:

\begin{definition}
	For a complex variable $z = x + iy$ the integral
	\begin{equation*}
		\int d^2 z \equiv \int_{-\infty}^{\infty} \int_{-\infty}^{\infty} dx\, dy,
	\end{equation*}
	or, in other words, stands for the two-dimensional integral over the complex plane.
\end{definition}

\begin{lemma}
\label{lmm:c-numbers:fourier-of-moments}
	If $\alpha$ and $\lambda$ are complex variables,
	then for any non-negative integers $r$ and $s$:
	\begin{equation}
	\begin{split}
		& \int d^2\alpha\, \alpha^r (\alpha^*)^s \exp(-\lambda \alpha^* + \lambda^* \alpha) \\
		& = \pi^2
			\left( -\frac{\partial}{\partial \lambda^*} \right)^r
			\left( \frac{\partial}{\partial \lambda} \right)^s
			\delta(\Real \lambda) \delta(\Imag \lambda)
	\end{split}
	\end{equation}
\end{lemma}
\begin{proof}
First, changing the variables in the integrals and using known Fourier transform relations, we can prove that for real $x$ and $v$, and non-negative integer $n$
\begin{equation*}
	\int\limits_{-\infty}^{\infty} dv\, v^n \exp(\pm 2 i x v)
	= \pi (\mp i / 2)^n \delta^{(n)}(x),
\end{equation*}
Expanding the $\alpha^r (\alpha^*)^s$ term using binomial theorem and using the above property, one can reach the statement of the lemma.
\end{proof}

A notable special case of \lmmref{c-numbers:fourier-of-moments} is
\begin{equation*}
	\int d^2\alpha \exp(-\lambda \alpha^* + \lambda^* \alpha)
	= \pi^2 \delta(\Real \lambda) \delta(\Imag \lambda).
\end{equation*}

\begin{lemma}
\label{lmm:c-numbers:zero-integrals}
	For any non-negative integers $r$, $s$ and complex $\alpha$:
	\begin{equation*}
	\begin{split}
		\int d^2\lambda
			\frac{\partial}{\partial \lambda} \left(
				\exp(-\lambda \alpha^* + \lambda^* \alpha)
				\exp(ixy) x^r y^s
			\right)
		& = 0, \\
		\int d^2\lambda
			\frac{\partial}{\partial \lambda^*}
			\left(
				\exp(-\lambda \alpha^* + \lambda^* \alpha)
				\exp(ixy) x^r y^s
			\right)
		& = 0,
	\end{split}
	\end{equation*}
	where $\lambda = x + iy$.
\end{lemma}
\begin{proof}
We will prove the first equation.
First, note that complex-valued integral of derivative is evaluated as
\begin{equation*}
\begin{split}
	\int d^2\lambda \frac{\partial}{\partial \lambda} f(\lambda, \lambda^*)
	& =	\frac{1}{2} \int\limits_{-\infty}^{\infty} dy \left(
			\left. g(x, y) \right|_{x=-\infty}^{\infty}
		\right) \\
	& - \frac{i}{2} \int\limits_{-\infty}^{\infty} dx \left(
			\left. h(x, y) \right|_{y=-\infty}^{\infty}
		\right),
\end{split}
\end{equation*}
where we expanded $f = g + ih$.
Thus
\begin{equation*}
\begin{split}
	& \int d^2\lambda
		\frac{\partial}{\partial \lambda} \left(
			\exp(-\lambda \alpha^* + \lambda^* \alpha)
			\exp(ixy) x^r y^s
		\right) \\
	& = \left(
			\frac{1}{2} \exp(2ixv) x^r \int dy \exp(iy(x-2u)) y^s
		\right)_{x = -\infty}^\infty \\
	& - \left(
			\frac{i}{2} \exp(-2ixy) y^s \int dx \exp(ix(y+2v)) x^r
		\right)_{y = -\infty}^\infty \\
	& = \left(
			\frac{1}{2} \exp(2ixv) x^r 2 \pi i^s \delta^{(s)}(x-2u)
		\right)_{x = -\infty}^\infty \\
	& - \left(
			\frac{i}{2} \exp(-2ixy) y^s 2 \pi i^r \delta^{(r)}(y+2v)
		\right)_{y = -\infty}^\infty \\
	& = 0,
\end{split}
\end{equation*}
because any derivative of delta function is zero on the infinity.
\end{proof}

% =============================================================================
\section{Single-mode Wigner representation}
% =============================================================================

We will need the displacement operator which was first introduced by Weyl~\cite{Weyl1950}:

\begin{definition}
\label{def:sm-wigner:displacement-op}
	If $\hat{a}^\dagger$ and $\hat{a}$ are bosonic creation and annihilation operators, displacement operator $\hat{D}$ is
	\begin{equation*}
		\hat{D}(\lambda, \lambda^*) = \exp(\lambda \hat{a}^\dagger - \lambda^* \hat{a}).
	\end{equation*}
\end{definition}

Using Baker-Hausdorff theorem to split non-commuting operators in the exponent, one can find that
\begin{eqn}
\label{eqn:sm-wigner:displacement-derivatives}
	\frac{\partial}{\partial \lambda} \hat{D}(\lambda, \lambda^*)
	= \hat{D}(\lambda, \lambda^*) (\hat{a}^\dagger + \frac{1}{2} \lambda^*)
	= (\hat{a}^\dagger - \frac{1}{2} \lambda^*) \hat{D}(\lambda, \lambda^*), \\
	-\frac{\partial}{\partial \lambda^*} \hat{D}(\lambda, \lambda^*)
	= \hat{D}(\lambda, \lambda^*) (\hat{a} + \frac{1}{2} \lambda)
	= (\hat{a} - \frac{1}{2} \lambda) \hat{D}(\lambda, \lambda^*).
\end{eqn}

Wigner transformation converts an operator $\hat{A}$ on a Hilbert space to a function $\mathcal{W}[\hat{A}](\alpha, \alpha^*)$ on phase space.
In terms of the displacement operator Wigner transformation $\mathcal{W}$ and Wigner function $W$ can be defined as

\begin{definition}
\label{def:sm-wigner:w-transformation}
	Wigner transformation of an operator $\hat{A}$:
	\begin{eqn*}
		\mathcal{W}[\hat{A}]
		= \frac{1}{\pi^2} \int d^2 \lambda \exp(-\lambda \alpha^* + \lambda^* \alpha)
			\Trace{ \hat{A} \hat{D}(\lambda, \lambda^*) }.
	\end{eqn*}
	Wigner function is a Wigner transformation of a density matrix:
	\begin{eqn*}
		W(\alpha, \alpha^*) \equiv \mathcal{W}[\hat{\rho}].
	\end{eqn*}
\end{definition}

The Wigner function always exists for any density matrix~\cite{Gardiner2004}.
The correspondence $W \leftrightarrow \hat{\rho}$ is a bijection.
In some cases it is convenient to use the Wigner function in form
\begin{equation}
\label{eqn:sm-wigner:w-function}
	W (\alpha, \alpha^*)
	= \frac{1}{\pi^2} \int d^2 \lambda \exp(-\lambda \alpha^* + \lambda^* \alpha)
		\chi_W (\lambda, \lambda^*),
\end{equation}
where $\chi_W (\lambda, \lambda^*)$ is the characteristic function:
\begin{equation}
	\chi_W (\lambda, \lambda^*)
	= \Trace{ \hat{\rho} \hat{D}(\lambda, \lambda^*) }.
\end{equation}

\begin{lemma}
\label{lmm:sm-wigner:zero-integrals}
	For any integer $m$ and $n$
	\begin{eqn*}
		\int d^2\lambda
			\frac{\partial}{\partial \lambda} & \left(
				\exp(-\lambda \alpha^* + \lambda^* \alpha)
				\left( \frac{\partial}{\partial \lambda} \right)^m
				\left( -\frac{\partial}{\partial \lambda^*} \right)^n
				\hat{D}(\lambda, \lambda^*)
			\right)
		= 0, \\
		\int d^2\lambda
			\frac{\partial}{\partial \lambda^*} & \left(
				\exp(-\lambda \alpha^* + \lambda^* \alpha)
				\left( \frac{\partial}{\partial \lambda} \right)^m
				\left( -\frac{\partial}{\partial \lambda^*} \right)^n
				\hat{D}(\lambda, \lambda^*)
			\right)
		= 0.
	\end{eqn*}
\end{lemma}
\begin{proof}
We will prove the first equation.
Expanding $\lambda = x + iy$ and applying Baker-Hausdorff theorem:
\begin{eqn}
	\hat{D}(\lambda, \lambda^*)
	= \exp(ixy) \exp(x(\hat{a}^\dagger - \hat{a})) \exp(iy(\hat{a}^\dagger + \hat{a}))
\end{eqn}
Expanding derivatives over $\partial/\partial\lambda$ and $\partial/\partial\lambda^*$ in terms of $\partial/\partial x$ and $\partial/\partial y$, and exponents in the expression for $\hat{D}$ as power series:
\begin{eqn}
	\int d^2\lambda
		\frac{\partial}{\partial \lambda} \left(
			\exp(-\lambda \alpha^* + \lambda^* \alpha)
			\left( \frac{\partial}{\partial \lambda} \right)^m
			\left( -\frac{\partial}{\partial \lambda^*} \right)^n
			\hat{D}(\lambda, \lambda^*)
		\right) \\
	= \sum_{r=0}^{\infty} \sum_{s=0}^{\infty} \left(
			\int d^2\lambda
			\frac{\partial}{\partial \lambda} \left(
				\exp(-\lambda \alpha^* + \lambda^* \alpha)
				\exp(ixy) f_{mnrs}(x, y)
			\right)
		\right)
		g_{rs}(\hat{a}, \hat{a}^\dagger) \\
	= 0,
\end{eqn}
where $f_{mnrs}(x, y)$ and $g_{rs}(\hat{a}, \hat{a}^\dagger)$ are some finite-order polynomials,
and we used \lmmref{c-numbers:zero-integrals} to evaluate integrals over $\lambda$.
\end{proof}

\begin{lemma}
	\label{lmm:sm-wigner:moments-from-chi}
	For any integer $r$ and $s$
	\begin{eqn*}
		\langle \symprod{ \hat{a}^r (\hat{a}^\dagger)^s } \rangle
		= \left.
			\left( \frac{\partial}{\partial \lambda} \right)^s
			\left( -\frac{\partial}{\partial \lambda^*} \right)^r
			\chi_W (\lambda, \lambda^*)
		\right|_{\lambda=0}.
	\end{eqn*}
\end{lemma}
\begin{proof}
The exponent in the $\chi_W$ can be expanded as
\begin{eqn}
	\exp (\lambda \hat{a}^\dagger - \lambda^* \hat{a})
	= \sum_{r,s}
		\frac{(-\lambda^*)^r \lambda^s}{r!s!}
		\symprod{ \hat{a}^r (\hat{a}^\dagger)^s }.
\end{eqn}
Thus
\begin{eqn}
	\chi_W(\lambda, \lambda^*)
	& = \sum_{r,s}
		\frac{(-\lambda^*)^r \lambda^s}{r!s!}
		\Trace{
			\hat{\rho} \symprod{ \hat{a}^r (\hat{a}^\dagger)^s }
		} \\
	& = \sum_{r,s}
		\frac{(-\lambda^*)^r \lambda^s}{r!s!}
		\langle \symprod{ \hat{a}^r (\hat{a}^\dagger)^s } \rangle.
\end{eqn}
Apparently, the application of $(\partial / \partial \lambda)^s$ and $(-\partial / \partial \lambda^*)^r$ will eliminate all lower order moments, and setting $\lambda = 0$ afterwards will eliminate all higher order moments, leaving only $\symprod{ \hat{a}^r (\hat{a}^\dagger)^s }$:
\begin{eqn}
	\left.
		\left( \frac{\partial}{\partial \lambda} \right)^s
		\left( -\frac{\partial}{\partial \lambda^*} \right)^r
		\chi_W (\lambda, \lambda^*)
	\right|_{\lambda=0}
	& = r! s! \frac{1}{r! s!}
		\langle \symprod{ \hat{a}^r (\hat{a}^\dagger)^s } \rangle \\
	& = \langle \symprod{ \hat{a}^r (\hat{a}^\dagger)^s } \rangle.
	\qedhere
\end{eqn}
\end{proof}

Now we can get the final relation.

\begin{theorem}
\label{thm:sm-wigner:moments}
	Expectations of symmetrically ordered operator products are moments of the Wigner function:
	\begin{eqn*}
		\langle \symprod{ \hat{a}^r (\hat{a}^\dagger)^s } \rangle
		= \int d^2\alpha\, \alpha^r (\alpha^*)^s W(\alpha, \alpha^*)
	\end{eqn*}
\end{theorem}
\begin{proof}
By definition of the Wigner function:
\begin{eqn}
\int d^2\alpha\, \alpha^r (\alpha^*)^s W(\alpha, \alpha^*) \\
	= \frac{1}{\pi^2} \Trace{ \hat{\rho}
			\int d^2\alpha\, \alpha^r (\alpha^*)^s
			\int d^2\lambda \exp(-\lambda \alpha^* + \lambda^* \alpha)
			\hat{D}(\lambda, \lambda^*)
		}
\end{eqn}
Integrating by parts and eliminating terms which fit \lmmref{sm-wigner:zero-integrals}:
\begin{eqn}
\fl	= \frac{1}{\pi^2} \Trace{ \hat{\rho}
			\int d^2\alpha \int d^2\lambda
			\exp(-\lambda \alpha^* + \lambda^* \alpha)
			\left( \frac{\partial}{\partial \lambda} \right)^s
			\left( -\frac{\partial}{\partial \lambda^*} \right)^r
			\hat{D} (\lambda, \lambda^*)
		}
\end{eqn}
Evaluating integral over $\alpha$ using \lmmref{c-numbers:fourier-of-moments}:
\begin{eqn}
	& = \int d^2\lambda\,
		\delta (\Real \lambda) \delta (\Imag \lambda)
		\left( \frac{\partial}{\partial \lambda} \right)^s
		\left( -\frac{\partial}{\partial \lambda^*} \right)^r
		\Trace{
			\hat{\rho}
			\hat{D}(\lambda, \lambda^*)
		} \\
	& = \left.
		\left( \frac{\partial}{\partial \lambda} \right)^s
		\left( -\frac{\partial}{\partial \lambda^*} \right)^r
		\chi_W (\lambda, \lambda^*)
	\right|_{\lambda=0}.
\end{eqn}
Now, recognising the final expression as a part of \lmmref{sm-wigner:moments-from-chi},
we immideately get the statement of the theorem.
\end{proof}

\begin{theorem}[Operator correspondences]
\label{thm:sm-wigner:correspondences}
\begin{eqn*}
	\mathcal{W} [ \hat{a} \hat{A} ]
		& = \left( \alpha + \frac{1}{2} \frac{\partial}{\partial \alpha^*} \right) \mathcal{W}[\hat{A}],
	\quad
	\mathcal{W} [ \hat{a}^\dagger \hat{A} ]
		= \left( \alpha^* - \frac{1}{2} \frac{\partial}{\partial \alpha} \right) \mathcal{W}[\hat{A}], \\
	\mathcal{W} [ \hat{A} \hat{a} ]
		& = \left( \alpha - \frac{1}{2} \frac{\partial}{\partial \alpha^*} \right) \mathcal{W}[\hat{A}],
	\quad
	\mathcal{W} [ \hat{A} \hat{a}^\dagger ]
		= \left( \alpha^* + \frac{1}{2} \frac{\partial}{\partial \alpha} \right) \mathcal{W}[\hat{A}].
\end{eqn*}
\end{theorem}
\begin{proof}
We will prove the first correspondence.
First, let us transform the trace using~\eqnref{sm-wigner:displacement-derivatives}:
\begin{eqn}
	\Trace{ \hat{a} \hat{A} \hat{D} }
	& = \Trace{ \hat{A} \hat{D} \hat{a}} \\
	& = \Trace{ \hat{A} \left(
		-\frac{\partial}{\partial \lambda^*}
		-\frac{1}{2} \lambda
	\right) \hat{D}} \\
	& = \left(
		-\frac{\partial}{\partial \lambda^*}
		-\frac{1}{2} \lambda
	\right) \Trace{ \hat{A} \hat{D}}
\end{eqn}
Now we need to somehow move this additional multiplier outside the integral in the expression for Wigner function:
\begin{eqn}
\fl	\mathcal{W} [ \hat{a} \hat{A} ]
	& = \frac{1}{\pi^2} \int d^2 \lambda \exp(-\lambda \alpha^* + \lambda^* \alpha)
		\Trace{ \hat{a} \hat{A} \hat{D}(\lambda, \lambda^*) } \\
\fl	& = \frac{1}{2} \frac{\partial}{\partial \alpha^*} \mathcal{W} [\hat{A}]
	- \frac{1}{\pi^2} \int d^2 \lambda \exp(-\lambda \alpha^* + \lambda^* \alpha)
		\frac{\partial}{\partial \lambda^*}
		\Trace{ \hat{A} \hat{D}(\lambda, \lambda^*) } \\
\fl	& = \frac{1}{2} \frac{\partial}{\partial \alpha^*} \mathcal{W} [\hat{A}]
	+ \frac{1}{\pi^2} \int d^2 \lambda \left(
		\frac{\partial}{\partial \lambda^*} \exp(-\lambda \alpha^* + \lambda^* \alpha)
	\right)
	\Trace{ \hat{A} \hat{D}(\lambda, \lambda^*) } \\
\fl	& = \left( \alpha + \frac{1}{2} \frac{\partial}{\partial \alpha^*} \right) \mathcal{W} [\hat{A}].
\end{eqn}
Note that we used~\lmmref{sm-wigner:zero-integrals} to move the partial derivative over $\lambda^*$.
\end{proof}

% =============================================================================
\section{Functional calculus}
% =============================================================================

This section outlines the functional calculus, which is heavily used throughout the paper.
Detailed description is given in~\cite{Dalton2011}, and here we only provide some important definitions and results which are used later in the paper.
In this section we will use the definitions from the \secref{func-operators}, namely the full basis $\fullbasis$ and the restricted basis $\restbasis$.
Given the basis, we can define the correspondence between some function of coordinates and its representation in mode space.

\begin{definition}
	Let $\mathbb{F}$ be a space of all functions of coordinates, which consists only of modes from $\restbasis$: $\mathbb{F}_{\restbasis} \equiv (\mathbb{R}^D \rightarrow \mathbb{C})_{\restbasis}$ (restricted functions).
	Composition transformation creates a function from a vector of mode populations:
	\begin{eqn*}
		\mathcal{C}_{\restbasis} :: \mathbb{C}^{|\restbasis|} \rightarrow \mathbb{F}_{\restbasis} \\
		\mathcal{C}_{\restbasis}(\balpha) = \sum_{\nvec \in \restbasis} \phi_{\nvec} \alpha_{\nvec}.
	\end{eqn*}
	Decomposition transformation, correspondingly, creates a vector of populations out of a function:
	\begin{eqn*}
		\mathcal{C}_{\restbasis}^{-1} :: \mathbb{F} \rightarrow \mathbb{C}^{|\restbasis|} \\
		(\mathcal{C}_{\restbasis}^{-1}[f])_{\nvec}
		= \int d\xvec \phi_{\nvec}^* f,\,{\nvec} \in \restbasis.
	\end{eqn*}
	Note that for any $f \in \mathbb{F}_{\restbasis}$, $\mathcal{C}_{\restbasis}(\mathcal{C}_{\restbasis}^{-1}[f]) \equiv f$.
\end{definition}

The result of any non-linear transformation of a function $f \in \mathbb{F}_{\restbasis}$ is not guaranteed to belong to $\mathbb{F}_{\restbasis}$ and requires explicit projection to be used with other restricted functions.
This applies to the delta function of coordinates.
To avoid confusion with the common delta function, we introduce the restricted delta function.

\begin{definition}
\label{def:func-calculus:restricted-delta}
	The restricted delta function $\delta_{\restbasis} \in \mathbb{F}_{\restbasis}$ is defined as
	\begin{eqn*}
		\delta_{\restbasis}(\xvec^\prime, \xvec)
		= \sum_{\nvec \in \restbasis} \phi_{\nvec}^{\prime*} \phi_{\nvec}.
	\end{eqn*}
	Note that $\delta_{\restbasis}^*(\xvec^\prime, \xvec) = \delta_{\restbasis}(\xvec, \xvec^\prime)$.
\end{definition}

Any function can be projected to $\restbasis$ using the projection transformation.

\begin{definition}
\label{def:func-calculus:projector}
	Projection transformation
	\begin{eqn*}
		\mathcal{P}_{\restbasis} ::
		\mathbb{F} \rightarrow \mathbb{F}_{\restbasis} \\
		\mathcal{P}_{\restbasis}[f](\xvec)
		& = (\mathcal{C}_{\restbasis}(\mathcal{C}_{\restbasis}^{-1}[f])) (\xvec) \\
		& = \sum_{\nvec \in \restbasis} \phi_{\nvec} \int
			d\xvec^\prime\, \phi_{\nvec}^{\prime*} f^\prime \\
		& = \int d\xvec^\prime \delta_{\restbasis}(\xvec^\prime, \xvec) f^\prime,
	\end{eqn*}
	Obviously, $\mathcal{P}_{\fullbasis} \equiv \mathds{1}$.
\end{definition}

The conjugate of $\mathcal{P}_{\restbasis}$ is thus defined as
\begin{eqn}
	(\mathcal{P}_{\restbasis}[f](\xvec))^*
	= \int d\xvec^\prime \delta_{\restbasis}^*(\xvec^\prime, \xvec) f^{\prime*}
	= \mathcal{P}_{\restbasis}^* [f^*](\xvec).
\end{eqn}

Let $\mathcal{F}[f] :: \mathbb{F}_{\restbasis} \rightarrow \mathbb{F}$ be some transformation (note that the result is not guaranteed to belong to the restricted basis).
Because of the bijection between $\mathbb{F}_{\restbasis}$ and $\mathbb{C}^{|\restbasis|}$, $\mathcal{F}$ can be alternatively treated as a function of a vector of complex numbers:
\begin{eqn}
	\mathcal{F} :: \mathbb{C}^{|\restbasis|} \rightarrow \mathbb{C}^\infty \\
	\mathcal{F}(\balpha) \equiv \mathcal{C}_{\restbasis}^{-1}[\mathcal{F}[\mathcal{C}_{\restbasis}(\balpha)]].
\end{eqn}
Using this correspondence, we can define the functional differentiation.

\begin{definition}
\label{def:func-calculus:func-diff}
	Functional derivative is defined as
	\begin{eqn*}
		\frac{\delta}{\delta f^\prime} ::
		\left(
			\mathbb{F}_{\restbasis} \rightarrow \mathbb{F}
		\right)
		\rightarrow
		\left(
			\mathbb{R}^D \rightarrow \mathbb{F}_{\restbasis} \rightarrow \mathbb{F}
		\right) \\
		\frac{\delta \mathcal{F}[f]}{\delta f^\prime}
		= \sum_{\nvec \in \restbasis} \phi_{\nvec}^{\prime*}
			\frac{\partial \mathcal{F}(\balpha)}{\partial \alpha_{\nvec}}.
	\end{eqn*}
\end{definition}

Note that the transformation being returned differs from the one which was taken: the result of the new transformation is a function of the additional variable from $\mathbb{R}^D$ ($\xvec^\prime$).
This variable comes from the function we are differentiating by.

Functional derivatives behave in many ways similar to Wirtinger derivatives.
The detailed treatment can be found in~\cite{Dalton2011}.
In particular, the following useful lemma gives us the ability to differentiate functionals based on the intuition for common functions:

\begin{lemma}
	If $g(z)$ is a function of complex variable that can be expanded into series of $z^n (z^*)^m$, and functional $\mathcal{F}[f, f^*] \equiv g(f, f^*)$, $\mathcal{F} \in \mathbb{F}_{\restbasis} \rightarrow \mathbb{F}$, then $\delta \mathcal{F} / \delta f^\prime$ and $\delta \mathcal{F} / \delta f^{\prime*}$ can be treated as partial differentiation of the functional of two independent variables $f$ and $f^*$.
	In other words:
	\begin{eqn*}
		\frac{\delta \mathcal{F}}{\delta f^\prime}
		= \delta_{\restbasis}(\xvec^\prime, \xvec)
			\frac{\partial g(f, f^*)}{\partial f},
		\quad
		\frac{\delta \mathcal{F}}{\delta f^{\prime*}}
		= \delta_{\restbasis}^*(\xvec^\prime, \xvec)
			\frac{\partial g(f, f^*)}{\partial f^*}
	\end{eqn*}
\end{lemma}

Functional integration is defined as

\begin{definition}
	\begin{eqn*}
		\int \delta^2 f :: (\mathbb{F}_{\restbasis} \rightarrow \mathbb{F}) \rightarrow \mathbb{C} \\
		\int \delta^2 f \mathcal{F}[f]
		= \int d^2\balpha \mathcal{F}(\balpha)
		= \left(
			\prod_{\nvec \in \restbasis} \int d^2\alpha_{\nvec}
		\right) \mathcal{F}(\balpha).
	\end{eqn*}
    If the basis contains an infinite number of modes, the integral is treated as a limit $|\restbasis| \rightarrow \infty$.
    \todo{Product of integrals means successive applications of those integrals --- do we need to state it explicitly?}
\end{definition}

Functional integration has the Fourier-like property analogous to Lemma~\lmmref{c-numbers:fourier-of-moments}, but its statement requires the definition of the delta functional:

\begin{definition}
\label{def:func-calculus:delta-functional}
	For a function $\Lambda \in \mathbb{F}_{\restbasis}$ the delta functional is
	\begin{eqn*}
		\Delta_{\restbasis}[\Lambda]
		\equiv \prod_{\nvec \in \restbasis} \delta(\Real \lambda_{\nvec}) \delta(\Imag \lambda_{\nvec}),
	\end{eqn*}
	where $\blambda = \mathcal{C}_{\restbasis}^{-1}[\Lambda]$.
\end{definition}

The delta functional has the same property as the common delta function:
\begin{eqn}
	\int \delta^2 \Lambda \mathcal{F}[\Lambda] \Delta_{\restbasis}[\Lambda]
	& = \left(
			\prod_{\nvec \in \restbasis} \int d^2\lambda_{\nvec}
		\right)
		\mathcal{F}(\blambda)
		\prod_{\nvec \in \restbasis} \delta(\Real \lambda_{\nvec}) \delta(\Imag \lambda_{\nvec}) \\
	& = \left. \mathcal{F}(\blambda) \right|_{\forall \nvec \in \restbasis\, \lambda_{\nvec} = 0} \\
	& = \left. \mathcal{F}[\Lambda] \right|_{\Lambda \equiv 0}
\end{eqn}

\begin{lemma}[Functional extension of \lmmref{c-numbers:fourier-of-moments}]
\label{lmm:func-calculus:fourier-of-moments}
	For $\Psi \in \mathbb{F}_{\restbasis}$ and $\Lambda \in \mathbb{F}_{\restbasis}$, and for any non-negative integers $r$ and $s$:
	\begin{eqn*}
		\int \delta^2\Psi\, \Psi^r (\Psi^*)^s \exp \left(
				\int d\xvec \left( -\Lambda \Psi^* + \Lambda^* \Psi \right)
			\right) \\
		= \pi^{2|\restbasis|}
			\left( -\frac{\delta}{\delta \Lambda^*} \right)^r
			\left( \frac{\delta}{\delta \Lambda} \right)^s
			\Delta_{\restbasis}[\Lambda]
	\end{eqn*}
\end{lemma}
\begin{proof}
The proof consists of expanding functions into sums of modes and applying \lmmref{c-numbers:fourier-of-moments} $|\restbasis|$ times.
\end{proof}

\begin{lemma}
\label{lmm:func-calculus:zero-integrals}
	For a bounded functional $F(\blambda, \blambda^*)$
	\begin{eqn*}
		\int \delta^2\Lambda
			\frac{\delta}{\delta \Lambda^\prime} \left(
				D[\Lambda, \Lambda^*, \Psi, \Psi^*]
				F[\Lambda, \Lambda^*]
			\right)
		& = 0 \\
		\int \delta^2\Lambda
			\frac{\delta}{\delta \Lambda^{\prime*}}
			\left(
				D[\Lambda, \Lambda^*, \Psi, \Psi^*]
				F[\Lambda, \Lambda^*]
			\right)
		& = 0.
	\end{eqn*}
\end{lemma}
\begin{proof}
Proved by expanding integrals and differentials into modes and applying \lmmref{c-numbers:zero-integrals}.
\end{proof}

\begin{lemma}
\label{lmm:func-calculus:zero-delta-integrals}
	For $\Lambda \in \mathbb{F}_{\restbasis}$ \todo{Any limitations on $F$?}
	\begin{eqn*}
		\int \delta^2\Lambda
			\frac{\delta}{\delta \Lambda} \left(
				\left(
					\left( \frac{\delta}{\delta \Lambda} \right)^s
					\left( -\frac{\delta}{\delta \Lambda^*} \right)^r
					\Delta_{\restbasis}[\Lambda]
				\right)
				F[\lambda, \lambda^*]
			\right)
		& = 0 \\
		\int \delta^2\Lambda
			\frac{\delta}{\delta \Lambda^*} \left(
				\left(
					\left( \frac{\delta}{\delta \Lambda} \right)^s
					\left( -\frac{\delta}{\delta \Lambda^*} \right)^r
					\Delta_{\restbasis}[\Lambda]
				\right)
				F[\lambda, \lambda^*]
			\right)
		& = 0 \\
	\end{eqn*}
\end{lemma}
\begin{proof}
Proved by expanding functional integration and differentials into modes and integrating separately over each $\lambda_{\nvec}$, using the fact that any differential of the delta function is zero on the infinity.
\end{proof}

In order to perform transformations of master equations, we will need a lemma that justifies the ``relocation'' of the Laplacian (which is a part of the kinetic term in the Hamiltonian) inside the functional integral.

\begin{lemma}
\label{lmm:func-calculus:move-laplacian}
	If $\mathcal{F} \in \mathbb{F}_{\restbasis} \rightarrow \mathbb{F}$, and $\forall \nvec \in \restbasis, \xvec \in \partial A$ $\phi_{\nvec}(\xvec) = 0$, then
	\begin{eqn*}
		\int\limits_A d\xvec \left(
			\nabla^2 \frac{\delta}{\delta \Psi}
		\right) \Psi \mathcal{F}[\Psi, \Psi^*]
		= \int\limits_A d\xvec \frac{\delta}{\delta \Psi}
		( \nabla^2 \Psi ) \mathcal{F}[\Psi, \Psi^*]
	\end{eqn*}
\end{lemma}
\begin{proof}
The proof consists of a function expansion into a mode sum and an application of Green's first identity.
\end{proof}

Note that the above lemma imposes an additional requirement for basis functions, but in practical applications it is always satisfied.
For example, in plane wave basis eigenfunctions are equal to zero at the border of the bounding box, and in harmonic oscillator basis they are equal to zero on the infinity (which can be considered the boundary of their integration area).
Hereinafter we will assume that this condition is true for any basis we work with.


\section*{References}
\bibliographystyle{unsrt}
\bibliography{qsim-long}

\end{document}