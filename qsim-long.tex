\documentclass[12pt]{iopart}
\usepackage{iopams}
\usepackage{amsthm}
\usepackage{dsfont}


\usepackage{color}
\newcommand{\todo}[1]{\textcolor{red}{[#1]}}

\newcommand{\jvec}{\boldsymbol{j}}
\newcommand{\kvec}{\boldsymbol{k}}
\newcommand{\lvec}{\boldsymbol{l}}
\newcommand{\mvec}{\boldsymbol{m}}
\newcommand{\nvec}{\boldsymbol{n}}
\newcommand{\pvec}{\boldsymbol{p}}
\newcommand{\xvec}{\boldsymbol{x}}
\newcommand{\zvec}{\boldsymbol{z}}
\newcommand{\Zvec}{\boldsymbol{Z}}

%\newcommand{\Tr}{\operatorname{Tr}}
\newcommand{\Trace}[1]{\Tr \left\{ #1 \right\}}

\newcommand{\symprod}[1]{\left\{ #1 \right\}_{\mathrm{sym}}}
\newcommand{\pathavg}[1]{\langle #1 \rangle_{\mathrm{paths}}}
\newcommand{\Real}{\mathrm{Re}}
\newcommand{\Imag}{\mathrm{Im}}

\newcommand{\Psivec}{\boldsymbol{\Psi}}
\newcommand{\Psiop}{\hat{\Psi}}
\newcommand{\Psiopvec}{\hat{\boldsymbol{\Psi}}}

\newcommand{\ecut}{\epsilon_{\mathrm{cut}}}
\newcommand{\fullbasis}{\mathbb{B}}
\newcommand{\restbasis}{\mathbb{M}}

\def\starteqalign#1\end{\eqalign{#1}\end} % magic to wrap \eqalign{} in the environment
\newenvironment{eqn}
	{\begin{eqnarray}\starteqalign}
	{\end{eqnarray}}
\newenvironment{eqns}
	{\begin{eqnarray}}
	{\end{eqnarray}}
\newenvironment{eqn*}
	{\begin{eqnarray*}}
	{\end{eqnarray*}}

\newcommand{\binom}[2]{{#1 \choose #2}}

\newcommand{\eqnref}[1]{\eref{eqn:#1}}
\newcommand{\figref}[1]{Fig.~\ref{fig:#1}}
\newcommand{\thmref}[1]{Theorem~\ref{thm:#1}}
\newcommand{\lmmref}[1]{Lemma~\ref{lmm:#1}}
\newcommand{\defref}[1]{Definition~\ref{def:#1}}
\newcommand{\secref}[1]{Section~\ref{sec:#1}}

\newtheorem{theorem}{Theorem}
\newtheorem{definition}{Definition}
\newtheorem{lemma}{Lemma}

\newcommand{\swinaffiliation}{Centre for Atom Optics and Ultrafast Spectroscopy, Swinburne University of Technology, Hawthorn, VIC 3122, Australia}

\begin{document}
\title{Wigner representation of BEC}

\author{Bogdan Opanchuk and Peter D Drummond}
\address{\swinaffiliation}
\ead{bogdan@opanchuk.net}

\date{\today}
\begin{abstract}
We develop a method of simulating full quantum dynamics of multi-mode multi-component Bose-Einstein condensate in a trap.
We use truncated Wigner representation to produce stochastic equations that can be solved using conventional methods.
Our approach describes spatial evolution of spinor components and properly accounts for nonlinear losses.
\end{abstract}

% Uncomment to set PACs
%\pacs{}

% uncomment if the separate page for title is needed
%\maketitle

% =============================================================================
\section{Introduction}
% =============================================================================

Introduction goes here.

% =============================================================================
\section{Master equation}
% =============================================================================

In this paper we consider a zero-temperature $C$-component BEC in $D$ effective dimensions.
The Hamiltonian for this system is expressed in terms of bosonic field creation and annihilation operators $\Psiop_j^{\dagger}(\xvec)$ and $\Psiop_j(\xvec)$, $j = 1 \ldots C$, which obey standard bosonic commutation relation
\begin{eqn}
\label{eqn:master-eqn:commutators}
	[ \Psiop_j, \Psiop_k^{\prime\dagger} ]
	= \delta_{jk} \delta(\xvec^\prime-\xvec),
\end{eqn}
where $\xvec \in \mathbb{R}^D$ is a $D$-dimensional coordinate vector, and we dubbed $\Psiop_j \equiv \Psiop_j(\xvec)$ and $\Psiop_k^\prime \equiv \Psiop_k(\xvec^\prime)$ for brevity (hereinafter the same abbreviation will be used for all functions of coordinates).
The second-quantized Hamiltonian for the system is
\begin{eqn}
\label{eqn:master-eqn:hamiltonian}
	\hat{H}
	= \int d\xvec \left\{
		\Psiop_j^{\dagger} K_{jk} \Psiop_k
		+ \frac{1}{2} \int d\xvec^\prime
			\Psiop_j^\dagger \Psiop_k^{\prime\dagger}
			U_{jk}(\xvec - \xvec^\prime)
			\Psiop_j^\prime \Psiop_k
	\right\},
\end{eqn}
where $U_{jk}$ is the two-body scattering potential, and the single-particle Hamiltonian $K_{jk}$ is
\begin{eqn}
\label{eqn:master-eqn:single-particle}
	K_{jk} = \left(
			-\frac{\hbar^2}{2m} \nabla^2 + \hbar \omega_j + V_j(\xvec)
		\right) \delta_{jk}
		+ \hbar \Omega_{jk}(t),
\end{eqn}
where $m$ is the atomic mass, $V_j$ is the external trapping potential for spin $j$, $\hbar \omega_j$ is the internal energy of spin $j$, and $\Omega_{jk}$ represents a time-dependent coupling that is used to rotate one spin projection into another.

If we impose an energy cutoff $\ecut$ and only take into account low-energy modes, the non-local scattering potential $U_{jk}(\xvec - \xvec^\prime)$ can be replaced by the contact potential $U_{jk} \delta(\xvec - \xvec^\prime)$~\cite{Morgan2000}, giving the effective Hamiltonian
\begin{eqn}
\label{eqn:master-eqn:effective-H}
	\hat{H}
	= \int d\xvec \left\{
		\Psiop_j^{\dagger} K_{jk} \Psiop_k
		+ \frac{U_{jk}}{2} \Psiop_j^\dagger \Psiop_k^\dagger \Psiop_j \Psiop_k
	\right\},
\end{eqn}
where $\Psiop_j^{\dagger}$ and $\Psiop_j$ are field operators in the new restricted basis of low-energy modes, which is described in detail in the next section.
For the $s$-wave scattering in three dimensions the coefficient is $U_{jk} = 4 \pi \hbar^2 a_{jk} / m$, where $a_{jk}$ is the scattering length.
Note that in general case the coefficient must be renormalised depending on the grid~\cite{Sinatra2002}, but the change is small if $dx \gg a_{jk}$.

The Markovian master equation for the system with the inclusion of losses can be written as~\cite{Jack2002}
\begin{eqn}
\label{eqn:master-eqn:master-eqn}
	\frac{d\hat{\rho}}{dt} =
		- \frac{i}{\hbar} \left[ \hat{H}, \hat{\rho} \right]
		+ \sum_{\lvec} \kappa_{\lvec} \int d\xvec
			\mathcal{L}_{\lvec} \left[ \hat{\rho} \right],
\end{eqn}
where $\lvec = (l_1, l_2, \ldots, l_C)$ is a tuple indicating the number of atoms from each component involved in the interaction, $C$ being the total number of components, and we have introduced local Liouville loss terms,
\begin{eqn}
\label{eqn:master-eqn:loss-term}
	\mathcal{L}_{\lvec} \left[ \hat{\rho} \right] =
		2\hat{O}_{\lvec} \hat{\rho} \hat{O}_{\lvec}^\dagger
		- \hat{O}_{\lvec}^\dagger \hat{O}_{\lvec} \hat{\rho}
		- \hat{\rho} \hat{O}_{\lvec}^\dagger \hat{O}_{\lvec}.
\end{eqn}
The reservoir coupling operators $\hat{O}_{\lvec}$ are products of local field annihilation operators:
\begin{eqn}
    \hat{O}_{\lvec}
    \equiv \hat{O}_{\lvec} (\Psiopvec)
    = \prod_{c=1}^C \Psiop_c^{l_c} (\xvec),
\end{eqn}
describing local $\left( \sum_{c=1}^C l_c \right)$-body collision losses.

%% =============================================================================
\section{Single-mode Wigner representation}
% =============================================================================

We will need the displacement operator which was first introduced by Weyl~\cite{Weyl1950}:

\begin{definition}
\label{def:sm-wigner:displacement-op}
	If $\hat{a}^\dagger$ and $\hat{a}$ are bosonic creation and annihilation operators, displacement operator $\hat{D}$ is
	\begin{equation*}
		\hat{D}(\lambda, \lambda^*) = \exp(\lambda \hat{a}^\dagger - \lambda^* \hat{a}).
	\end{equation*}
\end{definition}

Using Baker-Hausdorff theorem to split non-commuting operators in the exponent, one can find that
\begin{eqn}
\label{eqn:sm-wigner:displacement-derivatives}
	\frac{\partial}{\partial \lambda} \hat{D}(\lambda, \lambda^*)
	= \hat{D}(\lambda, \lambda^*) (\hat{a}^\dagger + \frac{1}{2} \lambda^*)
	= (\hat{a}^\dagger - \frac{1}{2} \lambda^*) \hat{D}(\lambda, \lambda^*), \\
	-\frac{\partial}{\partial \lambda^*} \hat{D}(\lambda, \lambda^*)
	= \hat{D}(\lambda, \lambda^*) (\hat{a} + \frac{1}{2} \lambda)
	= (\hat{a} - \frac{1}{2} \lambda) \hat{D}(\lambda, \lambda^*).
\end{eqn}

Wigner transformation converts an operator $\hat{A}$ on a Hilbert space to a function $\mathcal{W}[\hat{A}](\alpha, \alpha^*)$ on phase space.
In terms of the displacement operator Wigner transformation $\mathcal{W}$ and Wigner function $W$ can be defined as

\begin{definition}
\label{def:sm-wigner:w-transformation}
	Wigner transformation of an operator $\hat{A}$:
	\begin{eqn*}
		\mathcal{W}[\hat{A}]
		= \frac{1}{\pi^2} \int d^2 \lambda \exp(-\lambda \alpha^* + \lambda^* \alpha)
			\Trace{ \hat{A} \hat{D}(\lambda, \lambda^*) }.
	\end{eqn*}
	Wigner function is a Wigner transformation of a density matrix:
	\begin{eqn*}
		W(\alpha, \alpha^*) \equiv \mathcal{W}[\hat{\rho}].
	\end{eqn*}
\end{definition}

The Wigner function always exists for any density matrix~\cite{Gardiner2004}.
The correspondence $W \leftrightarrow \hat{\rho}$ is a bijection.
In some cases it is convenient to use the Wigner function in form
\begin{equation}
\label{eqn:sm-wigner:w-function}
	W (\alpha, \alpha^*)
	= \frac{1}{\pi^2} \int d^2 \lambda \exp(-\lambda \alpha^* + \lambda^* \alpha)
		\chi_W (\lambda, \lambda^*),
\end{equation}
where $\chi_W (\lambda, \lambda^*)$ is the characteristic function:
\begin{equation}
	\chi_W (\lambda, \lambda^*)
	= \Trace{ \hat{\rho} \hat{D}(\lambda, \lambda^*) }.
\end{equation}

\begin{lemma}
\label{lmm:sm-wigner:zero-integrals}
	For any integer $m$ and $n$
	\begin{eqn*}
		\int d^2\lambda
			\frac{\partial}{\partial \lambda} & \left(
				\exp(-\lambda \alpha^* + \lambda^* \alpha)
				\left( \frac{\partial}{\partial \lambda} \right)^m
				\left( -\frac{\partial}{\partial \lambda^*} \right)^n
				\hat{D}(\lambda, \lambda^*)
			\right)
		= 0, \\
		\int d^2\lambda
			\frac{\partial}{\partial \lambda^*} & \left(
				\exp(-\lambda \alpha^* + \lambda^* \alpha)
				\left( \frac{\partial}{\partial \lambda} \right)^m
				\left( -\frac{\partial}{\partial \lambda^*} \right)^n
				\hat{D}(\lambda, \lambda^*)
			\right)
		= 0.
	\end{eqn*}
\end{lemma}
\begin{proof}
We will prove the first equation.
Expanding $\lambda = x + iy$ and applying Baker-Hausdorff theorem:
\begin{eqn}
	\hat{D}(\lambda, \lambda^*)
	= \exp(ixy) \exp(x(\hat{a}^\dagger - \hat{a})) \exp(iy(\hat{a}^\dagger + \hat{a}))
\end{eqn}
Expanding derivatives over $\partial/\partial\lambda$ and $\partial/\partial\lambda^*$ in terms of $\partial/\partial x$ and $\partial/\partial y$, and exponents in the expression for $\hat{D}$ as power series:
\begin{eqn}
	\int d^2\lambda
		\frac{\partial}{\partial \lambda} \left(
			\exp(-\lambda \alpha^* + \lambda^* \alpha)
			\left( \frac{\partial}{\partial \lambda} \right)^m
			\left( -\frac{\partial}{\partial \lambda^*} \right)^n
			\hat{D}(\lambda, \lambda^*)
		\right) \\
	= \sum_{r=0}^{\infty} \sum_{s=0}^{\infty} \left(
			\int d^2\lambda
			\frac{\partial}{\partial \lambda} \left(
				\exp(-\lambda \alpha^* + \lambda^* \alpha)
				\exp(ixy) f_{mnrs}(x, y)
			\right)
		\right)
		g_{rs}(\hat{a}, \hat{a}^\dagger) \\
	= 0,
\end{eqn}
where $f_{mnrs}(x, y)$ and $g_{rs}(\hat{a}, \hat{a}^\dagger)$ are some finite-order polynomials,
and we used \lmmref{c-numbers:zero-integrals} to evaluate integrals over $\lambda$.
\end{proof}

\begin{lemma}
	\label{lmm:sm-wigner:moments-from-chi}
	For any integer $r$ and $s$
	\begin{eqn*}
		\langle \symprod{ \hat{a}^r (\hat{a}^\dagger)^s } \rangle
		= \left.
			\left( \frac{\partial}{\partial \lambda} \right)^s
			\left( -\frac{\partial}{\partial \lambda^*} \right)^r
			\chi_W (\lambda, \lambda^*)
		\right|_{\lambda=0}.
	\end{eqn*}
\end{lemma}
\begin{proof}
The exponent in the $\chi_W$ can be expanded as
\begin{eqn}
	\exp (\lambda \hat{a}^\dagger - \lambda^* \hat{a})
	= \sum_{r,s}
		\frac{(-\lambda^*)^r \lambda^s}{r!s!}
		\symprod{ \hat{a}^r (\hat{a}^\dagger)^s }.
\end{eqn}
Thus
\begin{eqn}
	\chi_W(\lambda, \lambda^*)
	& = \sum_{r,s}
		\frac{(-\lambda^*)^r \lambda^s}{r!s!}
		\Trace{
			\hat{\rho} \symprod{ \hat{a}^r (\hat{a}^\dagger)^s }
		} \\
	& = \sum_{r,s}
		\frac{(-\lambda^*)^r \lambda^s}{r!s!}
		\langle \symprod{ \hat{a}^r (\hat{a}^\dagger)^s } \rangle.
\end{eqn}
Apparently, the application of $(\partial / \partial \lambda)^s$ and $(-\partial / \partial \lambda^*)^r$ will eliminate all lower order moments, and setting $\lambda = 0$ afterwards will eliminate all higher order moments, leaving only $\symprod{ \hat{a}^r (\hat{a}^\dagger)^s }$:
\begin{eqn}
	\left.
		\left( \frac{\partial}{\partial \lambda} \right)^s
		\left( -\frac{\partial}{\partial \lambda^*} \right)^r
		\chi_W (\lambda, \lambda^*)
	\right|_{\lambda=0}
	& = r! s! \frac{1}{r! s!}
		\langle \symprod{ \hat{a}^r (\hat{a}^\dagger)^s } \rangle \\
	& = \langle \symprod{ \hat{a}^r (\hat{a}^\dagger)^s } \rangle.
	\qedhere
\end{eqn}
\end{proof}

Now we can get the final relation.

\begin{theorem}
\label{thm:sm-wigner:moments}
	Expectations of symmetrically ordered operator products are moments of the Wigner function:
	\begin{eqn*}
		\langle \symprod{ \hat{a}^r (\hat{a}^\dagger)^s } \rangle
		= \int d^2\alpha\, \alpha^r (\alpha^*)^s W(\alpha, \alpha^*)
	\end{eqn*}
\end{theorem}
\begin{proof}
By definition of the Wigner function:
\begin{eqn}
\int d^2\alpha\, \alpha^r (\alpha^*)^s W(\alpha, \alpha^*) \\
	= \frac{1}{\pi^2} \Trace{ \hat{\rho}
			\int d^2\alpha\, \alpha^r (\alpha^*)^s
			\int d^2\lambda \exp(-\lambda \alpha^* + \lambda^* \alpha)
			\hat{D}(\lambda, \lambda^*)
		}
\end{eqn}
Integrating by parts and eliminating terms which fit \lmmref{sm-wigner:zero-integrals}:
\begin{eqn}
\fl	= \frac{1}{\pi^2} \Trace{ \hat{\rho}
			\int d^2\alpha \int d^2\lambda
			\exp(-\lambda \alpha^* + \lambda^* \alpha)
			\left( \frac{\partial}{\partial \lambda} \right)^s
			\left( -\frac{\partial}{\partial \lambda^*} \right)^r
			\hat{D} (\lambda, \lambda^*)
		}
\end{eqn}
Evaluating integral over $\alpha$ using \lmmref{c-numbers:fourier-of-moments}:
\begin{eqn}
	& = \int d^2\lambda\,
		\delta (\Real \lambda) \delta (\Imag \lambda)
		\left( \frac{\partial}{\partial \lambda} \right)^s
		\left( -\frac{\partial}{\partial \lambda^*} \right)^r
		\Trace{
			\hat{\rho}
			\hat{D}(\lambda, \lambda^*)
		} \\
	& = \left.
		\left( \frac{\partial}{\partial \lambda} \right)^s
		\left( -\frac{\partial}{\partial \lambda^*} \right)^r
		\chi_W (\lambda, \lambda^*)
	\right|_{\lambda=0}.
\end{eqn}
Now, recognising the final expression as a part of \lmmref{sm-wigner:moments-from-chi},
we immideately get the statement of the theorem.
\end{proof}

\begin{theorem}[Operator correspondences]
\label{thm:sm-wigner:correspondences}
\begin{eqn*}
	\mathcal{W} [ \hat{a} \hat{A} ]
		& = \left( \alpha + \frac{1}{2} \frac{\partial}{\partial \alpha^*} \right) \mathcal{W}[\hat{A}],
	\quad
	\mathcal{W} [ \hat{a}^\dagger \hat{A} ]
		= \left( \alpha^* - \frac{1}{2} \frac{\partial}{\partial \alpha} \right) \mathcal{W}[\hat{A}], \\
	\mathcal{W} [ \hat{A} \hat{a} ]
		& = \left( \alpha - \frac{1}{2} \frac{\partial}{\partial \alpha^*} \right) \mathcal{W}[\hat{A}],
	\quad
	\mathcal{W} [ \hat{A} \hat{a}^\dagger ]
		= \left( \alpha^* + \frac{1}{2} \frac{\partial}{\partial \alpha} \right) \mathcal{W}[\hat{A}].
\end{eqn*}
\end{theorem}
\begin{proof}
We will prove the first correspondence.
First, let us transform the trace using~\eqnref{sm-wigner:displacement-derivatives}:
\begin{eqn}
	\Trace{ \hat{a} \hat{A} \hat{D} }
	& = \Trace{ \hat{A} \hat{D} \hat{a}} \\
	& = \Trace{ \hat{A} \left(
		-\frac{\partial}{\partial \lambda^*}
		-\frac{1}{2} \lambda
	\right) \hat{D}} \\
	& = \left(
		-\frac{\partial}{\partial \lambda^*}
		-\frac{1}{2} \lambda
	\right) \Trace{ \hat{A} \hat{D}}
\end{eqn}
Now we need to somehow move this additional multiplier outside the integral in the expression for Wigner function:
\begin{eqn}
\fl	\mathcal{W} [ \hat{a} \hat{A} ]
	& = \frac{1}{\pi^2} \int d^2 \lambda \exp(-\lambda \alpha^* + \lambda^* \alpha)
		\Trace{ \hat{a} \hat{A} \hat{D}(\lambda, \lambda^*) } \\
\fl	& = \frac{1}{2} \frac{\partial}{\partial \alpha^*} \mathcal{W} [\hat{A}]
	- \frac{1}{\pi^2} \int d^2 \lambda \exp(-\lambda \alpha^* + \lambda^* \alpha)
		\frac{\partial}{\partial \lambda^*}
		\Trace{ \hat{A} \hat{D}(\lambda, \lambda^*) } \\
\fl	& = \frac{1}{2} \frac{\partial}{\partial \alpha^*} \mathcal{W} [\hat{A}]
	+ \frac{1}{\pi^2} \int d^2 \lambda \left(
		\frac{\partial}{\partial \lambda^*} \exp(-\lambda \alpha^* + \lambda^* \alpha)
	\right)
	\Trace{ \hat{A} \hat{D}(\lambda, \lambda^*) } \\
\fl	& = \left( \alpha + \frac{1}{2} \frac{\partial}{\partial \alpha^*} \right) \mathcal{W} [\hat{A}].
\end{eqn}
Note that we used~\lmmref{sm-wigner:zero-integrals} to move the partial derivative over $\lambda^*$.
\end{proof}

% =============================================================================
\section{Field operators and restricted basis}
% =============================================================================

Multimode fields are described by operators $\Psiop_j^{\dagger}(\xvec)$ and $\Psiop_j(\xvec)$, where $\Psiop_j^{\dagger}(\xvec)$ creates a bosonic atom of spin $j$ at location $\xvec$, and $\Psiop_j(\xvec)$ destroys one; the commutators are
\begin{eqn}
\label{eqn:func-aux:commutators}
	[ \Psiop_j(\xvec), \Psiop_k^{\dagger}(\xvec^\prime) ]
	= \delta_{jk} \delta(\xvec^\prime-\xvec).
\end{eqn}
Field operators can be decomposed using a single-particle basis:
\begin{eqn}
	\Psiop_j(\xvec) = \sum_{\nvec} \phi_{\nvec}(\xvec) \hat{a}_{j,\nvec}.
\end{eqn}
Single mode operators $\hat{a}_{j,\nvec}$ obey bosonic commutation relations, the pair $j,\nvec$ serving as a mode identifier.

Projection transformation~\eqnref{func-calculus:projector} can be extended to work on operators.
The expression remains the same, and the type becomes
\begin{eqn}
	\hat{\mathcal{P}} :: \mathbb{FH} \rightarrow \mathbb{FH}_L,
\end{eqn}
where $\mathbb{FH}_L \equiv (\mathbb{R}^D \rightarrow \mathbb{H}_L)$, and $\mathbb{H}_L$ is the Hilbert space of the restricted subset of modes.
Being applied to the annihilation operator $\Psiop_j$, this transformation returns the restricted annihilation operator
\begin{eqn}
	\hat{\mathcal{P}} [\Psiop_j]
	= \sum_{\nvec \in L} \phi_{\nvec} (\xvec) \hat{a}_{j,\nvec}
	= \Psiop_{jL} (\xvec),
\end{eqn}
containing only modes from subset $L$.
Same as with functions, we will consider all field operators to be restricted and omit the index $L$.

Because of the restricted nature of the operator, commutation relations~\eqnref{func-aux:commutators} no longer apply.
The following ones should be used instead:
\begin{eqn}
\label{eqn:func-aux:restricted-commutators}
	\left[ \Psiop_j(\xvec), \Psiop_k(\xvec^\prime) \right]
	& = \left[ \Psiop_j^\dagger(\xvec), \Psiop_k^\dagger(\xvec^\prime) \right] = 0, \\
	\left[ \Psiop_j(\xvec), \Psiop_k^\dagger(\xvec^\prime) \right]
	& = \delta_{jk} \delta_L(\xvec^\prime, \xvec).
\end{eqn}

Let us now find the expression for high-order commutators of restricted field operators, analogous to the similar one for single-mode operators~\cite{Louisell1990}.

\begin{lemma}
	Abbreviating $\Psiop \equiv \Psiop(\xvec)$ and $\Psiop^\prime \equiv \Psiop(\xvec^\prime)$:
	\begin{eqn*}
		\left[ \Psiop, ( \Psiop^{\prime\dagger} )^l \right]
		& = l \delta_P (\xvec^\prime - \xvec) ( \Psiop^{\prime\dagger} )^{l-1}, \\
		\left[ \Psiop^\dagger, ( \Psiop^\prime )^l \right]
		& = - l \delta_P^* (\xvec^\prime - \xvec) ( \Psiop^\prime )^{l-1}.
	\end{eqn*}
\end{lemma}
\begin{proof}
Proved by induction.
\end{proof}

A further generalisation of these relations is

\begin{lemma}
\label{lmm:functional-commutators}
	\begin{eqn*}
		\left[ \Psiop, f( \Psiop^\prime, \Psiop^{\prime\dagger} ) \right]
		& = \delta_P (\xvec^\prime - \xvec) \frac{\partial f}{\partial \Psiop^{\prime\dagger}} \\
		\left[ \Psiop^\dagger, f( \Psiop^\prime, \Psiop^{\prime\dagger} ) \right]
		& = -\delta_P^* (\xvec^\prime - \xvec) \frac{\partial f}{\partial \Psiop^\prime},
	\end{eqn*}
	where $f(z, z^*)$ is a function that can be expanded in the power series of $z$ and $z^*$.
\end{lemma}

% =============================================================================
\section{Functional Wigner representation}
% =============================================================================

The single-mode Wigner transformation of the operator $\hat{A}$ is defined as
\begin{eqn}
	\mathcal{W}_{\mathrm{sm}}[\hat{A}]
	= \frac{1}{\pi^2} \int d^2 \lambda \exp(-\lambda \alpha^* + \lambda^* \alpha)
		\Trace{ \hat{A} \hat{D}(\lambda, \lambda^*) },
\end{eqn}
where the displacement operator $\hat{D}(\lambda, \lambda^*) = \exp(\lambda \hat{a}^\dagger - \lambda^* \hat{a})$ was first introduced by Weyl~\cite{Weyl1950}.
The detailed description of the Wigner function $W(\alpha, \alpha^*) \equiv \mathcal{W}_{\mathrm{sm}}[\hat{\rho}]$ can be found in~\cite{Gardiner2004}.
In this section we will extend this definition to the multimode case.

The important part of the definition is the functional analogue of the displacement operator.

\begin{definition}
    Functional displacement operator
	\begin{eqn*}
		\hat{D} :: \mathbb{F}_{\restbasis} \rightarrow \mathbb{H}_{\restbasis} \\
		\hat{D}[\Lambda, \Lambda^*] = \exp \int d\xvec \left(
			\Lambda(\xvec) \Psiop^\dagger(\xvec) - \Lambda^*(\xvec) \Psiop(\xvec)
		\right).
	\end{eqn*}
	It is also convenient to define the displacement functional as
	\begin{eqn*}
		D :: \mathbb{F}_{\restbasis} \rightarrow \mathbb{F}_{\restbasis} \rightarrow \mathbb{C} \\
		D[\Lambda, \Lambda^*, \Psi, \Psi^*] = \exp \int d\xvec \left(
			-\Lambda(\xvec) \Psi^*(\xvec) + \Lambda^*(\xvec) \Psi(\xvec)
		\right).
	\end{eqn*}
\end{definition}

It can be shown that the functional displacement operator has properties similar to its single-mode equivalent.

\begin{lemma}
\label{lmm:func-wigner:displacement-derivatives}
	\begin{eqn*}
		\frac{\delta}{\delta \Lambda^\prime} \hat{D}[\Lambda, \Lambda^*]
		= \hat{D}[\Lambda, \Lambda^*] (\Psiop^{\prime\dagger} + \frac{1}{2} \Lambda^{\prime*})
		= (\Psiop^{\prime\dagger} - \frac{1}{2} \Lambda^{\prime*}) \hat{D}[\Lambda, \Lambda^*], \\
		-\frac{\delta}{\delta \Lambda^{\prime*}} \hat{D}[\Lambda, \Lambda^*]
		= \hat{D}(\Lambda, \Lambda^*) (\Psiop^\prime + \frac{1}{2} \Lambda^\prime)
		= (\Psiop^\prime - \frac{1}{2} \Lambda^\prime) \hat{D}[\Lambda, \Lambda^*].
	\end{eqn*}
\end{lemma}
\begin{proof}
Proved using Baker-Hausdorff theorem and evaluating integrals.
\end{proof}

\begin{definition}
\label{eqn:func-wigner:w-transformation}
	Functional Wigner transformation $\mathcal{W}$ is defined as
	\begin{eqn*}
		\mathcal{W} :: \mathbb{FH}_{\restbasis} \rightarrow (\mathbb{F}_{\restbasis} \rightarrow \mathbb{C}) \\
		\mathcal{W}[\hat{A}]
		= \frac{1}{\pi^{2|\restbasis|}} \int \delta^2 \Lambda
			D[\Lambda, \Lambda^*, \Psi, \Psi^*]
			\Trace{ \hat{A} \hat{D}[\Lambda, \Lambda^*] }.
	\end{eqn*}
	It transforms an operator $\hat{A}$ on a restricted Hilbert space to a functional $(\mathcal{W}[\hat{A}])[\Psi, \Psi^*]$.
	The Wigner functional is
	\begin{eqn*}
		W :: \mathbb{F}_{\restbasis} \rightarrow \mathbb{C} \\
		W [\Psi, \Psi^*]
		\equiv \mathcal{W}[\hat{\rho}]
		= \frac{1}{\pi^{2|\restbasis|}} \int \delta^2 \Lambda
			D[\Lambda, \Lambda^*, \Psi, \Psi^*]
			\chi_W [\Lambda, \Lambda^*],
	\end{eqn*}
	where $\chi_W [\Lambda, \Lambda^*]$ is the characteristic functional
	\begin{eqn*}
		\chi_W :: \mathbb{F}_{\restbasis} \rightarrow \mathbb{R} \\
		\chi_W [\Lambda, \Lambda^*]
		= \Trace{ \hat{\rho} \hat{D}[\Lambda, \Lambda^*] }.
	\end{eqn*}
\end{definition}

\begin{lemma}
\label{lmm:func-wigner:zero-integrals}
	\begin{eqn*}
		\int \delta^2\Lambda
			\frac{\delta}{\delta \Lambda^\prime} \left(
				D[\Lambda, \Lambda^*, \Psi, \Psi^*]
				\left( \frac{\delta}{\delta \Lambda^\prime} \right)^r
				\left( -\frac{\delta}{\delta \Lambda^{\prime*}} \right)^s
				\hat{D}[\Lambda, \Lambda^*]
			\right)
		= 0, \\
		\int \delta^2\Lambda
			\frac{\delta}{\delta \Lambda^{\prime*}}
			\left(
				D[\Lambda, \Lambda^*, \Psi, \Psi^*]
				\left( \frac{\delta}{\delta \Lambda^\prime} \right)^r
				\left( -\frac{\delta}{\delta \Lambda^{\prime*}} \right)^s
				\hat{D}[\Lambda, \Lambda^*]
			\right)
		= 0.
	\end{eqn*}
\end{lemma}
\begin{proof}
Displacement operator and displacement functional can be represented as functions of vectors:
\begin{eqn}
	\hat{D}[\Lambda, \Lambda^*]
	= \prod_{\nvec \in \restbasis} \exp \left(
		\lambda_{\nvec} \hat{a}_{\nvec}^\dagger - \lambda_{\nvec}^* \hat{a}_{\nvec}
	\right),
\end{eqn}
\begin{eqn}
	D[\Lambda, \Lambda^*, \Psi, \Psi^*]
	= \prod_{\nvec \in \restbasis} \exp
		(-\lambda_{\nvec} \alpha_{\nvec}^* + \lambda_{\nvec}^* \alpha_{\nvec}),
\end{eqn}
The proof consists of substituting these in the equations from the statement and applying \lmmref{c-numbers:zero-integrals}.
\end{proof}

\begin{lemma}
\label{lmm:func-wigner:moments-from-chi}
    For any non-negative integer $r$ and $s$:
	\begin{eqn*}
		\langle \symprod{ (\Psiop^\prime)^r (\Psiop^{\prime\dagger})^s } \rangle
		= \left.
			\left( \frac{\delta}{\delta \Lambda^\prime} \right)^s
			\left( -\frac{\delta}{\delta \Lambda^{\prime*}} \right)^r
			\chi_W [\Lambda, \Lambda^*]
		\right|_{\Lambda \equiv 0}.
	\end{eqn*}
\end{lemma}
\begin{proof}
The proof follows the same general scheme from the single-mode case.
The displacement operator can be expanded as
\begin{eqn}
	\exp (\Lambda \Psiop^\dagger - \Lambda^* \Psiop)
	= \sum_{r,s}
		\frac{
			\symprod{
				\left( \int d\xvec \Lambda \Psiop^\dagger \right)^r
				\left( -\int d\xvec \Lambda^* \Psiop \right)^s
			}
		}
		{r!s!}.
\end{eqn}
We can swap functional derivative with both integration and multiplication by independent function, so:
\begin{eqn}
	\frac{\delta}{\delta \Lambda^\prime} \left( \int d\xvec \Lambda \Psiop^\dagger \right)^r
	= r \Psiop^{\prime\dagger} \left( \int d\xvec \Lambda \Psiop^\dagger \right)^{r-1},
\end{eqn}
and multiple application of the differential gives us
\begin{eqn}
	\left( \frac{\delta}{\delta \Lambda^\prime} \right)^r
	\left( \int d\xvec \Lambda \Psiop^\dagger \right)^r
	= r! ( \Psiop^{\prime\dagger} )^r.
\end{eqn}
Similarly for the other differential:
\begin{eqn}
	\left( -\frac{\delta}{\delta \Lambda^{\prime*}} \right)^s
	\left( -\int d\xvec \Lambda \Psiop^\dagger \right)^s
	= s! ( \Psiop^{\prime\dagger} )^s.
\end{eqn}

Thus, same as in single-mode case,
differentiation will eliminate all lower order terms in the expansion,
and all higher order terms will be eliminated by setting $\Lambda \equiv 0$,
leaving only one operator product with required order.
\end{proof}

\begin{theorem}
\label{thm:func-wigner:correspondences}
    If $\mathcal{W} [ \hat{A} ] \equiv (\mathcal{W} [ \hat{A} ]) [\Psi, \Psi^*]$, then
	\begin{eqn*}
		\mathcal{W} [ \Psiop \hat{A} ]
			& = \left( \Psi + \frac{1}{2} \frac{\delta}{\delta \Psi^*} \right) \mathcal{W}[\hat{A}],
		\quad
		\mathcal{W} [ \Psiop^\dagger \hat{A} ]
			= \left( \Psi^* - \frac{1}{2} \frac{\delta}{\delta \Psi} \right) \mathcal{W}[\hat{A}], \\
		\mathcal{W} [ \hat{A} \Psiop ]
			& = \left( \Psi - \frac{1}{2} \frac{\delta}{\delta \Psi^*} \right) \mathcal{W}[\hat{A}],
		\quad
		\mathcal{W} [ \hat{A} \Psiop^\dagger ]
			= \left( \Psi^* + \frac{1}{2} \frac{\delta}{\partial \Psi} \right) \mathcal{W}[\hat{A}].
	\end{eqn*}
\end{theorem}
\begin{proof}
The proof follows the same scheme as in the single-mode case, with \lmmref{func-wigner:displacement-derivatives} used to transform the $\hat{A}\hat{D}$ product inside the trace, and \lmmref{func-wigner:zero-integrals} to move the differentials.
\end{proof}

\begin{theorem}
\label{thm:func-wigner:moments}
    For any non-negative integer $r$ and $s$:
	\begin{eqn*}
		\langle \symprod{ \Psiop^r (\Psiop^\dagger)^s } \rangle
		= \int \delta^2\Psi\, \Psi^r (\Psi^*)^s W[\Psi, \Psi^*].
	\end{eqn*}
\end{theorem}
\begin{proof}
By definition of Wigner functional:
\begin{eqn}
	\int \delta^2\Psi\, \Psi^r (\Psi^*)^s W[\Psi, \Psi^*] \\
	= \frac{1}{\pi^{2|\restbasis|}} \Trace{ \hat{\rho}
		\int \delta^2\Psi\, \Psi^r (\Psi^*)^s
		\int \delta^2\Lambda D[\Lambda, \Lambda^*, \Psi, \Psi^*]
		\hat{D}[\Lambda, \Lambda^*]
	}
\end{eqn}
Integrating by parts and eliminating terms which fit \lmmref{func-wigner:zero-integrals}:
\begin{eqn}
\fl	= \frac{1}{\pi^{2|\restbasis|}} \Trace{ \hat{\rho}
		\int \delta^2\Psi \int \delta^2\Lambda
		D[\Lambda, \Lambda^*, \Psi, \Psi^*]
		\left( \frac{\delta}{\delta \Lambda} \right)^s
		\left( -\frac{\delta}{\delta \Lambda^*} \right)^r
		\hat{D}[\Lambda, \Lambda^*]
	}
\end{eqn}
Evaluating the integral over $\Psi$ using \lmmref{func-calculus:fourier-of-moments}:
\begin{eqn*}
	& = \int \delta^2\Lambda\,
		\Delta_{\restbasis}[\Lambda]
		\left( \frac{\delta}{\delta \Lambda} \right)^s
		\left( -\frac{\delta}{\delta \Lambda^*} \right)^r
		\Trace{
			\hat{\rho}
			\hat{D}[\Lambda, \Lambda^*]
		} \\
	& = \left.
		\left( \frac{\delta}{\delta \Lambda} \right)^s
		\left( -\frac{\delta}{\delta \Lambda^*} \right)^r
		\chi_W [\Lambda, \Lambda^*]
	\right|_{\Lambda \equiv 0}.
\end{eqn*}
Now, recognising the final expression as a part of \lmmref{func-wigner:moments-from-chi},
we immediately get the statement of the theorem.
\end{proof}

\thmref{func-wigner:moments} and \thmref{func-wigner:correspondences} can be further extended to the case of several components.
$C$ components with $|\restbasis|$ modes each are equivalent to a single set of $C|\restbasis|$ modes, which means that \thmref{func-wigner:moments} and \thmref{func-wigner:correspondences} can be applied.
For convenience, these theorems can be re-formulated in multi-component terms.

\begin{theorem}[Multi-component extension of \thmref{func-wigner:correspondences}]
\label{thm:func-wigner:mc-correspondences}
    If $\mathcal{W} [ \hat{A} ] \equiv (\mathcal{W} [ \hat{A} ]) [\Psivec, \Psivec^*]$, then
    \begin{eqn*}
    	\mathcal{W} [ \Psiop_c \hat{A} ]
    		& = \left( \Psi_c + \frac{1}{2} \frac{\delta}{\delta \Psi_c^*} \right) \mathcal{W}[\hat{A}],
    	\quad
    	\mathcal{W} [ \Psiop_c^\dagger \hat{A} ]
    		= \left( \Psi_c^* - \frac{1}{2} \frac{\delta}{\delta \Psi_c} \right) \mathcal{W}[\hat{A}], \\
    	\mathcal{W} [ \hat{A} \Psiop_c ]
    		& = \left( \Psi_c - \frac{1}{2} \frac{\delta}{\delta \Psi_c^*} \right) \mathcal{W}[\hat{A}],
    	\quad
    	\mathcal{W} [ \hat{A} \Psiop_c^\dagger ]
    		= \left( \Psi_c^* + \frac{1}{2} \frac{\delta}{\partial \Psi_c} \right) \mathcal{W}[\hat{A}].
    \end{eqn*}
\end{theorem}

\begin{theorem}[Multi-component extension of \thmref{func-wigner:moments}]
\label{thm:func-wigner:mc-moments}
	For any non-negative integer $r_c$ and $s_c$, $c \in [1, C]$
	\begin{eqn*}
	    \langle \symprod{ \prod_{c=1}^C \Psiop_c^{r_c} (\Psiop_c^\dagger)^{s_c} } \rangle
		= \left( \prod_{c=1}^C \int \delta^2\Psi_c \right)
		\left( \prod_{c=1}^C \Psi_c^{r_c} (\Psi_c^*)^{s_c} \right) W[\Psivec, \Psivec^*].
	\end{eqn*}
\end{theorem}

% =============================================================================
\section{Specific cases of transformations}
% =============================================================================

This section contains some theorems concerning transformations of specific operator sequences, which will be useful when transforming the master equation.

\begin{theorem}
\label{thm:transformations:w-commutator1}
    \begin{eqn*}
    	\mathcal{W} \left[ [\int d\xvec \Psiop_j^\dagger \Psiop_k, \hat{A}] \right]
    	= \int d\xvec \left(
    		- \frac{\delta}{\delta \Psi_j} \Psi_k
    		+ \frac{\delta}{\delta \Psi_k^*} \Psi_j^*
    	\right) \mathcal{W}[\hat{A}].
    \end{eqn*}
\end{theorem}
\begin{proof}
Proved straightforwardly using \thmref{func-wigner:mc-correspondences} and the relation
\begin{eqn}
	\Psi_k \frac{\delta}{\delta \Psi_j} \mathcal{F}
	= \left(
		\frac{\delta}{\delta \Psi_j} \Psi_k
		- \delta_{jk} \delta_{\restbasis}(\xvec, \xvec)
	\right) \mathcal{F}.
\end{eqn}
\end{proof}

Commutators with the Laplacian inside require somewhat special treatment, because it acts on basis functions and, in general, cannot be dragged around like a constant.
For our purposes we only need one specific case, and, fortunately, in this case it does act like a constant.

\begin{theorem}
\label{thm:transformations:w-laplacian-commutator1}
    \begin{eqn*}
    	\mathcal{W} \left[
    		\int d\xvec [\Psiop^\dagger(\xvec) \nabla^2 \Psiop(\xvec), \hat{A}]
    	\right]
    	= \int d\xvec \left(
    		- \frac{\delta}{\delta \Psi} \nabla^2 \Psi
    		+ \frac{\delta}{\delta \Psi^*} \nabla^2 \Psi^*
    	\right) \mathcal{W}[\hat{A}].
    \end{eqn*}
\end{theorem}
\begin{proof}
Proved using \thmref{func-wigner:mc-correspondences} and \lmmref{func-calculus:move-laplacian}.
\end{proof}

\begin{theorem}
\label{thm:transformations:w-commutator2}
    \begin{eqn*}
    	\mathcal{W} \left[
    		[
    			\int d\xvec \int d\xvec^\prime
    			\Psiop_j^\dagger \Psiop_k^{\prime\dagger} \Psiop_j^\prime \Psiop_k,
    			\hat{A}
    		]
    	\right] \\
    	= \int d\xvec \int d\xvec^\prime \left(
    		\frac{\delta}{\delta \Psi_j} \left(
    			- \Psi_j^\prime \Psi_k \Psi_k^{\prime*}
    			+ \frac{1}{2} \delta_{jk} \delta_{\restbasis}(\xvec^\prime, \xvec^\prime) \Psi_k
    			+ \frac{1}{2} \delta_{\restbasis}(\xvec, \xvec^\prime) \Psi_j^\prime
    		\right) \right . \\
    	\left. + \frac{\delta}{\delta \Psi_j^{\prime*}} \left(
    			\Psi_j^* \Psi_k \Psi_k^{\prime*}
    			- \frac{1}{2} \delta_{jk} \delta_{\restbasis}(\xvec, \xvec) \Psi_k^{\prime*}
    			- \frac{1}{2} \delta_{\restbasis}(\xvec, \xvec^\prime) \Psi_j^*
    		\right) \right. \\
    	\left. + \frac{\delta}{\delta \Psi_k^\prime} \left(
    			- \Psi_j^\prime \Psi_j^* \Psi_k
    			+ \frac{1}{2} \delta_{jk} \delta_{\restbasis}(\xvec, \xvec) \Psi_j^\prime
    			+ \frac{1}{2} \delta_{\restbasis}(\xvec^\prime, \xvec) \Psi_k
    		\right) \right .\\
    	\left. + \frac{\delta}{\delta \Psi_k^*} \left(
    			\Psi_j^\prime \Psi_j^* \Psi_k^{\prime*}
    			- \frac{1}{2} \delta_{jk} \delta_{\restbasis}(\xvec^\prime, \xvec^\prime) \Psi_j^*
    			- \frac{1}{2} \delta_{\restbasis}(\xvec^\prime, \xvec) \Psi_k^{\prime*}
    		\right) \right. \\
    	\left.
    			+ \frac{\delta}{\delta \Psi_j}
    			\frac{\delta}{\delta \Psi_j^{\prime*}}
    			\frac{\delta}{\delta \Psi_k^\prime}
    			\frac{1}{4} \Psi_k
    			- \frac{\delta}{\delta \Psi_j}
    			\frac{\delta}{\delta \Psi_j^{\prime*}}
    			\frac{\delta}{\delta \Psi_k^*}
    			\frac{1}{4} \Psi_k^{\prime*}
    		\right. \\
    	\left.
    			+ \frac{\delta}{\delta \Psi_k^\prime}
    			\frac{\delta}{\delta \Psi_k^*}
    			\frac{\delta}{\delta \Psi_j}
    			\frac{1}{4} \Psi_j^\prime
    			- \frac{\delta}{\delta \Psi_k^\prime}
    			\frac{\delta}{\delta \Psi_k^*}
    			\frac{\delta}{\delta \Psi_j^{\prime*}}
    			\frac{1}{4} \Psi_j^*
    	\right) \mathcal{W}[\hat{A}].
    \end{eqn*}
\end{theorem}
\begin{proof}
Proof is the same as in case of \thmref{transformations:w-commutator1}.
\end{proof}

\begin{lemma}
\label{lmm:transformations:swap-differential}
    For $\mathcal{F} \in \mathbb{F}_{\restbasis} \rightarrow \mathbb{F}$ and any non-negative integer $a$, $b$:
    \begin{eqn*}
    	\Psi(\xvec)^a \left( \frac{\delta}{\delta \Psi(\xvec)} \right)^b \mathcal{F}[\Psi, \Psi^*] \\
    	= \sum_{j=0}^{\min(a, b)}
    		\binom{b}{j} \frac{(-1)^j a!}{(a - j)!}
    		\delta_{\restbasis}(\xvec, \xvec)^j
    		\left( \frac{\delta}{\delta \Psi(\xvec)} \right)^{b - j}
    		\Psi(\xvec)^{a - j}
    		\mathcal{F}[\Psi, \Psi^*]
    \end{eqn*}
\end{lemma}
\begin{proof}
Proved straightforwardly by induction.
\end{proof}

\begin{lemma}[Sum rearrangement]
\label{lmm:transformations:sum-rearrangement}
    For any non-negative integer $l$, $u$:
    \begin{eqn*}
    	\sum_{j=0}^l \sum_{k=0}^{\min(l-u,j)} x^{j-k} Q(j, k)
    	= \sum_{v=0}^l x^v \sum_{k=0}^{l-\max(u,v)} Q(v + k, k).
    \end{eqn*}
\end{lemma}
\begin{proof}
Can be proved either by formal manipulation with sets, or by drawing a picture.
\end{proof}

\begin{theorem}
\label{thm:transformations:w-losses}
    If loss operator $\hat{\mathcal{L}}_{\lvec}$ is defined as
    \begin{eqn*}
    	\hat{\mathcal{L}}_{\lvec} [\hat{A}]
    	= 2 \hat{O}_{\lvec} \hat{A} \hat{O}_{\lvec}^\dagger
    		- \hat{O}_{\lvec}^\dagger \hat{O}_{\lvec} \hat{A}
    		- \hat{A} \hat{O}_{\lvec}^\dagger \hat{O}_{\lvec},
    \end{eqn*}
    where
    \begin{eqn*}
    	\hat{O}_{\lvec}
    	\equiv \hat{O}_{\lvec} (\Psiopvec)
    	= \prod_{c=1}^C \Psiop_c^{l_c} (\xvec),
    \end{eqn*}
    then its Wigner transformation is
    \begin{eqn*}
\fl    	\mathcal{W} \left[ \int d\xvec \hat{\mathcal{L}}_{\lvec} [\hat{A}] \right]
    	= \int d\xvec
    		\sum_{j_1=0}^{l_1} \sum_{k_1=0}^{l_1} \ldots
    		\sum_{j_C=0}^{l_C} \sum_{k_C=0}^{l_C}
    			\left(
    				\prod_{c=1}^C
    					\left( \frac{\delta}{\delta \Psi_c^*} \right)^{j_c}
    					\left( \frac{\delta}{\delta \Psi_c} \right)^{k_c}
    			\right)
    			L_{\jvec, \kvec}
    		\mathcal{W}[\hat{A}],
    \end{eqn*}
    where
    \begin{eqn*}
\fl    	L_{\jvec, \kvec}
    	= \left( 2 - (-1)^{\sum_c j_c} - (-1)^{\sum_c k_c} \right) \\
    		\prod_{c=1}^C \left(
    			\sum_{m_c=0}^{l_c - \max(j_c, k_c)}
    			Q_c(j_c, k_c, m_c)
    			\delta_{\restbasis}(\xvec, \xvec)^{m_c}
    			\Psi_c^{l_c - j_c - m_c}
    			(\Psi_c^*)^{l_c - k_c - m_c}
    		\right),
    \end{eqn*}
    and
    \begin{eqn*}
        Q_c(j, k, m)
    	= (-1)^m \left( \frac{1}{2} \right)^{j + k + m}
    		\frac{(l_c!)^2}{m! j! k! (l_c - k - m)! (l_c - j - m)!}.
    \end{eqn*}
\end{theorem}
\begin{proof}
Proved by applying \thmref{func-wigner:mc-correspondences}, expanding products using binomial theorem, using \lmmref{transformations:swap-differential} to move differentials to front, and applying \lmmref{transformations:sum-rearrangement} to transform summations.
\end{proof}

% =============================================================================
\section{Wigner truncation and Fokker-Planck equation}
% =============================================================================

Now we have all necessary tools to transform the master equation~\eqnref{master-eqn:master-eqn} with the Wigner transformation from \defref{func-wigner:w-transformation} to the partial differential equation.

Namely, the single-particle term~\eqnref{master-eqn:single-particle} is transformed using \thmref{transformations:w-commutator1} and \thmref{transformations:w-laplacian-commutator1} (since $K_j$ is basically a sum of Laplacian operator and functions of $\xvec$):
\begin{eqn}
	\mathcal{W} \left[ [ \int d\xvec \Psiop_j^\dagger K_{jk} \Psiop_k, \hat{\rho} ] \right]
	= \int d\xvec \left(
			- \frac{\delta}{\delta \Psi_j} K_{jk} \Psi_k
			+ \frac{\delta}{\delta \Psi_k^*} K_{jk} \Psi_j^*
		\right)
		W,
\end{eqn}
where Wigner function $W = \mathcal{W}[\hat{\rho}]$.
Nonlinear term is transformed with \thmref{transformations:w-commutator2} (minding the locality of interaction and assuming $U_{kj} = U_{jk}$):
\begin{eqn}
\fl	\mathcal{W} \left[
		[
			\int d\xvec \frac{U_{jk}}{2}
				\Psiop_j^\dagger \Psiop_k^\dagger \Psiop_j \Psiop_k,
			\hat{\rho}
		]
	\right]
	= & \int d\xvec U_{jk} \left(
		\frac{\delta}{\delta \Psi_j} \left(
			- \Psi_j \Psi_k \Psi_k^*
			+ \frac{\delta_{\restbasis}(\xvec, \xvec)}{2} ( \delta_{jk} \Psi_k + \Psi_j )
		\right) \right. \\
	&	\left. + \frac{\delta}{\delta \Psi_j^*} \left(
			\Psi_j^* \Psi_k \Psi_k^*
			- \frac{\delta_{\restbasis}(\xvec, \xvec)}{2} ( \delta_{jk} \Psi_k^* + \Psi_j^* )
		\right) \right. \\
	&	\left.
			+ \frac{\delta}{\delta \Psi_j}
			\frac{\delta}{\delta \Psi_j^*}
			\frac{\delta}{\delta \Psi_k}
			\frac{1}{4} \Psi_k
			- \frac{\delta}{\delta \Psi_j}
			\frac{\delta}{\delta \Psi_j^*}
			\frac{\delta}{\delta \Psi_k^*}
			\frac{1}{4} \Psi_k^*
		\right) W.
\end{eqn}

Loss terms~\eqnref{master-eqn:loss-term} are transformed with \thmref{transformations:w-losses}.
\todo{Not writing the resulting expression here, because with the absence of truncation it is too long, and is practically the same as in theorem statement.}

Assuming that $K_{jk}$, $U_{jk}$ and $\kappa_{\lvec}$ are real-valued, all the transformations described above result in a partial differential equation for $W$ of the form
\begin{eqn}
\fl	\frac{\partial W}{\partial t} = \int d^D\xvec \left\{
    	- \sum_{j=1}^C \frac{\delta}{\delta \Psi_j} A_j
    	- \sum_{j=1}^C \frac{\delta}{\delta \Psi_j^*} A_j^*
    	+ \sum_{j=1}^C \sum_{k=1}^C \frac{\delta^2}{\delta \Psi_j^* \delta \Psi_k} D_{jk}
		+ \mbox{O} \left[ \frac{\delta^3}{\delta\Psi_j^3} \right]
	\right\} W.
\end{eqn}
The terms of order higher than 2 are produced both by the nonlinear term in the Hamiltonian and loss terms.
Such an equation could be solved without additional approximations if there were only orders up to 3 (which means the absence of losses)~\cite{Polkovnikov2003}, but in most cases all terms except for first- and second-order ones are truncated.
In the assumption of the state being coherent, the condition for truncation can be shown to be~\cite{Sinatra2002}
\begin{eqn}
    N \gg |\restbasis|,
\end{eqn}
where $N$ is the number of atoms.
This condition is equivalent to~\cite{Norrie2006}
\begin{eqn}
    \delta_{\restbasis_j}(\xvec, \xvec) \ll | \Psi_j |^2.
\end{eqn}

Wigner truncation allows us to simplify the results of \thmref{transformations:w-commutator2} and \thmref{transformations:w-losses}.

\begin{lemma}
    Assuming the conditions for Wigner truncation are satisfied,
    the result of Wigner transformation of the nonlinear term can be written as
    \begin{eqn*}
    	\mathcal{W} \left[
    		[
    			\frac{U_{jk}}{2}
    				\Psiop_j^\dagger \Psiop_k^\dagger \Psiop_j \Psiop_k,
    			\hat{\rho}
    		]
    	\right]
    	= U_{jk} \left(
    		\frac{\delta}{\delta \Psi_j^*} \Psi_j^* \Psi_k \Psi_k^*
    		- \frac{\delta}{\delta \Psi_j} \Psi_j \Psi_k \Psi_k^*
    	\right) W.
    \end{eqn*}
\end{lemma}

\begin{lemma}
    Assuming the conditions for Wigner truncation are satisfied, the result of Wigner transformation of the loss term can be written as
    \begin{eqn*}
\fl    	\mathcal{W}[\mathcal{L}_{\lvec}[\hat{\rho}]]
    	= \sum_{n=1}^C
    			\frac{\delta}{\delta \Psi_n^*} \frac{\partial O_{\lvec}}{\partial \Psi_n} O_{\lvec}^*
    	+ \sum_{n=1}^C
    		\frac{\delta}{\delta \Psi_n} \frac{\partial O_{\lvec}^*}{\partial \Psi_n^*} O_{\lvec}
    	+ \sum_{n=1}^C \sum_{p=1}^C
    		\frac{\delta^2}{\delta \Psi_n^* \delta \Psi_p}
    		\frac{\partial O_{\lvec}}{\partial \Psi_n}
    		\frac{\partial O_{\lvec}^*}{\partial \Psi_p^*},
    \end{eqn*}
    where $O_{\lvec} \equiv O_{\lvec}[\Psivec] = \prod_{c=1}^C \Psi_c^{l_c}$.
\end{lemma}
\begin{proof}
The proof is basically a simplification of the result of \thmref{transformations:w-losses} under certain conditions.
First, we are neglecting all occurrences of $\delta_{\restbasis}$, which means setting $m_c = 0$ for every $c$.
Second, we are dropping all terms with high order differentials,
which can be expressed as limiting $\sum j_c + \sum k_c \le 2$.
The only combinations of $j_c$ and $k_c$ for which $Z(\jvec, \kvec)$ is not zero are thus
$\{ j_c = \delta_{cn}, k_c = 0, n \in [1, C] \}$,
$\{ j_c = 0, k_c = \delta_{cn}, n \in [1, C] \}$ and
$\{ j_c = \delta_{cn}, k_c = \delta_{cp}, n \in [1, C], p \in [1, C] \}$.
These combinations produce terms with $\delta/\delta \Psi_n^*$,
$\delta/\delta \Psi_n$ and
$\delta^2/\delta \Psi_p \delta \Psi_n^*$ respectively.
Applying these conditions one can get the statement of the theorem.
\end{proof}

Thus the truncated Fokker-Planck equation is
\begin{eqn}
\fl	\frac{dW}{dt}
	= \int d\xvec \left(
		- \sum_{j=1}^C \frac{\delta}{\delta \Psi_j} \mathcal{A}^{(j)}
		- \sum_{j=1}^C \frac{\delta}{\delta \Psi_j^*} (\mathcal{A}^{(j)})^*
		+ \sum_{j=1}^C \sum_{k=1}^C \frac{\delta^2}{\delta \Psi_j^* \delta \Psi_k} D_{jk}
	\right) W,
\end{eqn}
where
\begin{eqn}
	\mathcal{A}^{(j)} = -\frac{i}{\hbar} \left(
			\sum_{k=1}^C K_{jk} \Psi_k
			+ \sum_{k=1}^C U_{jk} \Psi_j \Psi_k \Psi_k^*
		\right)
		- \sum_{\lvec} \kappa_{\lvec} \frac{\partial O_{\lvec}^*}{\partial \Psi_j^*} O_{\lvec},
\end{eqn}
and
\begin{eqn}
	D_{jk} = \sum_{\lvec} \kappa_{\lvec}
		\frac{\partial O_{\lvec}}{\partial \Psi_j}
		\frac{\partial O_{\lvec}^*}{\partial \Psi_k^*}.
\end{eqn}

Since the diffusion matrix is positive-definite, the truncated Wigner function $W$ is a probability distribution
Therefore the equation can be further transformed to the equivalent set of stochastic differential equations in It\^{o} form as described by \thmref{app-fpe:fpe-sde-func}:
\begin{eqn}
\label{eqn:fpe:sdes}
	d\Psi_j = \mathcal{P}_{\restbasis_j} \left[
		\mathcal{A}^{(j)} dt + \sum_{\lvec} \mathcal{B}_{\lvec}^{(j)} Q_{\lvec}
	\right],
\end{eqn}
where
\begin{eqn}
    \mathcal{B}_{\lvec}^{(j)} = \sqrt{\kappa_{\lvec}} \frac{\partial O_{\lvec}^*}{\partial \Psi_j^*},
\end{eqn}
and $Q_{\lvec}$ is a functional Wiener process:
\begin{eqn}
	Q_{\lvec} = \sum_{\nvec \in \fullbasis} \phi_j Z_{\lvec,\nvec},
\end{eqn}
and $Z_{\lvec,\nvec}$ are, in turn, independent complex-valued Wiener processes.
Alternatively, in Stratonovich form the SDEs look like
\begin{eqn}
	d\Psi_j = \mathcal{P}_{\restbasis_j} \left[
		(\mathcal{A}^{(j)} - \mathcal{S}^{(j)}) dt + \sum_{\lvec} B_{\lvec}^{(j)} Q_{\lvec}
	\right],
\end{eqn}
where the Stratonovich term is
\begin{eqn}
	\mathcal{S}^{(j)}
	= \sum_{n=1}^C \sum_{\lvec} \kappa_{\lvec}
		\frac{\partial O_{\lvec}}{\partial \Psi_n}
		\left(\frac{\partial^2 O_{\lvec}}{\partial \Psi_n \partial \Psi_j} \right)^*
		\delta_{\restbasis_n} (\xvec, \xvec).
\end{eqn}

These equations can now be solved using conventional methods, and any required expectations symmetrically ordered operator products can be obtained from their solution using \thmref{func-wigner:moments}:
\begin{eqn}
    \langle \symprod{
        \prod_{c=1}^C \Psiop_c^{r_c} (\Psiop_c^\dagger)^{s_c}
    } \rangle
    & = \int \delta^2\Psi_1 \ldots \int \delta^2\Psi_C\,
		    \prod_{c=1}^C \Psi_c^{r_c} (\Psi_c^*)^{s_c} W \\
    & \approx \pathavg{
        \prod_{c=1}^C \Psi_c^{r_c} (\Psi_c^*)^{s_c}
    },
\end{eqn}
where $r_c$ and $s_c$ is some set of non-negative integers, and $\pathavg{}$ stands for the average over the simulation paths.

% =============================================================================
\section{Initial states}
% =============================================================================

Initial values for the numerical integration of equations~\eqnref{fpe:sdes} are obtained by finding the Wigner transformation of the density matrix for the desired initial state, and then sampling the initial values according to the resulting Wigner function.
As an example, consider the simple case with a coherent initial state.

\begin{theorem}
    The Wigner distribution for a multi-mode coherent state with expectation values
    $\alpha_{\nvec}(0) = \alpha_{\nvec}^{(0)}$, $\nvec \in \restbasis$ is
    \begin{eqn*}
    	W_c (\balpha^{(0)})
    	= \left( \frac{2}{\pi} \right)^{|\restbasis|} \prod_{\nvec \in \restbasis}
    		\exp(-2 |\alpha_{\nvec} - \alpha_{\nvec}^{(0)}|^2).
    \end{eqn*}
\end{theorem}
\begin{proof}
The density matrix of the state is
\begin{eqn}
	\hat{\rho}
	= \vert \alpha_{\nvec}^{(0)},\, \nvec \in \restbasis \rangle
		\langle \alpha_{\nvec}^{(0)},\, \nvec \in \restbasis \vert
	= \left( \prod_{\nvec \in \restbasis} \vert \alpha_{\nvec}^{(0)} \rangle \right)
		\left( \prod_{\nvec \in \restbasis} \langle \alpha_{\nvec}^{(0)} \vert \right).
\end{eqn}
Then the characteristic function is
\begin{eqn}
	\chi_W (\balpha^{(0)})
	= \prod_{\nvec \in \restbasis}
		\langle \alpha_{\nvec}^{(0)} \vert
		\hat{D}_{\nvec} (\lambda_{\nvec}, \lambda_{\nvec}^*)
		\vert \alpha_{\nvec}^{(0)} \rangle
\end{eqn}
Using the properties of the displacement operator, this can be transformed to
\begin{eqn}
	\chi_W (\balpha^{(0)})
	= \prod_{\nvec \in \restbasis}
		\exp(
			- \lambda_{\nvec}^* \alpha_{\nvec}^{(0)}
			+ \lambda_{\nvec} (\alpha_{\nvec}^{(0)})^*
			- \frac{1}{2} |\lambda|^2
		).
\end{eqn}
Finally, Wigner function is
\begin{eqn}
\fl	W_c (\balpha^{(0)})
	= \frac{1}{\pi^{2|\restbasis|}} \prod_{\nvec \in \restbasis} \left(
		\int d^2\lambda_{\nvec}
			\exp(
				- \lambda_{\nvec} (\alpha_{\nvec}^* - (\alpha_{\nvec}^{(0)})^*)
				+ \lambda_{\nvec}^* (\alpha_{\nvec} - \alpha_{\nvec}^{(0)})
				- \frac{1}{2} |\lambda|^2
			)
	\right) \\
	= \left( \frac{2}{\pi} \right)^{|\restbasis|} \prod_{\nvec \in \restbasis}
		\exp(-2 |\alpha_{\nvec} - \alpha_{\nvec}^{(0)}|^2).
	\qedhere
\end{eqn}
\end{proof}

The resulting Wigner distribution is a product of independent complex-valued Gaussian distributions for each mode,
with the expectation value equal to the expectation value of the mode,
and the variance equal to $\frac{1}{2}$.
Therefore the initial state can be sampled as
\begin{eqn}
	\alpha_{\nvec} = \alpha_{\nvec}^{(0)} + \frac{1}{\sqrt{2}} \eta_{\nvec},
\end{eqn}
where $\eta_{\nvec}$ are normally distributed complex random numbers with zero mean,
$\langle \eta_{\mvec} \eta_{\nvec} \rangle = 0$ and
$\langle \eta_{\mvec} \eta_{\nvec}^* \rangle = \delta_{\mvec,\nvec}$
(in other words, with components distributed independently with variance $\frac{1}{2}$).
This looks like adding half a ``vacuum particle'' to each mode.
In functional form this can be written as
\begin{eqn}
	\Psi_j(\xvec, 0)
	= \Psi_j^{(0)}(\xvec, 0)
		+ \sum_{\nvec \in \restbasis} \frac{\eta_{j,\nvec}}{2} \phi_{\nvec},
\end{eqn}
where $\Psi_j^{(0)}(\xvec, 0)$ is the ``classical'' ground state of the system.

\todo{We can describe the Wigner function for the number state here, as an example of non-positive Wigner function.}

% =============================================================================
\section{Conclusion}
% =============================================================================

Conclusion goes here.

\appendix

% =============================================================================
\section{Wirtinger differentiation}
% =============================================================================

In this paper we are using differentiation of complex functions extensively.
Instead of classical definition of the differential which only works for holomorphic functions we use Wirtinger differentiation~\cite{Wirtinger1927}.
One can find thorough description of these rules, for example, in~\cite{Kreutz-Delgado2009}; in this section we will only outline the basics.

\begin{definition}
	For a complex variable $z = x + iy$ and a function $f(z) = u(x, y) + iv(x, y)$ the Wirtinger differential is
	\begin{equation*}
		\frac{df(z)}{dz}
		= \frac{1}{2} \left(
			\frac{\partial f}{\partial x} - i \frac{\partial f}{\partial y}
		\right).
	\end{equation*}
\end{definition}

One can easily prove that if $f(z)$ is holomorphic, then the above definition coincides with the classical differential for complex functions.
Wirtinger differential obeys sum, product, quotient, and chain differentiation rules (the former one is applied as if $f(z) \equiv f(z, z^*)$).

We will need some lemmas about integration.
For convenience, we will use the following definition:

\begin{definition}
	For a complex variable $z = x + iy$ the integral
	\begin{equation*}
		\int d^2 z \equiv \int_{-\infty}^{\infty} \int_{-\infty}^{\infty} dx\, dy,
	\end{equation*}
	or, in other words, stands for the two-dimensional integral over the complex plane.
\end{definition}

\begin{lemma}
\label{lmm:c-numbers:fourier-of-moments}
	If $\alpha$ and $\lambda$ are complex variables,
	then for any non-negative integers $r$ and $s$:
	\begin{equation}
	\begin{split}
		& \int d^2\alpha\, \alpha^r (\alpha^*)^s \exp(-\lambda \alpha^* + \lambda^* \alpha) \\
		& = \pi^2
			\left( -\frac{\partial}{\partial \lambda^*} \right)^r
			\left( \frac{\partial}{\partial \lambda} \right)^s
			\delta(\Real \lambda) \delta(\Imag \lambda)
	\end{split}
	\end{equation}
\end{lemma}
\begin{proof}
First, changing the variables in the integrals and using known Fourier transform relations, we can prove that for real $x$ and $v$, and non-negative integer $n$
\begin{equation*}
	\int\limits_{-\infty}^{\infty} dv\, v^n \exp(\pm 2 i x v)
	= \pi (\mp i / 2)^n \delta^{(n)}(x),
\end{equation*}
Expanding the $\alpha^r (\alpha^*)^s$ term using binomial theorem and using the above property, one can reach the statement of the lemma.
\end{proof}

A notable special case of \lmmref{c-numbers:fourier-of-moments} is
\begin{equation*}
	\int d^2\alpha \exp(-\lambda \alpha^* + \lambda^* \alpha)
	= \pi^2 \delta(\Real \lambda) \delta(\Imag \lambda).
\end{equation*}

\begin{lemma}
\label{lmm:c-numbers:zero-integrals}
	For any non-negative integers $r$, $s$ and complex $\alpha$:
	\begin{equation*}
	\begin{split}
		\int d^2\lambda
			\frac{\partial}{\partial \lambda} \left(
				\exp(-\lambda \alpha^* + \lambda^* \alpha)
				\exp(ixy) x^r y^s
			\right)
		& = 0, \\
		\int d^2\lambda
			\frac{\partial}{\partial \lambda^*}
			\left(
				\exp(-\lambda \alpha^* + \lambda^* \alpha)
				\exp(ixy) x^r y^s
			\right)
		& = 0,
	\end{split}
	\end{equation*}
	where $\lambda = x + iy$.
\end{lemma}
\begin{proof}
We will prove the first equation.
First, note that complex-valued integral of derivative is evaluated as
\begin{equation*}
\begin{split}
	\int d^2\lambda \frac{\partial}{\partial \lambda} f(\lambda, \lambda^*)
	& =	\frac{1}{2} \int\limits_{-\infty}^{\infty} dy \left(
			\left. g(x, y) \right|_{x=-\infty}^{\infty}
		\right) \\
	& - \frac{i}{2} \int\limits_{-\infty}^{\infty} dx \left(
			\left. h(x, y) \right|_{y=-\infty}^{\infty}
		\right),
\end{split}
\end{equation*}
where we expanded $f = g + ih$.
Thus
\begin{equation*}
\begin{split}
	& \int d^2\lambda
		\frac{\partial}{\partial \lambda} \left(
			\exp(-\lambda \alpha^* + \lambda^* \alpha)
			\exp(ixy) x^r y^s
		\right) \\
	& = \left(
			\frac{1}{2} \exp(2ixv) x^r \int dy \exp(iy(x-2u)) y^s
		\right)_{x = -\infty}^\infty \\
	& - \left(
			\frac{i}{2} \exp(-2ixy) y^s \int dx \exp(ix(y+2v)) x^r
		\right)_{y = -\infty}^\infty \\
	& = \left(
			\frac{1}{2} \exp(2ixv) x^r 2 \pi i^s \delta^{(s)}(x-2u)
		\right)_{x = -\infty}^\infty \\
	& - \left(
			\frac{i}{2} \exp(-2ixy) y^s 2 \pi i^r \delta^{(r)}(y+2v)
		\right)_{y = -\infty}^\infty \\
	& = 0,
\end{split}
\end{equation*}
because any derivative of delta function is zero on the infinity.
\end{proof}

% =============================================================================
\section{Functional calculus}
% =============================================================================

This section outlines the functional calculus, which is heavily used throughout the paper.
Detailed description is given in~\cite{Dalton2011}, and here we only provide some important definitions and results which are used later in the paper.
In this section we will use the definitions from the \secref{func-operators}, namely the full basis $\fullbasis$ and the restricted basis $\restbasis$.
Given the basis, we can define the correspondence between some function of coordinates and its representation in mode space.

\begin{definition}
	Let $\mathbb{F}$ be a space of all functions of coordinates, which consists only of modes from $\restbasis$: $\mathbb{F}_{\restbasis} \equiv (\mathbb{R}^D \rightarrow \mathbb{C})_{\restbasis}$ (restricted functions).
	Composition transformation creates a function from a vector of mode populations:
	\begin{eqn*}
		\mathcal{C}_{\restbasis} :: \mathbb{C}^{|\restbasis|} \rightarrow \mathbb{F}_{\restbasis} \\
		\mathcal{C}_{\restbasis}(\balpha) = \sum_{\nvec \in \restbasis} \phi_{\nvec} \alpha_{\nvec}.
	\end{eqn*}
	Decomposition transformation, correspondingly, creates a vector of populations out of a function:
	\begin{eqn*}
		\mathcal{C}_{\restbasis}^{-1} :: \mathbb{F} \rightarrow \mathbb{C}^{|\restbasis|} \\
		(\mathcal{C}_{\restbasis}^{-1}[f])_{\nvec}
		= \int d\xvec \phi_{\nvec}^* f,\,{\nvec} \in \restbasis.
	\end{eqn*}
	Note that for any $f \in \mathbb{F}_{\restbasis}$, $\mathcal{C}_{\restbasis}(\mathcal{C}_{\restbasis}^{-1}[f]) \equiv f$.
\end{definition}

The result of any non-linear transformation of a function $f \in \mathbb{F}_{\restbasis}$ is not guaranteed to belong to $\mathbb{F}_{\restbasis}$ and requires explicit projection to be used with other restricted functions.
This applies to the delta function of coordinates.
To avoid confusion with the common delta function, we introduce the restricted delta function.

\begin{definition}
\label{def:func-calculus:restricted-delta}
	The restricted delta function $\delta_{\restbasis} \in \mathbb{F}_{\restbasis}$ is defined as
	\begin{eqn*}
		\delta_{\restbasis}(\xvec^\prime, \xvec)
		= \sum_{\nvec \in \restbasis} \phi_{\nvec}^{\prime*} \phi_{\nvec}.
	\end{eqn*}
	Note that $\delta_{\restbasis}^*(\xvec^\prime, \xvec) = \delta_{\restbasis}(\xvec, \xvec^\prime)$.
\end{definition}

Any function can be projected to $\restbasis$ using the projection transformation.

\begin{definition}
\label{def:func-calculus:projector}
	Projection transformation
	\begin{eqn*}
		\mathcal{P}_{\restbasis} ::
		\mathbb{F} \rightarrow \mathbb{F}_{\restbasis} \\
		\mathcal{P}_{\restbasis}[f](\xvec)
		& = (\mathcal{C}_{\restbasis}(\mathcal{C}_{\restbasis}^{-1}[f])) (\xvec) \\
		& = \sum_{\nvec \in \restbasis} \phi_{\nvec} \int
			d\xvec^\prime\, \phi_{\nvec}^{\prime*} f^\prime \\
		& = \int d\xvec^\prime \delta_{\restbasis}(\xvec^\prime, \xvec) f^\prime,
	\end{eqn*}
	Obviously, $\mathcal{P}_{\fullbasis} \equiv \mathds{1}$.
\end{definition}

The conjugate of $\mathcal{P}_{\restbasis}$ is thus defined as
\begin{eqn}
	(\mathcal{P}_{\restbasis}[f](\xvec))^*
	= \int d\xvec^\prime \delta_{\restbasis}^*(\xvec^\prime, \xvec) f^{\prime*}
	= \mathcal{P}_{\restbasis}^* [f^*](\xvec).
\end{eqn}

Let $\mathcal{F}[f] :: \mathbb{F}_{\restbasis} \rightarrow \mathbb{F}$ be some transformation (note that the result is not guaranteed to belong to the restricted basis).
Because of the bijection between $\mathbb{F}_{\restbasis}$ and $\mathbb{C}^{|\restbasis|}$, $\mathcal{F}$ can be alternatively treated as a function of a vector of complex numbers:
\begin{eqn}
	\mathcal{F} :: \mathbb{C}^{|\restbasis|} \rightarrow \mathbb{C}^\infty \\
	\mathcal{F}(\balpha) \equiv \mathcal{C}_{\restbasis}^{-1}[\mathcal{F}[\mathcal{C}_{\restbasis}(\balpha)]].
\end{eqn}
Using this correspondence, we can define the functional differentiation.

\begin{definition}
\label{def:func-calculus:func-diff}
	Functional derivative is defined as
	\begin{eqn*}
		\frac{\delta}{\delta f^\prime} ::
		\left(
			\mathbb{F}_{\restbasis} \rightarrow \mathbb{F}
		\right)
		\rightarrow
		\left(
			\mathbb{R}^D \rightarrow \mathbb{F}_{\restbasis} \rightarrow \mathbb{F}
		\right) \\
		\frac{\delta \mathcal{F}[f]}{\delta f^\prime}
		= \sum_{\nvec \in \restbasis} \phi_{\nvec}^{\prime*}
			\frac{\partial \mathcal{F}(\balpha)}{\partial \alpha_{\nvec}}.
	\end{eqn*}
\end{definition}

Note that the transformation being returned differs from the one which was taken: the result of the new transformation is a function of the additional variable from $\mathbb{R}^D$ ($\xvec^\prime$).
This variable comes from the function we are differentiating by.

Functional derivatives behave in many ways similar to Wirtinger derivatives.
The detailed treatment can be found in~\cite{Dalton2011}.
In particular, the following useful lemma gives us the ability to differentiate functionals based on the intuition for common functions:

\begin{lemma}
	If $g(z)$ is a function of complex variable that can be expanded into series of $z^n (z^*)^m$, and functional $\mathcal{F}[f, f^*] \equiv g(f, f^*)$, $\mathcal{F} \in \mathbb{F}_{\restbasis} \rightarrow \mathbb{F}$, then $\delta \mathcal{F} / \delta f^\prime$ and $\delta \mathcal{F} / \delta f^{\prime*}$ can be treated as partial differentiation of the functional of two independent variables $f$ and $f^*$.
	In other words:
	\begin{eqn*}
		\frac{\delta \mathcal{F}}{\delta f^\prime}
		= \delta_{\restbasis}(\xvec^\prime, \xvec)
			\frac{\partial g(f, f^*)}{\partial f},
		\quad
		\frac{\delta \mathcal{F}}{\delta f^{\prime*}}
		= \delta_{\restbasis}^*(\xvec^\prime, \xvec)
			\frac{\partial g(f, f^*)}{\partial f^*}
	\end{eqn*}
\end{lemma}

Functional integration is defined as

\begin{definition}
	\begin{eqn*}
		\int \delta^2 f :: (\mathbb{F}_{\restbasis} \rightarrow \mathbb{F}) \rightarrow \mathbb{C} \\
		\int \delta^2 f \mathcal{F}[f]
		= \int d^2\balpha \mathcal{F}(\balpha)
		= \left(
			\prod_{\nvec \in \restbasis} \int d^2\alpha_{\nvec}
		\right) \mathcal{F}(\balpha).
	\end{eqn*}
    If the basis contains an infinite number of modes, the integral is treated as a limit $|\restbasis| \rightarrow \infty$.
    \todo{Product of integrals means successive applications of those integrals --- do we need to state it explicitly?}
\end{definition}

Functional integration has the Fourier-like property analogous to Lemma~\lmmref{c-numbers:fourier-of-moments}, but its statement requires the definition of the delta functional:

\begin{definition}
\label{def:func-calculus:delta-functional}
	For a function $\Lambda \in \mathbb{F}_{\restbasis}$ the delta functional is
	\begin{eqn*}
		\Delta_{\restbasis}[\Lambda]
		\equiv \prod_{\nvec \in \restbasis} \delta(\Real \lambda_{\nvec}) \delta(\Imag \lambda_{\nvec}),
	\end{eqn*}
	where $\blambda = \mathcal{C}_{\restbasis}^{-1}[\Lambda]$.
\end{definition}

The delta functional has the same property as the common delta function:
\begin{eqn}
	\int \delta^2 \Lambda \mathcal{F}[\Lambda] \Delta_{\restbasis}[\Lambda]
	& = \left(
			\prod_{\nvec \in \restbasis} \int d^2\lambda_{\nvec}
		\right)
		\mathcal{F}(\blambda)
		\prod_{\nvec \in \restbasis} \delta(\Real \lambda_{\nvec}) \delta(\Imag \lambda_{\nvec}) \\
	& = \left. \mathcal{F}(\blambda) \right|_{\forall \nvec \in \restbasis\, \lambda_{\nvec} = 0} \\
	& = \left. \mathcal{F}[\Lambda] \right|_{\Lambda \equiv 0}
\end{eqn}

\begin{lemma}[Functional extension of \lmmref{c-numbers:fourier-of-moments}]
\label{lmm:func-calculus:fourier-of-moments}
	For $\Psi \in \mathbb{F}_{\restbasis}$ and $\Lambda \in \mathbb{F}_{\restbasis}$, and for any non-negative integers $r$ and $s$:
	\begin{eqn*}
		\int \delta^2\Psi\, \Psi^r (\Psi^*)^s \exp \left(
				\int d\xvec \left( -\Lambda \Psi^* + \Lambda^* \Psi \right)
			\right) \\
		= \pi^{2|\restbasis|}
			\left( -\frac{\delta}{\delta \Lambda^*} \right)^r
			\left( \frac{\delta}{\delta \Lambda} \right)^s
			\Delta_{\restbasis}[\Lambda]
	\end{eqn*}
\end{lemma}
\begin{proof}
The proof consists of expanding functions into sums of modes and applying \lmmref{c-numbers:fourier-of-moments} $|\restbasis|$ times.
\end{proof}

\begin{lemma}
\label{lmm:func-calculus:zero-integrals}
	For a bounded functional $F(\blambda, \blambda^*)$
	\begin{eqn*}
		\int \delta^2\Lambda
			\frac{\delta}{\delta \Lambda^\prime} \left(
				D[\Lambda, \Lambda^*, \Psi, \Psi^*]
				F[\Lambda, \Lambda^*]
			\right)
		& = 0 \\
		\int \delta^2\Lambda
			\frac{\delta}{\delta \Lambda^{\prime*}}
			\left(
				D[\Lambda, \Lambda^*, \Psi, \Psi^*]
				F[\Lambda, \Lambda^*]
			\right)
		& = 0.
	\end{eqn*}
\end{lemma}
\begin{proof}
Proved by expanding integrals and differentials into modes and applying \lmmref{c-numbers:zero-integrals}.
\end{proof}

\begin{lemma}
\label{lmm:func-calculus:zero-delta-integrals}
	For $\Lambda \in \mathbb{F}_{\restbasis}$ \todo{Any limitations on $F$?}
	\begin{eqn*}
		\int \delta^2\Lambda
			\frac{\delta}{\delta \Lambda} \left(
				\left(
					\left( \frac{\delta}{\delta \Lambda} \right)^s
					\left( -\frac{\delta}{\delta \Lambda^*} \right)^r
					\Delta_{\restbasis}[\Lambda]
				\right)
				F[\lambda, \lambda^*]
			\right)
		& = 0 \\
		\int \delta^2\Lambda
			\frac{\delta}{\delta \Lambda^*} \left(
				\left(
					\left( \frac{\delta}{\delta \Lambda} \right)^s
					\left( -\frac{\delta}{\delta \Lambda^*} \right)^r
					\Delta_{\restbasis}[\Lambda]
				\right)
				F[\lambda, \lambda^*]
			\right)
		& = 0 \\
	\end{eqn*}
\end{lemma}
\begin{proof}
Proved by expanding functional integration and differentials into modes and integrating separately over each $\lambda_{\nvec}$, using the fact that any differential of the delta function is zero on the infinity.
\end{proof}

In order to perform transformations of master equations, we will need a lemma that justifies the ``relocation'' of the Laplacian (which is a part of the kinetic term in the Hamiltonian) inside the functional integral.

\begin{lemma}
\label{lmm:func-calculus:move-laplacian}
	If $\mathcal{F} \in \mathbb{F}_{\restbasis} \rightarrow \mathbb{F}$, and $\forall \nvec \in \restbasis, \xvec \in \partial A$ $\phi_{\nvec}(\xvec) = 0$, then
	\begin{eqn*}
		\int\limits_A d\xvec \left(
			\nabla^2 \frac{\delta}{\delta \Psi}
		\right) \Psi \mathcal{F}[\Psi, \Psi^*]
		= \int\limits_A d\xvec \frac{\delta}{\delta \Psi}
		( \nabla^2 \Psi ) \mathcal{F}[\Psi, \Psi^*]
	\end{eqn*}
\end{lemma}
\begin{proof}
The proof consists of a function expansion into a mode sum and an application of Green's first identity.
\end{proof}

Note that the above lemma imposes an additional requirement for basis functions, but in practical applications it is always satisfied.
For example, in plane wave basis eigenfunctions are equal to zero at the border of the bounding box, and in harmonic oscillator basis they are equal to zero on the infinity (which can be considered the boundary of their integration area).
Hereinafter we will assume that this condition is true for any basis we work with.

% =============================================================================
\section{Functional Fokker-Planck equation}
% =============================================================================

The general approach to numerical solution of the Fokker-Planck equation is to transform it to the equivalent set of stochastic differential equations (SDEs).
In the textbooks this transformation is defined for real variables only~\cite{Risken1996}, while we have functional FPE with complex-valued functions.

Our starting point is the reformulation of the theorem for real-valued multivariable FPE from~\cite{Risken1996} in terms of vectors and matrices:

\begin{lemma}[FPE--SDEs correspondence in convenient form.]
\label{lmm:app-fpe:fpe-sde-real}
    If $\zvec^T \equiv (z_1 \ldots z_M)$ is a set of real-valued variables,
    Fokker-Planck equation
    \begin{eqn*}
    	\frac{dW}{dt}
    	= -\boldsymbol{\partial}_{\zvec}^T \boldsymbol{a} W
    	+ \frac{1}{2} \Trace{ \boldsymbol{\partial}_{\zvec} \boldsymbol{\partial}_{\zvec}^T B B^T } W
    \end{eqn*}
    is equivalent to a set of stochastic differential equations in It\^{o} form
    \begin{eqn*}
    	d\zvec = \boldsymbol{a} dt + B d\Zvec
    \end{eqn*}
    and to a set of stochastic differential equations in Stratonovich form
    \begin{eqn*}
    	d\zvec = (\boldsymbol{a} - \boldsymbol{s})dt + B d\Zvec,
    \end{eqn*}
    where the noise-induced (or spurious) drift vector $\boldsymbol{s}$ has elements
    \begin{eqn*}
    	s_i
    	= \sum_{k,j} B_{kj} \frac{\partial}{\partial z_k} B_{ij}
    	= \Trace{B^T \boldsymbol{\partial}_z \boldsymbol{e}_i^T B},
    \end{eqn*}
    $\boldsymbol{e}_i$ being the unit vector with elements $(\boldsymbol{e}_i)_j = \delta_{ij}$.
    Here $W \equiv W(\zvec)$ is a probability distribution,
    $\boldsymbol{a} \equiv \boldsymbol{a}(\zvec)$ is a vector function,
    $B \equiv B(\zvec)$ is a matrix function ($B$ having size $M \times L$, where $L$ corresponds to the number of noise sources),
    $\boldsymbol{\partial}_{\zvec}^T \equiv (\partial_{z_1} \ldots \partial_{z_M})$ is a vector differential,
    and $d\Zvec$ is a standard $L$-dimensional real-valued Wiener process.
\end{lemma}
\begin{proof}
For details see~\cite{Risken1996}, sections 3.3 and 3.4.
\end{proof}

\begin{theorem}
\label{thm:app-fpe:fpe-sde-complex}
    If $\boldsymbol{\alpha}^T \equiv (\alpha_1 \ldots \alpha_M)$ is a set of complex-valued variables,
    Fokker-Planck equation
    \begin{eqn*}
    	\frac{dW}{dt}
    	= -\boldsymbol{\partial}_{\boldsymbol{\alpha}}^T \boldsymbol{a} W - \boldsymbol{\partial}_{\boldsymbol{\alpha}^*}^T \boldsymbol{a}^* W
    	+ \Trace{ \boldsymbol{\partial}_{\boldsymbol{\alpha}^*} \boldsymbol{\partial}_{\boldsymbol{\alpha}}^T B B^H } W
    \end{eqn*}
    is equivalent to a set of stochastic differential equations in It\^{o} form
    \begin{eqn*}
    	d\boldsymbol{\alpha} = \boldsymbol{a} dt + B d\Zvec,
    \end{eqn*}
    or to Stratonovich form
    \begin{eqn*}
    	d\boldsymbol{\alpha} = (\boldsymbol{a} - \boldsymbol{s}) dt + B d\Zvec,
    \end{eqn*}
    where noise-induced drift term is
    \begin{eqn*}
    	s_j = \Trace{ B^H \boldsymbol{\partial}_{\boldsymbol{\alpha}^*} \boldsymbol{e}_j^T B },
    \end{eqn*}
    and $d\Zvec = (d\boldsymbol{X} + id\boldsymbol{Y}) / \sqrt{2}$ is an $M$-dimensional complex-valued Wiener process,
    containing two real-valued $L$-dimensional Wiener processes $d\boldsymbol{X}$ and $d\boldsymbol{Y}$.
\end{theorem}
\begin{proof}
Proved straightforwardly by transforming the equation to real variables and applying \lmmref{app-fpe:fpe-sde-real}.
\end{proof}

\begin{theorem}[Multi-component extension of \thmref{app-fpe:fpe-sde-complex}]
\label{thm:app-fpe:mc-fpe-sde}
    If $\boldsymbol{\alpha}^{(c)},\, c = 1..C$ are $C$ sets of complex variables $\boldsymbol{\alpha}^{(c)} \equiv (\alpha_1^{(c)} \ldots \alpha_M^{(c)})$, then the Fokker-Planck equation
    \begin{eqn}
    	\frac{dW}{dt}
    	= & - \sum_{c=1}^C \boldsymbol{\partial}_{\boldsymbol{\alpha}^{(c)}}^T \boldsymbol{a}^{(c)} W
    	- \sum_{c=1}^C \boldsymbol{\partial}_{(\boldsymbol{\alpha}^{(c)})^*}^T (\boldsymbol{a}^{(c)})^* W \\
        & + \sum_{m=1}^c \sum_{n=1}^c
    		\Trace{
    			\boldsymbol{\partial}_{(\boldsymbol{\alpha}^{(m)})^*}
    			\boldsymbol{\partial}_{\boldsymbol{\alpha}^{(n)}}^T
    			B^{(n)} (B^{(m)})^H
    		} W
    \end{eqn}
    is equivalent to a set of stochastic differential equations in It\^{o} form
    \begin{eqn}
    	d\boldsymbol{\alpha}^{(c)} = \boldsymbol{a}^{(c)} dt + B^{(c)} d\Zvec,\, c = 1..C
    \end{eqn}
    or to Stratonovich form
    \begin{eqn*}
    	d\boldsymbol{\alpha}^{(c)} = (\boldsymbol{a}^{(c)} - \boldsymbol{s}^{(c)}) dt + B^{(c)} d\Zvec,
    \end{eqn*}
    where noise-induced drift term is
    \begin{eqn*}
    	s_j^{(c)} = \sum_{d=1}^C
    		\Trace{ (B^{(d)})^H \boldsymbol{\partial}_{(\boldsymbol{\alpha}^{(d)})^*} \boldsymbol{e}_j^T B^{(c)} },
    \end{eqn*}
    and $d\Zvec$ is an $L$-dimensional complex-valued Wiener process.
\end{theorem}
\begin{proof}
Proved by joining vectors from all components into one vector and applying \thmref{app-fpe:fpe-sde-complex}.
\end{proof}

\begin{theorem}
\label{thm:app-fpe:fpe-sde-func}
    For the probability distribution $W[\Psivec, \Psivec^*] \in (\mathbb{F}_{\restbasis}^C \rightarrow \mathbb{R})$ the functional FPE
    \begin{eqn*}
\fl    	\frac{dW}{dt}
    	= \int d\xvec \left(
    		- \sum_{j=1}^C \frac{\delta}{\delta \Psi_j} \mathcal{A}_j
    		- \sum_{j=1}^C \frac{\delta}{\delta \Psi_j^*} \mathcal{A}_j^*
    		+ \sum_{j=1}^C \sum_{k=1}^C \frac{\delta^2}{\delta \Psi_j^* \delta \Psi_k}
    			\sum_{\lvec} \mathcal{B}_{\lvec}^{(k)} (\mathcal{B}_{\lvec}^{(j)})^*
    	\right) W
    \end{eqn*}
    is equivalent to the set of SDEs in It\^{o} form
    \begin{eqn*}
    	d\Psi_j = \mathcal{P}_{\restbasis_j} \left[
    		\mathcal{A}^{(j)} dt + \sum_{\lvec} \mathcal{B}_{\lvec}^{(j)} dQ_{\lvec}
    	\right],
    \end{eqn*}
    or in Stratonovich form
    \begin{eqn*}
    	d\Psi_j = \mathcal{P}_{\restbasis_j} \left[
    		(\mathcal{A}^{(j)} - \mathcal{S}^{(j)}) dt + \sum_{\lvec} \mathcal{B}_{\lvec}^{(j)} dQ_{\lvec}
    	\right],
    \end{eqn*}
    where
    \begin{eqn*}
    	\mathcal{S}^{(j)} = \sum_{c=1}^C \sum_{\lvec}
    		(\mathcal{B}_{\lvec}^{(c)})^*
    		\frac{\delta}{\delta \Psi_c^*}
    		\mathcal{B}_{\lvec}^{(j)},
    \end{eqn*}
    and $Q_{\lvec}$ is a functional Wiener process:
    \begin{eqn*}
    	Q_{\lvec} = \sum_{\nvec \in \fullbasis} \phi_{\nvec} Z_{\lvec,\nvec}.
    \end{eqn*}
\end{theorem}
\begin{proof}
Proved by expanding functional derivatives and applying \thmref{app-fpe:mc-fpe-sde}.
The diffusion term has to be transformed in order to conform to the theorem:
\begin{eqn}
\fl	\int d\xvec \phi_{j,\mvec} \phi_{k,\nvec}^* \sum_{\lvec} \mathcal{B}_{\lvec}^{(k)} (\mathcal{B}_{\lvec}^{(j)})^*
	& = \int d\xvec \int d\xvec^\prime
			\phi_{j,\mvec}^\prime \phi_{k,\nvec}^*
			\sum_{\lvec} (\mathcal{B}_{\lvec}^{(j)})^{\prime *} \mathcal{B}_{\lvec}^{(k)}
			\delta(\xvec - \xvec^\prime) \\
	& = \int d\xvec \int d\xvec^\prime
			\phi_{j,\mvec}^\prime \phi_{k,\nvec}^*
			\sum_{\lvec} (\mathcal{B}_{\lvec}^{(j)})^{\prime *} \mathcal{B}_{\lvec}^{(k)}
			\sum_{\pvec \in \fullbasis} \phi_{\pvec}^{\prime*} \phi_{\pvec} \\
	& = \sum_{\pvec \in \fullbasis, \lvec}
		\int d\xvec
			\phi_{j,\mvec} (\mathcal{B}_{\lvec}^{(j)})^* \phi_{\pvec}^*
		\int d\xvec
			\phi_{k,\nvec}^* \mathcal{B}_{\lvec}^{(k)} \phi_{\pvec}
\end{eqn}
Grouping terms back and recognising the definition of projection transformation, one gets the statement of the theorem.
\end{proof}


\section*{References}
%\bibliographystyle{unsrt} % recommended by IOP guide, but produces results inconsistent with their requirements
\bibliographystyle{iopart-num} % actually does the job
\bibliography{qsim-long}

\end{document}