% =============================================================================
\section{Wigner truncation and Fokker-Planck equation}
% =============================================================================

Now we have all necessary tools to transform the master equation~\eqnref{master-eqn:master-eqn} with the Wigner transformation from \defref{func-wigner:w-transformation} to the partial differential equation.

Namely, the single-particle term~\eqnref{master-eqn:single-particle} is transformed using \thmref{transformations:w-commutator1} and \thmref{transformations:w-laplacian-commutator1} (since $K_j$ is basically a sum of Laplacian operator and functions of $\xvec$):
\begin{eqn}
	\mathcal{W} \left[ [ \int d\xvec \Psiop_j^\dagger K_{jk} \Psiop_k, \hat{\rho} ] \right]
	= \int d\xvec \left(
			- \frac{\delta}{\delta \Psi_j} K_{jk} \Psi_k
			+ \frac{\delta}{\delta \Psi_k^*} K_{jk} \Psi_j^*
		\right)
		W,
\end{eqn}
where Wigner function $W = \mathcal{W}[\hat{\rho}]$.
Nonlinear term is transformed with \thmref{transformations:w-commutator2} (minding the locality of interaction and assuming $U_{kj} = U_{jk}$):
\begin{eqn}
\fl	\mathcal{W} \left[
		[
			\int d\xvec \frac{U_{jk}}{2}
				\Psiop_j^\dagger \Psiop_k^\dagger \Psiop_j \Psiop_k,
			\hat{\rho}
		]
	\right]
	= & \int d\xvec U_{jk} \left(
		\frac{\delta}{\delta \Psi_j} \left(
			- \Psi_j \Psi_k \Psi_k^*
			+ \frac{\delta_{\restbasis}(\xvec, \xvec)}{2} ( \delta_{jk} \Psi_k + \Psi_j )
		\right) \right. \\
	&	\left. + \frac{\delta}{\delta \Psi_j^*} \left(
			\Psi_j^* \Psi_k \Psi_k^*
			- \frac{\delta_{\restbasis}(\xvec, \xvec)}{2} ( \delta_{jk} \Psi_k^* + \Psi_j^* )
		\right) \right. \\
	&	\left.
			+ \frac{\delta}{\delta \Psi_j}
			\frac{\delta}{\delta \Psi_j^*}
			\frac{\delta}{\delta \Psi_k}
			\frac{1}{4} \Psi_k
			- \frac{\delta}{\delta \Psi_j}
			\frac{\delta}{\delta \Psi_j^*}
			\frac{\delta}{\delta \Psi_k^*}
			\frac{1}{4} \Psi_k^*
		\right) W.
\end{eqn}

Loss terms~\eqnref{master-eqn:loss-term} are transformed with \thmref{transformations:w-losses}.
\todo{Not writing the resulting expression here, because with the absence of truncation it is too long, and is practically the same as in theorem statement.}

Assuming that $K_{jk}$, $U_{jk}$ and $\kappa_{\lvec}$ are real-valued, all the transformations described above result in a partial differential equation for $W$ of the form
\begin{eqn}
\fl	\frac{\partial W}{\partial t} = \int d^D\xvec \left\{
    	- \sum_{j=1}^C \frac{\delta}{\delta \Psi_j} A_j
    	- \sum_{j=1}^C \frac{\delta}{\delta \Psi_j^*} A_j^*
    	+ \sum_{j=1}^C \sum_{k=1}^C \frac{\delta^2}{\delta \Psi_j^* \delta \Psi_k} D_{jk}
		+ \mbox{O} \left[ \frac{\delta^3}{\delta\Psi_j^3} \right]
	\right\} W.
\end{eqn}
The terms of order higher than 2 are produced both by the nonlinear term in the Hamiltonian and loss terms.
Such an equation could be solved without additional approximations if there were only orders up to 3 (which means the absence of losses)~\cite{Polkovnikov2003}, but in most cases all terms except for first- and second-order ones are truncated.
In the assumption of the state being coherent, the condition for truncation can be shown to be~\cite{Sinatra2002}
\begin{eqn}
    N \gg |\restbasis|,
\end{eqn}
where $N$ is the number of atoms.
This condition is equivalent to~\cite{Norrie2006}
\begin{eqn}
    \delta_{\restbasis_j}(\xvec, \xvec) \ll | \Psi_j |^2.
\end{eqn}

Wigner truncation allows us to simplify the results of \thmref{transformations:w-commutator2} and \thmref{transformations:w-losses}.

\begin{lemma}
    Assuming the conditions for Wigner truncation are satisfied,
    the result of Wigner transformation of the nonlinear term can be written as
    \begin{eqn*}
    	\mathcal{W} \left[
    		[
    			\frac{U_{jk}}{2}
    				\Psiop_j^\dagger \Psiop_k^\dagger \Psiop_j \Psiop_k,
    			\hat{\rho}
    		]
    	\right]
    	= U_{jk} \left(
    		\frac{\delta}{\delta \Psi_j^*} \Psi_j^* \Psi_k \Psi_k^*
    		- \frac{\delta}{\delta \Psi_j} \Psi_j \Psi_k \Psi_k^*
    	\right) W.
    \end{eqn*}
\end{lemma}

\begin{lemma}
    Assuming the conditions for Wigner truncation are satisfied, the result of Wigner transformation of the loss term can be written as
    \begin{eqn*}
\fl    	\mathcal{W}[\mathcal{L}_{\lvec}[\hat{\rho}]]
    	= \sum_{n=1}^C
    			\frac{\delta}{\delta \Psi_n^*} \frac{\partial O_{\lvec}}{\partial \Psi_n} O_{\lvec}^*
    	+ \sum_{n=1}^C
    		\frac{\delta}{\delta \Psi_n} \frac{\partial O_{\lvec}^*}{\partial \Psi_n^*} O_{\lvec}
    	+ \sum_{n=1}^C \sum_{p=1}^C
    		\frac{\delta^2}{\delta \Psi_n^* \delta \Psi_p}
    		\frac{\partial O_{\lvec}}{\partial \Psi_n}
    		\frac{\partial O_{\lvec}^*}{\partial \Psi_p^*},
    \end{eqn*}
    where $O_{\lvec} \equiv O_{\lvec}[\Psivec] = \prod_{c=1}^C \Psi_c^{l_c}$.
\end{lemma}
\begin{proof}
The proof is basically a simplification of the result of \thmref{transformations:w-losses} under certain conditions.
First, we are neglecting all occurrences of $\delta_{\restbasis}$, which means setting $m_c = 0$ for every $c$.
Second, we are dropping all terms with high order differentials,
which can be expressed as limiting $\sum j_c + \sum k_c \le 2$.
The only combinations of $j_c$ and $k_c$ for which $Z(\jvec, \kvec)$ is not zero are thus
$\{ j_c = \delta_{cn}, k_c = 0, n \in [1, C] \}$,
$\{ j_c = 0, k_c = \delta_{cn}, n \in [1, C] \}$ and
$\{ j_c = \delta_{cn}, k_c = \delta_{cp}, n \in [1, C], p \in [1, C] \}$.
These combinations produce terms with $\delta/\delta \Psi_n^*$,
$\delta/\delta \Psi_n$ and
$\delta^2/\delta \Psi_p \delta \Psi_n^*$ respectively.
Applying these conditions one can get the statement of the theorem.
\end{proof}

Thus the truncated Fokker-Planck equation is
\begin{eqn}
\fl	\frac{dW}{dt}
	= \int d\xvec \left(
		- \sum_{j=1}^C \frac{\delta}{\delta \Psi_j} A_j
		- \sum_{j=1}^C \frac{\delta}{\delta \Psi_j^*} A_j^*
		+ \sum_{j=1}^C \sum_{k=1}^C \frac{\delta^2}{\delta \Psi_j^* \delta \Psi_k} D_{jk}
	\right) W,
\end{eqn}
where
\begin{eqn}
	A_j = -\frac{i}{\hbar} \left(
			\sum_{k=1}^C K_{jk} \Psi_k
			+ \sum_{k=1}^C U_{jk} \Psi_j \Psi_k \Psi_k^*
		\right)
		- \sum_{\lvec} \kappa_{\lvec} \frac{\partial O_{\lvec}^*}{\partial \Psi_j^*} O_{\lvec},
\end{eqn}
and
\begin{eqn}
	D_{jk} = \sum_{\lvec} \kappa_{\lvec}
		\frac{\partial O_{\lvec}}{\partial \Psi_j}
		\frac{\partial O_{\lvec}^*}{\partial \Psi_k^*}.
\end{eqn}

Since the diffusion matrix is positive-definite, the truncated Wigner function $W$ is a probability distribution
Therefore the equation can be further transformed to the equivalent set of stochastic differential equations in It\^{o} form as described by \thmref{app-fpe:fpe-sde-func}.

\begin{eqn}
\label{eqn:fpe:sdes}
	d\Psi_j = \mathcal{P}_{\restbasis_j} \left[
		A^{(j)} dt + \sum_{\lvec} B_{\lvec}^{(j)} Q_{\lvec}
	\right],
\end{eqn}
where
\begin{eqn}
    B_{\lvec}^{(j)} = \sqrt{\kappa_{\lvec}} \frac{\partial O_{\lvec}}{\partial \Psi_j},
\end{eqn}
and $Q_{\lvec}$ is a functional Wiener process:
\begin{eqn}
	Q_{\lvec} = \sum_{\nvec \in \fullbasis} \phi_j Z_{\lvec,\nvec},
\end{eqn}
and $Z_{\lvec,\nvec}$ are, in turn, independent complex-valued Wiener processes.
Alternatively, in Stratonovich form the SDEs look like
\begin{eqn}
	d\Psi_j = \mathcal{P}_{\restbasis_j} \left[
		(\mathcal{A}^{(j)} - \mathcal{S}^{(j)}) dt + \sum_{\lvec} B_{\lvec}^{(j)} Q_{\lvec}
	\right],
\end{eqn}
where the Stratonovich term is
\begin{eqn}
	\mathcal{S}^{(j)}
	= \sum_{n=1}^C \sum_{\lvec} \kappa_{\lvec}
		\frac{\partial O_{\lvec}}{\partial \Psi_n}
		\left(\frac{\partial^2 O_{\lvec}}{\partial \Psi_n \partial \Psi_j} \right)^*
		\delta_{\restbasis_n} (\xvec, \xvec).
\end{eqn}

These equations can now be solved using conventional methods, and any required expectations symmetrically ordered operator products can be obtained from their solution using \thmref{func-wigner:moments}:
\begin{eqn}
    \langle \symprod{
        \prod_{c=1}^C \Psiop_c^{r_c} (\Psiop_c^\dagger)^{s_c}
    } \rangle
    & = \int \delta^2\Psi_1 \ldots \int \delta^2\Psi_C\,
		    \prod_{c=1}^C \Psi_c^{r_c} (\Psi_c^*)^{s_c} W \\
    & \approx \pathavg{
        \prod_{c=1}^C \Psi_c^{r_c} (\Psi_c^*)^{s_c}
    },
\end{eqn}
where $r_c$ and $s_c$ is some set of non-negative integers, and $\pathavg{}$ stands for the average over the simulation paths.
