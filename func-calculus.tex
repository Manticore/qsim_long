% =============================================================================
\section{Functional calculus}
% =============================================================================

This section outlines the functional calculus, which is heavily used throughout the paper.
Detailed description is given in~\cite{Dalton2011}, and here we only provide some important definitions and results which are used later in the paper.
In this section we will use the definitions from the \secref{func-operators}, namely the full basis $\fullbasis$ and the restricted basis $\restbasis$.
Given the basis, we can define the correspondence between some function of coordinates and its representation in mode space.

\begin{definition}
	Let $\mathbb{F}$ be a space of all functions of coordinates, which consists only of modes from $\restbasis$: $\mathbb{F}_{\restbasis} \equiv (\mathbb{R}^D \rightarrow \mathbb{C})_{\restbasis}$ (restricted functions).
	Composition transformation creates a function from a vector of mode populations:
	\begin{eqn*}
		\mathcal{C}_{\restbasis} :: \mathbb{C}^{|\restbasis|} \rightarrow \mathbb{F}_{\restbasis} \\
		\mathcal{C}_{\restbasis}(\balpha) = \sum_{\nvec \in L} \phi_{\nvec} \alpha_{\nvec}.
	\end{eqn*}
	Decomposition transformation, correspondingly, creates a vector of populations out of a function:
	\begin{eqn*}
		\mathcal{C}_{\restbasis}^{-1} :: \mathbb{F} \rightarrow \mathbb{C}^{|\restbasis|} \\
		(\mathcal{C}_{\restbasis}^{-1}[f])_{\nvec}
		= \int d\xvec \phi_{\nvec}^* f,\,{\nvec} \in \restbasis.
	\end{eqn*}
	Note that for any $f \in \mathbb{F}_{\restbasis}$, $\mathcal{C}_{\restbasis}(\mathcal{C}_{\restbasis}^{-1}[f]) \equiv f$.
\end{definition}

The result of any non-linear transformation of a function $f \in \mathbb{F}_{\restbasis}$ is not guaranteed to belong to $\mathbb{F}_{\restbasis}$ and requires explicit projection to be used with other restricted functions.
This applies to the delta function of coordinates.
To avoid confusion with the common delta function, we introduce the restricted delta function.

\begin{definition}
\label{def:func-calculus:restricted-delta}
	The restricted delta function $\delta_{\restbasis} \in \mathbb{F}_{\restbasis}$ is defined as
	\begin{eqn*}
		\delta_{\restbasis}(\xvec^\prime, \xvec)
		= \sum_{\nvec \in \restbasis} \phi_{\nvec}^{\prime*} \phi_{\nvec}.
	\end{eqn*}
	Note that $\delta_{\restbasis}^*(\xvec^\prime, \xvec) = \delta_{\restbasis}(\xvec, \xvec^\prime)$.
\end{definition}

Any function can be projected to $\restbasis$ using the projection transformation.

\begin{definition}
\label{def:func-calculus:projector}
	Projection transformation
	\begin{eqn*}
		\mathcal{P}_{\restbasis} ::
		\mathbb{F} \rightarrow \mathbb{F}_{\restbasis} \\
		\mathcal{P}_{\restbasis}[f](\xvec)
		& = (\mathcal{C}_{\restbasis}(\mathcal{C}_{\restbasis}^{-1}[f])) (\xvec) \\
		& = \sum_{\nvec \in \restbasis} \phi_{\nvec} \int
			d\xvec^\prime\, \phi_{\nvec}^{\prime*} f^\prime \\
		& = \int d\xvec^\prime \delta_{\restbasis}(\xvec^\prime, \xvec) f^\prime,
	\end{eqn*}
	Obviously, $\mathcal{P}_{\fullbasis} \equiv \mathds{1}$.
\end{definition}

The conjugate of $\mathcal{P}_{\restbasis}$ is thus defined as
\begin{eqn}
	(\mathcal{P}_{\restbasis}[f](\xvec))^*
	= \int d\xvec^\prime \delta_{\restbasis}^*(\xvec^\prime, \xvec) f^{\prime*}
	= \mathcal{P}_{\restbasis}^* [f^*](\xvec).
\end{eqn}

Let $\mathcal{F}[f] :: \mathbb{F}_{\restbasis} \rightarrow \mathbb{F}$ be some transformation (note that the result is not guaranteed to belong to the restricted basis).
Because of the bijection between $\mathbb{F}_{\restbasis}$ and $\mathbb{C}^{|\restbasis|}$, $\mathcal{F}$ can be alternatively treated as a function of a vector of complex numbers:
\begin{eqn}
	\mathcal{F} :: \mathbb{C}^{|\restbasis|} \rightarrow \mathbb{C}^\infty \\
	\mathcal{F}(\balpha) \equiv \mathcal{C}_{\restbasis}^{-1}[\mathcal{F}[\mathcal{C}_{\restbasis}(\balpha)]].
\end{eqn}
Using this correspondence, we can define the functional differentiation.

\begin{definition}
\label{def:func-calculus:func-diff}
	Functional derivative is defined as
	\begin{eqn*}
		\frac{\delta}{\delta f^\prime} ::
		\left(
			\mathbb{F}_{\restbasis} \rightarrow \mathbb{F}
		\right)
		\rightarrow
		\left(
			\mathbb{R}^D \rightarrow \mathbb{F}_{\restbasis} \rightarrow \mathbb{F}
		\right) \\
		\frac{\delta \mathcal{F}[f]}{\delta f^\prime}
		= \sum_{\nvec \in \restbasis} \phi_{\nvec}^{\prime*}
			\frac{\partial \mathcal{F}(\balpha)}{\partial \alpha_{\nvec}}.
	\end{eqn*}
\end{definition}

Note that the transformation being returned differs from the one which was taken: the result of the new transformation is a function of the additional variable from $\mathbb{R}^D$ ($\xvec^\prime$).
This variable comes from the function we are differentiating by.

Functional derivatives behave in many ways similar to Wirtinger derivatives.
The detailed treatment can be found in~\cite{Dalton2011}.
In particular, the following useful lemma gives us the ability to differentiate functionals based on the intuition for common functions:

\begin{lemma}
	If $g(z)$ is a function of complex variable that can be expanded into series of $z^n (z^*)^m$, and functional $\mathcal{F}[f, f^*] \equiv g(f, f^*)$, $\mathcal{F} \in \mathbb{F}_{\restbasis} \rightarrow \mathbb{F}$, then $\delta \mathcal{F} / \delta f^\prime$ and $\delta \mathcal{F} / \delta f^{\prime*}$ can be treated as partial differentiation of the functional of two independent variables $f$ and $f^*$.
	In other words:
	\begin{eqn*}
		\frac{\delta \mathcal{F}}{\delta f^\prime}
		= \delta_{\restbasis}(\xvec^\prime, \xvec)
			\frac{\partial g(f, f^*)}{\partial f},
		\quad
		\frac{\delta \mathcal{F}}{\delta f^{\prime*}}
		= \delta_{\restbasis}^*(\xvec^\prime, \xvec)
			\frac{\partial g(f, f^*)}{\partial f^*}
	\end{eqn*}
\end{lemma}

Functional integration is defined as

\begin{definition}
	\begin{eqn*}
		\int \delta^2 f :: (\mathbb{F}_{\restbasis} \rightarrow \mathbb{F}) \rightarrow \mathbb{C} \\
		\int \delta^2 f \mathcal{F}[f]
		= \int d^2\balpha \mathcal{F}(\balpha)
		= \left(
			\prod_{\nvec \in \restbasis} \int d^2\alpha_{\nvec}
		\right) \mathcal{F}(\balpha).
	\end{eqn*}
    If the basis contains an infinite number of modes, the integral is treated as a limit $|\restbasis| \rightarrow \infty$.
    \todo{Product of integrals means successive applications of those integrals --- do we need to state it explicitly?}
\end{definition}

Functional integration has the Fourier-like property analogous to Lemma~\lmmref{c-numbers:fourier-of-moments}, but its statement requires the definition of the delta functional:

\begin{definition}
\label{def:func-calculus:delta-functional}
	For a function $\Lambda \in \mathbb{F}_{\restbasis}$ the delta functional is
	\begin{eqn*}
		\Delta_{\restbasis}[\Lambda]
		\equiv \prod_{\nvec \in \restbasis} \delta(\Real \lambda_{\nvec}) \delta(\Imag \lambda_{\nvec}),
	\end{eqn*}
	where $\blambda = \mathcal{C}_{\restbasis}^{-1}[\Lambda]$.
\end{definition}

The delta functional has the same property as the common delta function:
\begin{eqn}
	\int \delta^2 \Lambda \mathcal{F}[\Lambda] \Delta_{\restbasis}[\Lambda]
	& = \left(
			\prod_{\nvec \in \restbasis} \int d^2\lambda_{\nvec}
		\right)
		\mathcal{F}(\blambda)
		\prod_{\nvec \in \restbasis} \delta(\Real \lambda_{\nvec}) \delta(\Imag \lambda_{\nvec}) \\
	& = \left. \mathcal{F}(\blambda) \right|_{\forall \nvec \in \restbasis\, \lambda_{\nvec} = 0} \\
	& = \left. \mathcal{F}[\Lambda] \right|_{\Lambda \equiv 0}
\end{eqn}

\begin{lemma}[Functional extension of \lmmref{c-numbers:fourier-of-moments}]
\label{lmm:func-calculus:fourier-of-moments}
	For $\Psi \in \mathbb{F}_{\restbasis}$ and $\Lambda \in \mathbb{F}_{\restbasis}$, and for any non-negative integers $r$ and $s$:
	\begin{eqn*}
		\int \delta^2\Psi\, \Psi^r (\Psi^*)^s \exp \left(
				\int d\xvec \left( -\Lambda \Psi^* + \Lambda^* \Psi \right)
			\right) \\
		= \pi^{2|\restbasis|}
			\left( -\frac{\delta}{\delta \Lambda^*} \right)^r
			\left( \frac{\delta}{\delta \Lambda} \right)^s
			\Delta_{\restbasis}[\Lambda]
	\end{eqn*}
\end{lemma}
\begin{proof}
The proof consists of expanding functions into sums of modes and applying \lmmref{c-numbers:fourier-of-moments} $|\restbasis|$ times.
\end{proof}

In order to perform transformations of master equations, we will need a lemma that justifies the ``relocation'' of the Laplacian (which is a part of the kinetic term in the Hamiltonian) inside the functional integral.

\begin{lemma}
\label{lmm:func-calculus:move-laplacian}
	If $\mathcal{F} \in \mathbb{F}_{\restbasis} \rightarrow \mathbb{F}$, and $\forall \nvec \in \restbasis, \xvec \in \partial A$ $\phi_{\nvec}(\xvec) = 0$, then
	\begin{eqn*}
		\int\limits_A d\xvec \left(
			\nabla^2 \frac{\delta}{\delta \Psi}
		\right) \Psi \mathcal{F}[\Psi, \Psi^*]
		= \int\limits_A d\xvec \frac{\delta}{\delta \Psi}
		( \nabla^2 \Psi ) \mathcal{F}[\Psi, \Psi^*]
	\end{eqn*}
\end{lemma}
\begin{proof}
The proof consists of a function expansion into a mode sum and an application of Green's first identity.
\end{proof}

Note that the above lemma imposes an additional requirement for basis functions, but in practical applications it is always satisfied.
For example, in plane wave basis eigenfunctions are equal to zero at the border of the bounding box, and in harmonic oscillator basis they are equal to zero on the infinity (which can be considered the boundary of their integration area).
Hereinafter we will assume that this condition is true for any basis we work with.
